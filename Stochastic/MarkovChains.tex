\subsection{ Using Markov Chains to model the weather}

Suppose that the probabilities of weather conditions (modeled as either rainy or sunny), given the weather on the preceding day,
can be represented by a \textbf{\textit{transition matrix}}:

\[
    P = \begin{bmatrix}
        0.9 & 0.1 \\
        0.5 & 0.5
    \end{bmatrix}
\]

The matrix $P$ represents the weather model in which a sunny day is 90%
likely to be followed by another sunny day, and a rainy day is 50% likely to
be followed by another rainy day.  The columns can be labelled "sunny" and
"rainy", and the rows can be labelled in the same order.  

(''P'')<sub>''i j''</sub> is the probability that, if a given day is of type ''i'', it will be
followed by a day of type ''j''.

Notice that the rows of $P$P'' sum to 1,  this is because ''P'' is a \textbf{\textit{stochastic matrix}}.

\subsection{ Predicting the weather }

The weather on day 0 is known to be sunny.  This is represented by a vector in which the "sunny" entry is 100%, and the "rainy" entry is 0%:

[\
    \mathbf{x}^{(0)} = \begin{bmatrix}
        1 & 0
    \end{bmatrix}
]\

The weather on day 1 can be predicted by:

\[
    \mathbf{x}^{(1)} = \mathbf{x}^{(0)} P  = 
    \begin{bmatrix}
        1 & 0
    \end{bmatrix}
    \begin{bmatrix}
        0.9 & 0.1 \\
        0.5 & 0.5
    \end{bmatrix}
    
    = \begin{bmatrix}
        0.9 & 0.1
    \end{bmatrix} 
]\

Thus, there is a 90% chance that day 1 will also be sunny.

The weather on day 2 can be predicted in the same way:

[\
    \mathbf{x}^{(2)} =\mathbf{x}^{(1)} P  = \mathbf{x}^{(0)} P^2 
    = \begin{bmatrix}
        1 & 0
    \end{bmatrix}
    \begin{bmatrix}
        0.9 & 0.1 \\
        0.5 & 0.5
    \end{bmatrix}^2
    
    = \begin{bmatrix}
        0.86 & 0.14
    \end{bmatrix} 
\]
or
[\
    \mathbf{x}^{(2)} =\mathbf{x}^{(1)} P 
    = \begin{bmatrix}
        0.9 & 0.1
    \end{bmatrix}
    \begin{bmatrix}
        0.9 & 0.1 \\
        0.5 & 0.5
    \end{bmatrix}
    
    = \begin{bmatrix}
        0.86 & 0.14
    \end{bmatrix} 
]\

%----------------------------------------------------------------------------%

General rules for day ''n'' are:

: <math>
    \mathbf{x}^{(n)} = \mathbf{x}^{(n-1)} P 
</math>

: <math>
    \mathbf{x}^{(n)} = \mathbf{x}^{(0)} P^n 
</math>

=== Steady state of the weather ===

In this example, predictions for the weather on more distant days are increasingly
inaccurate and tend towards a [[steady state vector]].  This vector represents
the probabilities of sunny and rainy weather on all days, and is independent
of the initial weather.

The steady state vector is defined as:

<math>
    \mathbf{q} = \lim_{n \to \infty} \mathbf{x}^{(n)}
</math>

but only converges to a strictly positive vector if ''P'' is a regular transition matrix (that is, there
is at least one ''P''<sup>''n''</sup> with all non-zero entries).

Since the '''q''' is independent from initial conditions, it must be unchanged when transformed by ''P''.  This makes it an [[eigenvector]] (with [[eigenvalue]] 1), and means it can be derived from ''P''.  For the weather example:

<math>
    \begin{matrix}
        P & = & \begin{bmatrix}
            0.9 & 0.1 \\
            0.5 & 0.5
        \end{bmatrix}
        \\
       \mathbf{q} P  & = & \mathbf{q}
        & \mbox{(} \mathbf{q} \mbox{ is unchanged by } P \mbox{.)}
        \\
        & = & \mathbf{q}I 
        \\
       \mathbf{q} (P - I)  & = & \mathbf{0} \\
        & = & \mathbf{q} \left( \begin{bmatrix}
            0.9 & 0.1 \\
            0.5 & 0.5
        \end{bmatrix}
        -
        \begin{bmatrix}
            1 & 0 \\
            0 & 1
        \end{bmatrix}
        \right) 
        \\
        & = & \mathbf{q} \begin{bmatrix}
            -0.1 & 0.1 \\
            0.5 & -0.5
        \end{bmatrix} 
    \end{matrix}
</math>
<math>
     \begin{bmatrix}
        q_1 & q_2
    \end{bmatrix}
    \begin{bmatrix}
        -0.1 & 0.1 \\
        0.5 & -0.5
    \end{bmatrix}
    = \begin{bmatrix}
        0 & 0
    \end{bmatrix}
</math>

So 
<math>
    -0.1 q_1 + 0.5 q_2 = 0
</math>
and since they are a probability vector we know that 
<math>
q_1 + q_2 = 1.
</math>

Solving this pair of simultaneous equations gives the steady state distribution:

<math>
    \begin{bmatrix}
        q_1 & q_2
    \end{bmatrix}
    = \begin{bmatrix}
        0.833 & 0.167
    \end{bmatrix}
</math>

In conclusion, in the long term, about 83.3% of days are sunny.

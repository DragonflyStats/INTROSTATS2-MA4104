


Chapter 4
\chapter{CONTINUOUS RANDOM VARIABLES}

%% IN Tl-llS CHAPTERI


\begin{enumerate}
\item Continuous Random Variables
\item Continuous Probability Distribution
\item Distribution Functions for Continuous
Random Variables
\item Expected Values
\item Variance
\item Properties of Expected Values
and Variances
\item Graphical Interpretations
\end{enumerate}

\subsection*{Continuous Random Variables}
A nondiscrete random variable X is said to be absolutely continuous, or
simply continuous, if its distribution function may be represented as
% % 34
% % Copyright 2001 by the McGraW-Hill Companies, Inc. Click Hone for Tenns ofUse



% % - CHAPTER 4: Continuous Random Variables 35
F(X)=P(X$X)=l_f(M)<1M (1)
where the function fix) has the properties
l, f (x) 2 0
2. [foo dx =1

%===================================================================================================== %

\subsection*{Continuous Probability Distribution}

It follows from the above that if X is a continuous random variable, then the probability that X takes on any one particular value is zero, whereas the interval probability that X lies between two different values. say
a and b, is given by
:1
\[P(a<X<b)= \int_a^b f(x) dx (2)\]
b

Formally, if X is a continuous random variable, then it has a probability density function $f(x)$, and therefore its probability of falling into a given interval, say [a, b] is given by the integral


  \[  \Pr[a\le X\le b] = \int_a^b f(x) \, dx \]

%-----------------------------------------%

Because a probability distribution Pr on the real line is determined by the probability of a scalar random variable X being in a half-open interval $(-\infty, x]$, the probability distribution is completely characterized by its cumulative distribution function:

\[ F(x) = \Pr \left[ X \le x \right] \qquad \text{ for all } x \in \mathbb{R}.\]
%============================================================================================================================================== %

\subsubsection*{Example 4.1.}
If an individual were selected at random from a large
group of adult males, the probability that his height X is precisely 68
inches (ile., 68.000... inches) would be zero. However, there is a prob-
ability greater than zero that X is between 67.000“. inches and
68.000“. inches,

A function fix) that satisfies the above
requirements is called a probability function
or probability distribution for a continuous
random variable, but it is more often called
a probability density function or simply den-
sity function. Any function fix) satisfying the two properties above will
automatically be a density function, and required probabilities can be
obtained from (2).



% % -36 PROBABILITY AND STATISTICS
\subsubsection*{Example 4.2.} Find the constant c such that the function
. cxz 0 < x < 3
f(X) =
O otherwise
is a density function, and then find P(l < X < 2).
Notice that if c 2 O, then Property 1 is satisfied. So f(x) must satisfy
Property 2 in order for it to be a density function. Now
co 3 3
3
J-f(x)dx=_|-cxz dx=i =9c
—»<- 0 3 0
. . l . .
and since this must equal 1, we have c = 5 , and our density function
is
hz 0 < x < 3
foo = 9
O otherwise
Next,
P(kX<2)= —x dx=7 =7—7=7
27
%============================================================================================================================================== %

\subsection*{Distribution Functions for Continuous Random Variables}

Recall the cumulative distribution function, or distribution function, for
a random variable is defined by

% %- CHAPTER 4: Continuous Random Variables 37

F(x) = P(X § X) (3)
Where x is any real number, i.e,, —><> S x § ><>. So,
X
Fm = I foo ax <4)

Example 4.3. Find the distribution function for example 4.2‘
x x 3
F(X)= I/'(X)dX=[%x2d_x=;-7
»>< O
Where x S 3.
%============================================================================================================================================== %

There is a nice relationship between the distribution function and
the density function. To see this relationship, consider the probability
that a random variable X takes on a value, x, and a value fairly close to
x, say x + Ax.
The probability that X is between x and x + Ax is given by
X+A>C
P(xSXSx+Ax)= J fwd” (5)
X
so that if Ax is small, we have approximately
P(X s X s X + Ax) +f(x)Ax (6)
We also see from (1) on differentiating both sides that
@ = /<X> <7)
dx
at all points wheref(x) is continuous, i.e,, the derivative of the distribu-
tion function is the density function.

%==================================================================================================================== %

% % - 38 PROBABILITY AND STATISTICS
\subsubsection*{Expected Values}
If X is a continuous random variable having probability density function
flx), then it can be shown that
E[g<x>] = jg<x>_r<x>dx <8)
Example 4.4. The density function of a random X is given by
éx O<x<Z \%
/'<X>=  2%
O otherwise
The expected value of X is then
BC 2 2 3 2
x x 4
E(X)-Jxf(x)dx- £5xjdx_£?dx_?0_;
Q-€,,~
X
>-
%============================================================================================================================================== %

\subsubsection*{Variance}
If X is a continuous random variable having probability density function
f(x), then the variance is given by
vi = E[(X—~)*] = J <x—/N /(X) wt <9>
provided that the integral converges,

\subsubsection*{Example 4.5. }
Find the variance and standard deviation of the ran-
dom variable from Example 4.4, using the fact that the mean was found

% CHAPTER 4: Continuous Random Variables 39
tobe }1=E(X)=%.
1
“Helix-21 le (X-21 /Mi (X-2) lixldcé
N
X~—.x
N
x~—.x
_ _ _ 2 5
and so the standard deviation 1S (7 = 5 = T .

Recall that the variance (or standard deviation) is a measure of the
dispersion, or scatter, of the values of the random variable about the
mean pt, If the values tend to be concentrated near the mean, the vari-
ance is small; while if the values tend to be distributed far from the
mean, the variance is large. The situation is indicated graphically in
Figure 4-1 for the case of two continuous distributions having the same
mean ;1.
Small vnrinncc
Luge variance
#
Figure 4-1
%============================================================================================================================================== %

\subsection*{Properties of Expected Values and Variances}
In Chapter Three, We discussed several theorems that applied to expect-
ed values and variances of random variables. Since these theorems
apply to any random variable, we can apply them to continuous random
variables as well as their discrete counterparts‘


% %== 40 PROBABILITY AND STATISTICS
%============================================================================================================================================== %
\subsubsection*{Example 4.6.}
 Given the probability density function in Example
4.4, find E(3X) and Var(3X).
Using our the direct computational method,
K» 2 2 1 7
1 3 >
E(3X)= I 3X f(x)dx=_[31[5xjdx=J'5x2 <1x="-2— =4
—w 0 O 0
Using Theorems 3-l and 3-2, respectively, we could have found these
much easier as follows:
E(3X) = 3E(X) =  = 4
or
4 4 4
E(3X) = E(X+X+X) = E(X)+E(X)+E(X)=€+§+€ = 4
Using Theorem 3-5, the variance is also quite simple to find:
2 2
\[Var(3X) = 3^2Var(X) = \ldots \]
* Note!
These theorems aren’t just for show! They can make your
work much easier, so learn them and take advantage of them.

%------------------------------------------------------------------------------------------------------------ %

% % - CHAPTER 4: Continuous Random Variables 41
\subsection*{Graphical Interpretations}
If f(x) is the density function for a random variable X, then we can rep-
resent y :f(x) by a curve, as seen below in Figure 4-2. Since f(x) 2 0,
the curve cannot fall below the x-axis. The entire area bounded by the
curve and the x-axis must be 1 because of property 2 listed above.
Geometrically, the probability that X is between a and b, i.e.,
P(a < X < b), is then represented by the area shown shaded, in Figure
4-2.
/(I)
a b x
Figure 4-2
The distribution function F(x) = P(X £ x) is a monotonically
increasing function that increases from O to l and is represented by a
curve as in the following figure:
F (x)
I
X
Figure 4-3

%========================================================================%
\end{document}

\documentclass[a4]{beamer}
\usepackage{amssymb}
\usepackage{graphicx}
\usepackage{subfigure}
\usepackage{newlfont}
\usepackage{amsmath,amsthm,amsfonts}
%\usepackage{beamerthemesplit}
\usepackage{pgf,pgfarrows,pgfnodes,pgfautomata,pgfheaps,pgfshade}
\usepackage{mathptmx}  % Font Family
\usepackage{helvet}   % Font Family
\usepackage{color}

\mode<presentation> {
 \usetheme{Default} % was
 \useinnertheme{rounded}
 \useoutertheme{infolines}
 \usefonttheme{serif}
 %\usecolortheme{wolverine}
% \usecolortheme{rose}
\usefonttheme{structurebold}
}

\setbeamercovered{dynamic}

\title[MA4413]{Statistics for Computing \\ {\normalsize Lecture 1B}}
\author[Kevin O'Brien]{Kevin O'Brien \\ {\scriptsize Kevin.obrien@ul.ie}}
\date{Autumn Semester 2011}
\institute[Maths \& Stats]{Dept. of Mathematics \& Statistics, \\ University \textit{of} Limerick}

\renewcommand{\arraystretch}{1.5}

\begin{document}


\begin{frame}
\titlepage
\end{frame}

\frame{

\begin{enumerate}
\item Contingency Tables
\item Conditional Probability: Worked Examples
\item Joint Probability Tables
\item The Multiplication Rule
\item Law of Total Probability
\item Bayes' Theorem
\item Exam standard Probability Question
\item Random Variables
\end{enumerate}

}
%-------------------------------------------------------%
\frame{
\frametitle{Contingency Tables}
Suppose there are 100 students in a first year college intake.  \begin{itemize} \item 44 are male and are studying computer science, \item 18 are male and studying statistics \item 16 are female and studying computer science, \item 22 are female and studying statistics. \end{itemize}

We assign the names $M$, $F$, $C$ and $S$ to the events that a student, randomly selected from this group, is male, female, studying computer science, and studying statistics respectively.
}
%-------------------------------------------------------%
\frame{
\frametitle{Contingency Tables}
The most effective way to handle this data is to draw up a table. We call this a \textbf{\emph{contingency table}}.
A contingency table is a table in which all possible events (or outcomes) for one variable are listed as
row headings, all possible events for a second variable are listed as column headings, and the value entered in
each cell of the table is the frequency of each joint occurrence.


\begin{center}
\begin{tabular}{|c|c|c|c|}
  \hline
  % after \\: \hline or \cline{col1-col2} \cline{col3-col4} ...
    & C & S & Total \\ \hline
  M & 44 & 18 & 62 \\ \hline
  F & 16 & 22 & 38 \\ \hline
  Total & 60 & 40 & 100 \\ \hline
\end{tabular}
\end{center}

}
%-------------------------------------------------------%
\frame{
\frametitle{Contingency Tables}
It is now easy to deduce the probabilities of the respective events, by looking at the totals for each row and column.
\begin{itemize}
\item P(C) = 60/100 = 0.60
\item P(S) = 40/100 = 0.40
\item P(M) = 62/100 = 0.62
\item P(F) = 38/100 = 0.38
\end{itemize}
\textbf{Remark:}\\
The information we were originally given can also be expressed as:
\begin{itemize}
\item $P(C \cap M) = 44/100 = 0.44$
\item $P(C \cap F) = 16/100 = 0.16$
\item $P(S \cap M) = 18/100 = 0.18$
\item $P(S \cap F) = 22/100 = 0.22$
\end{itemize}
}
%-------------------------------------------------------%
\frame{
\frametitle{Joint Probability Tables}

A \textbf{\emph{joint probability table}} is similar to a contingency table, but for that the value entered in
each cell of the table is the probability of each joint occurrence. Often, the probabilities in such a table are based
on observed frequencies of occurrence for the various joint events.
\begin{center}
\begin{tabular}{|c|c|c|c|}
  \hline
  % after \\: \hline or \cline{col1-col2} \cline{col3-col4} ...
    & C & S & Total \\ \hline
  M & 0.44 & 0.18 & 0.62 \\ \hline
  F & 0.16 & 0.22 & 0.38 \\ \hline
  Total & 0.60 & 0.40 & 1.00 \\ \hline
\end{tabular}
\end{center}
}
%-------------------------------------------------------%
\frame{
\frametitle{Marginal Probabilities}
\begin{itemize}
\item In the context of joint probability tables, a  \textbf{\emph{marginal probability}} is so named because it is a marginal total of
a row or a column. \item Whereas the probability values in the cells of the table are probabilities of joint occurrence, the marginal
probabilities are the simple (i.e. unconditional) probabilities of particular events.
\item From the first year intake example, the marginal probabilities are $P(C)$, $P(S)$, $P(M)$ and $P(F)$ respectively.
\end{itemize}

}
%-------------------------------------------------------%
\frame{
\frametitle{Conditional Probabilities : Example 1}

Recall the definition of conditional probability:
\[ P(A|B) = \frac{P(A \cap B)}{P(B)} \]

Using this formula, compute the following:
\begin{enumerate}
\item $P(C|M)$ : Probability that a student is a computer science student, given that he is male.
\item $P(S|M)$ : Probability that a student studies statistics, given that he is male.
\item $P(F|S)$ : Probability that a student is female, given that she studies statistics.
\item $P(S|F)$ : Probability that a student studies statistics, given that she is female.
\end{enumerate}
Refer back to the contingency table to appraise your results.
}
%-------------------------------------------------------%
\frame{
\frametitle{Conditional Probabilities : Example 1}

\textbf{Part 1)} Probability that a student is a computer science student, given that he is male.
\[ P(C|M) = \frac{P(C \cap M)}{P(M)}  = \frac{0.44}{0.62} = 0.71 \]
\textbf{Part 2)} Probability that a student studies statistics, given that he is male.
\[ P(S|M) = \frac{P(S \cap M)}{P(M)}  = \frac{0.18}{0.62} = 0.29 \]

}

%-------------------------------------------------------%
\frame{
\frametitle{Conditional Probabilities : Example 1}

\textbf{Part 3)} Probability that a student is female, given that she studies statistics.
\[ P(F|S) = \frac{P(F \cap S)}{P(S)}  = \frac{0.22}{0.40} = 0.55 \]

\textbf{Part 4)} Probability that a student studies statistics, given that she is female.
\[ P(S|F) = \frac{P(S \cap F)}{P(F)}  = \frac{0.22}{0.38} = 0.58 \]


Remark: $P(S \cap F)$ is the same as $P(F \cap S)$.


}

Questions
Q1. Deltatech software has 350 programmers divided into two groups with 200 in Group A and 150 in Group B. In order to compare the efficiencies of the two groups, the programmers are observed for  1 day.
∙•••••••• The 200 programmers in Group A averaged 45.2 lines of code with a standard deviation of 8.4.
∙•••••••• The 150 programmers in Group B averaged 42.7 lines of code with a standard deviation of 5.2.


Let xa denote the average number of lines of code per day produced by programmers in Group A and
let xb be the corresponding quantity for Group B.
Provide an estimate of a — b and calculate an approximate 95% confidence interval for a — b .
Q2. Using the data in Q1, test the claim that Group A are more efficient than Group B by
    Interpreting the 95% confidence interval.
    Computing the appropriate test statistic.
    Computing the appropriate p-value.


Q3. A research company is comparing cars on the basis of fuel consumption.
Given the following data:
%==============================================================%


	&	Type A	&	Type B	\\	\hline
Sample Size	&	11	&	10	\\	\hline
Sample Meab	&	Mean 8.3miles/L	&	9.6miles/L	\\	\hline
Standard Deviation 	&	1.99m.iles/L	&	2.37miles/ L	\\	\hline


Test at a 0.05 level of significance that the fuel consumption is the same for both types of car.
Q4. Suppose that the sample sizes in Q3 were 110 and 100 respectively. Again test at a 0.05 level of significance that the fuel consumption is the same for both types of car.
Q5. The strength of concrete depends, to some extent, on the method used for drying. Two different methods showed the following results for independently tested specimens.  ( You may assume that there are equal variances).




%==============================================================%

(i) Does Method 1 appear to produce concrete with a greater mean strength? State your conclusions clearly.
(ii)Construct a 95% confidence interval for the difference between the two means. Interpret this interval.
Q6. Two procedures for sintering copper are compared by testing each procedure on six different types of powder. The measurement of interest is the porosity of each test specimen.
The results of the test are as follows:  



Powder	&	Procedure 1	&	 Procedure 2	\\	\hline
1	&	21	&	23	\\	\hline
2	&	27	&	26	\\	\hline
3	&	18	&	21	\\	\hline
4	&	22	&	24	\\	\hline
5	&	26	&	25	\\	\hline
6	&	19	&	16	\\	\hline




ls there a difference between the true average porosity measurements for the two procedures, at a significance level of 5%.
%==============================================================%
Q7. A coal-fired power plant is considering two different systems for pollution abatement. The first system has reduced the emission of pollutants to acceptable levels 68% of the time, as determined from 200 air samples. The second, more expensive system has reduced the emission of
pollutants to acceptable levels 70% of the time, as determined from 250 air samples. lf the expen sive system is significantly more eilective than the inexpensive system in reducing the pollutants to acceptable levels, then the management of the power plant will install the expensive system.

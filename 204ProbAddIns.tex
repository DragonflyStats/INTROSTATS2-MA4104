\documentclass[]{article}

\usepackage{framed}
\begin{document}


%-----------------------------------------------------------%
\section{Gambler's Ruin - Probability Distribution of Duraction}
Construct a probability distribution of the Durations


\begin{itemize}
\item How likely is it that game lasts 50 rounds?
\item How likely is it that game lasts 100 rounds?
\item How likely is it that game lasts 1000 rounds?
\end{itemize}

How do we go about thus? Run the procedure a really large number of times.

\begin{framed}
\begin{verbatim}
#Let see how long this simulation takes
Time1 <- Sys.time()

# Set up an empty vector 
DurDist <-numeric()

# Set the number of trials
M <- 100000

for i in 1:m
   { 
   GambRuinFunc() 
   DurDist=c(DurDist,Dur)
   }

#simulation is completed   
Time2 <- Sys.time()

#summary of distribution
summary(DurDist)

diff(Time1,Times2)
\end{verbatim}
\end{framed}
%-----------------------------------------------------------%
%---------------------------------------------------------------------------- %
\section{Monty Hall Add Ins}

If you know the answer - keep quiet!
%PICTURES
The Monty Hall problem is a counter-intuitive statistics puzzle:

\begin{itemize}
\item There are 3 doors, behind which are two goats and a car.
\item You pick a door (call it door A). You’re hoping for the car of course.
\item Monty Hall, the game show host, examines the other doors (B and C) and always opens one of them with a goat (Both doors might have goats; he’ll randomly pick one to open) 
\end{itemize}
Here’s the game: Do you stick with door A (original guess) or switch to the other unopened door? Does it matter?
\newpage
%----------------------------------------------------------------------------- %
%REVEAL
\section{Monty Hall - the Aswer}
Surprisingly, the odds aren’t 50-50. If you switch doors you’ll win 2/3 of the time!

Today let’s get an intuition for why a simple game could be so baffling. The game is really about re-evaluating your decisions as new information emerges.
%----------------------------------------------------------------------------- %

\end{document}

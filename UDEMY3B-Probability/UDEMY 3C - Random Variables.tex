
\documentclass{beamer}

\usepackage{amsmath}
\usepackage{amssymb}

\begin{document}
\section{Random variables}
\begin{frame}
\frametitle{Random Variables}
\vspace{-1cm}
\Large
\begin{itemize}
\item A random variable is defined as a numerical event whose value is determined by a chance process.
\item When probability values are assigned to all possible numerical values of a random variable $X$, either by a listing
or by a mathematical function, the result is a \textit{\textbf{probability distribution}}. 
\end{itemize}
\end{frame}
%--------------------------------------------------------- %
\begin{frame}
\frametitle{Random Variables}
\Large
\vspace{-1cm}
\begin{itemize}
\item The sum of the probabilities for all the possible numerical outcomes must equal one. 
\item Individual probability values may be denoted by the symbol $f(x)$,
which indicates that a mathematical function is involved, by $P(X=x)$, which recognizes that the random
variable can have various specific values, or simply by P(x).
\end{itemize}
\end{frame}
%--------------------------------------------------------- %
\section{Discrete random variables}

\begin{frame}
\frametitle{Discrete random variables}
\Large
\vspace{-0.7cm}
\textbf{Discrete Random Variables}
\begin{itemize}
\item For a discrete random variable observed values can occur only at isolated points along a scale of values. In other words, observed values must be integers.
\item Consider a six sided die: the only possible observed values are 1, 2, 3, 4, 5 and 6. 
\item It is not possible to observe values that are real numbers, such as 2.091.
\large
\item \textit{(Remark: it is possible for the average of a discrete random variable to be a real number.)}
%\item Therefore, it is possible that all numerical values for the variable can be listed in a table with accompanying
%probabilities. 
%\item
%There are several standard probability distributions that can serve as models for a wide variety of discrete random variables involved in business applications. 
\end{itemize}
\end{frame}
\begin{frame}
\frametitle{Discrete random variables}
\Large
\vspace{-1cm}
\textbf{Discrete Random Variables}
\begin{itemize}
%\item For a discrete random variable observed values can occur only at isolated points along a scale of values. 
%\item Consider a six sided die: the only possible observed values are 1, 2, 3, 4, 5 and 6. It is not possible to observe values such as 5.732.
\item Therefore, it is possible that all numerical values for the variable can be listed in a table with accompanying
probabilities. 
\item
There are several standard probability distributions that can serve as models for a wide variety of discrete random variables involved in business applications. 
\end{itemize}
\end{frame}
%--------------------------------------------------------- %
%--------------------------------------------------------- %
\begin{frame}
\frametitle{Discrete probability distributions}
\Large
\vspace{-1cm}
The discrete probability distributions that described in this course are
\begin{itemize}
\item the binomial distribution, 
\item the geometric distribution,
\item the hypergeometric distribution, 
\item the Poisson distributions.
\end{itemize}
\end{frame}

%\\
%For a continuous random variable all possible fractional values of the variable cannot be listed, and
%therefore the probabilities that are determined by a mathematical function are portrayed graphically by a
%probability density function, or probability curve.
%\end{frame}
\end{document}




\documentclass{beamer}

\usepackage{amsmath}
\usepackage{amssymb}

\begin{document}

\section{Basic Definitions of probability}
%---------------------------------------------------------------------------------%
\begin{frame}
\frametitle{Probability}
\Large
\begin{itemize}
\item The symbol P is used to designate the probability of an event. Thus P(A) denotes the probability that event
A will occur in a single observation or experiment.

\item The smallest value that a probability statement can have is 0 (indicating the event is impossible) and the
largest value it can have is 1 (indicating the event is certain to occur). 
\item Thus, in general:
$0 	\leq P(A) \leq 1$
\end{itemize}
\end{frame}

%---------------------------------------------------------------------------------%
\begin{frame} 
\frametitle{Mutually exclusive events}
\Large
\begin{itemize}
\item In a given observation or experiment, an event must either occur or not occur. \item Therefore, the sum of the
probability of occurrence plus the probability of nonoccurrence always equals 1. \item Thus, where $A^{\prime}$ indicates the nonoccurrence of event A, we have
$P(A) + P(A^{c}) =  1$
\end{itemize}
\end{frame}
%---------------------------------------------------------------------------------%
\begin{frame} 
\frametitle{Mutually exclusive events}
\Large
\begin{itemize}
\item 
Two or more events are mutually exclusive, or disjoint, if they cannot occur together. That is, the occurrence
of one event automatically precludes the occurrence of the other event (or events). 
\item For instance, suppose we
consider the two possible events ``ace" and ``king" with respect to a card being drawn from a deck of playing
cards. 
\item These two events are mutually exclusive, because any given card cannot be both an ace and a king.
\item Two or more events are nonexclusive when it is possible for them to occur together.
\end{itemize} 
\end{frame}
%---------------------------------------------------------------------------------%
\begin{frame} 
\Large
\begin{itemize}
\item Note that this definition does not indicate that such events must necessarily always occur jointly. For instance, suppose we consider the two possible events ``ace" and ``spade". 
\item These events are not mutually exclusive, because a given card can be both an ace and a spade; however, it does not follow that every ace is a spade or every spade is an ace.
\end{itemize}
\end{frame}
%---------------------------------------------------------------------------------%
\begin{frame}
\frametitle{General rule of addition}
\begin{itemize}
\item For events that are not mutually exclusive, the probability of the joint occurrence of the two events is
subtracted from the sum of the simple probabilities of the two events. We can represent the probability of joint
occurrence by P(A and B).\item  In the language of set theory this is called the intersection of A and B and the
probability is designated by P(A and B).  Thus, the rule of addition for events that are not mutually exclusive is
\[ P(A \mbox{ or }B) = P(A)+ P(B) - P(A \mbox{ and }B)\]
\end{itemize}
\end{frame}
%---------------------------------------------------------------------------------%
\begin{frame} 
\frametitle{Example}
\Large
When drawing a card from a deck of playing cards, the events ``ace" and ``spade" are not mutually
exclusive. The probability of drawing an ace (A) or spade (S) (or both) in a single draw is
\begin{eqnarray} P(A \mbox{ or }B) &=& P(A) + P(S) - P(A \mbox{ and }B)\\ &=& 4/52 + 13/52 -1/52 \\&=& 16/52 \\
&=& \textbf{4/13} 
\end{eqnarray}
\end{frame}
%---------------------------------------------------------------------------------%
\section*{Independent events}
\begin{frame}
\Large
\begin{itemize}
\item Two events are independent when the occurrence or nonoccurrence of one event has no effect on the
probability of occurrence of the other event. 

\item Two events are dependent when the occurrence or nonoccurrence
of one event does affect the probability of occurrence of the other event.
\end{itemize}

\end{frame}
%---------------------------------------------------------------------------------%


\section*{Conceptual approaches}
\begin{frame}
\Large
\begin{itemize}
\item Historically, three different conceptual approaches have been developed for defining probability and for
determining probability values: the classical, relative frequency, and subjective approaches.\item If N(A) possible elementary outcomes are favorable to event A,
N(S) possible outcomes are included in the sample space, and all the elementary outcomes are equally likely and
mutually exclusive, then the probability that event A will occur is
\[P(A) = \frac{N(A)}{N(S)}\]
\end{itemize}
\end{frame}
\begin{frame} \frametitle{Examples}
\Large
When a fair dice is thrown, what are the possible outcomes? There are 6 possible outcomes. 

The dice can role any number between one and six. Each outcome is equally likely. The probability of each outcome is 1/6.
\end{frame}
%---------------------------------------------------------------------------------%
\begin{frame} 

\Large
In a well-shuffled deck of cards which contains 4 aces and 48 other cards, the probability of an ace (A)
being obtained on a single draw is;
\[ P(A)= N(A)/ N(S) = 4/52 = 1/13 \]
\end{frame}
%---------------------------------------------------------------------------------%
\begin{frame} 

\frametitle{Bayes’ theorem}
\Large
In its simplest algebraic form, Bayes’ theorem is concerned with determining the conditional probability of
event A given that event B has occurred. 

The general form of Bayes’ theorem is
\[ P(A|B) =
\frac{P(A \mbox{ and }B)}{P(B)} \]
\end{frame}
%---------------------------------------------------------------------------------%
\end{document}

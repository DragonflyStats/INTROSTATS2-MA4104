%============================================================================================================ %
\section*{Chapter 5-EXAMPLES OF RANDOM VARIABLES}

\begin{itemize}
\item Binomial Distribution
\item Properties of Binomial Distributions
\item The Normal Distribution
\item Examples of the Normal Distribution
\item Poisson Distributions
\item Relationships between Binomial and
Normal Distributions
\item Relationships between Binomial and
Poisson Distributions
\item Relationships between Poisson and
Normal Distributions
\item Central Limit Theorem
\item Law of Large Numbers
\end{itemize}
% % PAge 42
% % - Copyright 2001 by the McGraW-Hill Companies, Inc. Click Hem for Terms of Use


%%-CHAPTER 5: Examples of Random Variables 43
%============================================================================================================================================== %
\newpage
\section*{Binomial Distribution}
Suppose that we have an experiment such as tossing a coin or die repeatedly or choosing a %
marble from an urn repeatedly. Each toss or selection is called a trial. In any single trial T‘
there will be a probability associated with a particular event such as head on the coin, four
on the die, or selection of a specific color marble. 
%============================================================================================================================================== %
\begin{itemize}
\item In some cases this probability will not
change from one trial to the next (as in tossing
a coin or die), Such trials are then said to be independent and are often
called Bernoulli trials after James Bernoulli who investigated them at
the end of the seventeenth century,
\item Let p be the probability that an event will happen in any single
Bernoulli trial (called the probability ofsuccess). Then q = l —p is the
probability that the event will fail to happen in any single trial (called
the probability of failure).
\end{itemize}

 The probability that the event will happen
exactly x times in n trials (i.e., x successes and n — x failures will occur)
is given by the probability function
f<x) = P(X= x) =£nj 1131"“ =,LP‘q"" (I)
x x.(n—x)!
where the random variable X denotes the number of successes in n tri-
als andx= O, l,  n.
%============================================================================================================================================== %

\subsubsection*{Example 5.1. }
The probability of getting exactly 2 heads in 6 tosses
of a fair coin is
2 b—2 2 4
Wtilei ta I as ta 

The discrete probability function f(x) is often called the binomial distri-
bution since x = O, l, 2,  , n, it corresponds to successive terms in the
binomial expansion


% %- 44 PROBABILITY AND STATISTICS
(q+p)i, =6/1 +[Uqn—1p+[;jq»—2p2 +.,.+pn = ifljpxqnq (2)
><
c
The special case of a binomial distribution With n = l is also called the
Bernoulli distribution.
%=================================================================== %
\subsection*{Properties of Binomial Distributions}
As with other distributions, we would like to know the descriptive sta-
tistics for the binomial distribution. They are as follows:
Mean $\mu=np$
Variance $\sigma^2 = np (1 — p)$
Standard Deviation $\sigma = \sqrt{np(l — p)}$
%============================================================================================================================================== %

\subsubsection*{Example 5.2.} 
Toss a fair coin 100 times, and count the number of
heads that appear. Find the mean, variance, and standard deviation of
this experiment.

In 100 tosses of a fair coin, the expected or mean number of heads is
\[ \mu = E(X) = 100 \times 0.5 = 50\]

The variance is found to be
\[ \sigma^2 = 100 \times 0.5 \times 0.5 = 25\]

This means the standard deviation is

\[ \sigma = \sqrt{100 \times 0.5 \times 0.5 } = sqrt{25} = \mbox{5 units}\]


% % = CHAPTER 5; Examples of Random Variables 45
%============================================================================================================================================== %
\newpage
\subsection*{The Normal Distribution}

One of the most important examples of a continuous probability distribution is the normal distribution, sometimes called the Gaussian distribution. 
The density function for this distribution is given by
1 -(.t-nil/262
= 7 — M < < cc 3
f(X) (“[2-7; e x ( )

where $\mu$ and $\sigma$ are the mean and standard deviation, respectively. The
corresponding distribution function is given by
F(x) = P(X s X) =   I e-(t._ll)2/26: dv (4)

If X has the distribution function listed above, then we say that the
random variable X is normally distributed with mean |.1 and variance 62.

If we let Z be the random variable corresponding to the following
Z3 (5)
0'
then Z is called the standard variable corresponding to X4 The mean or
expected value of Z is O and the standard deviation is l. In such cases
the density function for Z can be obtained from the definition of a nor-
mal distribution by allowing /1 = 0 and 0'2 : 1, yielding
/(Z) = fie” <6)

This is often referred to as the standard normal density functions
The corresponding distribution function is given by



% % - 46 PROBABILITY AND STATISTICS
F(z) = P(ZS z)=* ‘” ” =e j “M du (7)
Q _
Fl
Z._,.
(B
'§“
N _
+
Q _
Fl
O_..
W
We sometimes call the value z of the standardized variable Z the
standard score.
%======================================================== %
A graph of the standard normal density function, sometimes called
the standard normal curve, is shown in Figure 5-l. In this graph we
have indicated the areas Within 1, 2, and 3 standard deviations of the
mean (i.e., between z = -1 and +1, z = -2 and +2. Z = -3 and +3) as
equal, respectively, to 68.27\%, 95.45\%, and 99.73\% of the total area,
which is one. This means that
\begin{verbatim}
    P(—l szs l)=0.6827
	P <—2 s z s 2) = 0.9545
	P (-3 s zs 3) = 0.9973
\end{verbatim}

* Note!
The normal distribution is very important! It will quite often
come up in practice, so make sure you understand how to
use this distribution.

%========================================================================================================= %
%%-CHAPTER 5: Examples of Random Variables 47
1(1)
it
G
|
-1 -i -| n | 2 1 z
~— osrm —-
-——— 95.45%
‘ 99. 73% :
Figure 5-1
%--------------------------------------%
A table giving the areas under the curve bounded by the ordinates
at z = O and any positive value of z is given in Appendix B. From this
table the areas between any two ordinates can be found by using the
symmetry of the curve about Z = O.

%========================================================================================================= %

\subsubsection*{Examples of the Normal Distribution}
Since this distribution is so important, we will now run through a few
examples of how to use the distribution.

\subsubsection{Example 5.3.} 
Find the area under the standard normal curve between Z = O and Z = 1.2,

Using the table in Appendix B, proceed down the column marked
Z until entry 1.2 is reached. Then proceed right to column marked O. The
result, 0.3849, is the required area and represents the probability that Z
is between O and 1,2. Therefore, P(O S Z S 1.2) = 0,3849.


% % 48 PROBABILITY AND STATISTICS
%============================================================================================================================================== %

Example 5.4. Find the area under the standard normal curve
between z = —0.46 and z : 2.21.
I-O I-I1
Figure 5-2
Consider the following picture of the density curve.
0.46 2 I
- J
Figure 5-3
The required area can be broken down into two parts. First, the area
between z = -0.46 and z = 0, and secondly, the area between Z = 0 and
z=0andz=2.2I4
%======================================================================================================================= %
Since the nonnal curve is symmetric, the area between z = —O,46
and Z = 0 is the same as the area between z = O and Z = Q46. Using
Appendix B, we can see that this area is 0417721 In other words,
P(-046 s z s 0 = P(0 s z s 046) : 01772
%----------------------------------%
Using Appendix B to find the area between z = 0 and Z = 2.21 is
found to be 0.4864 This means
\[P(0 \leq Z \leq 2.21) = 0.4864\]
This allows us to determine the required area as follows:

%%-CHAPTER 5; Examples of Random Variables 49

Total Area = (area between z = -0.46 and z = 0) +
(area between z = O and z : 2.21)
= 0.1722 + 0.4864
= 0.6586
Therefore P(%).46 S Z S 2.21) = 0.6636.
\subsection{Example 5.5.} 
The mean Weight of 500 male students at a certain
college is 151 lb and the standard deviation is 15 lb. Assuming the
weights are normally distributed, find how many students weigh (a)
between 120 and 155 lb, (b) more than 185 lb.
(a) If weights are recorded to the nearest pound, then weights recorded
as being between 120 and 155 lb can actually have any value from 119.5
to 155.5 lb.
We need to find the standard scores for 119.5 and 155.5.
119.5 lb in standard units : (119.5 -151)/15
= -2.10
155.5 1b in standard units = (155.5 —15l)/15
= 0.30
4.10 no
Figure S-4

%==================================================================================================================== %

50 PROBABILITY AND STATISTICS
\begin{verbatim}
Required proportion of students
= (area between z = —2.l0 and z = 030)
= (area between z = -2.10 and z = O)
+ (area between z = 0 and Z = 0.30)
= 0.4821 + 0.1179
= 0.6000
\end{verbatim}

This means that of the 500 male students polled, 60% of them weigh
between 120 and 155 1b. Then the number of students in this range is
(500)(0A6000) = 300‘
(b) Notice that students Weighing more than 185 lb must Weigh at least
185.5 lb.
185.5 lb in standard units = (185.5 — 151)/15
: 2.30
1J0
%-----------------------------------------------------------------------------------%
Figure 5-5
Required proportion of students
= (area to the right of z = 2.30)
= (area to the right of z = 0)
— (area between z = 0 and Z = 2.30)
= 0,5 — 0,4893
= 00107
Then the number weighing more than 185 1b is (500)(0,0107) = 5.



% % - CHAPTER 5; Examples of Random Variables 51
If $W$ denotes the Weight of a student chosen at random, we can summarize the above results in terms of probability by writing
P(119,5 2 W§155.5 = 0.6000 P(W2185.5)= 0.0107
%============================================================================================================================================== %
\subsection*{Poisson Distributions}

LetX be a discrete random variable that can take on the values 0, 1, 2,. ..
such that the probability function of X is given by
x —?t
f(x)=P(x=x)=}“—)6e_— x=O,1,2,... (8)
Where K is a given positive constant. This distribution is called the
Poisson distribution (after S, D. Poisson, who discovered it in the early
part of the nineteenth century), and a random variable having this dis-
tribution is said to be Poisson distributed,
The values of the Poisson distribution can be obtained by using
Appendix F, which gives values of fl for various values of 7».
%============================================================================================================================================== %

\subsection{Example 5.6.}
If the probability that an individual will suffer a bad
reaction from injection of a given serum is 0.001, determine the proba-
bility that out of 2000 individuals, (a) exactly 3, (b) more than 2, indi-
viduals will suffer a bad reaction.

Let X denote the number of individuals suffering a bad reaction. X
is Bernoulli distributed, but since bad reactions are assumed to be rare
events, We can suppose that X is Poisson distributed, i.e,,
.\ )-A
P(X= x) = 1; where Z. = np = (2000)(0,00l) = 2

%%-52 PROBABILITY AND STATISTICS

(21)
3 -2
P<x=3>=%=0.1s0
(b)
P(X>2) = l—[P(X=0)+P(X=l)+P(X=2)]
2024 2'e'2 22e'2
= l— —+—+—
0! l! 2!
= l —5e‘2
= 0.323
An exact evaluation of the probabilities using the binomial distrib-
ution would require much more labor.
%----------------------------------------------------------------------------%
\section{Relationships between Binomial and Normal Distributions}
Ifn is large and if neither p nor q is too close to zero, the binomial dis-
tribution can be closely approximated by a normal distribution with
standardized random variable given by
z = ixwp (9)
\/ mm
%=========================================================================================== %

Here X is the random variable giving the number of successes in n Bernoulli trials and p is the probability of success. The approximation
becomes better with increasing n and is exact in the limiting case. In
practice, the approximation is very good if both np and nq are greater
than 5. The fact that the binomial distribution approaches the normal
distribution can be described by Writing

%%-CHAPTER 5; Examples of Random Variables 53

n—>~ , np
- X’"P 1 I7 —|42/2
limP afiaib =7 e du (10)
l q l Pl

In words, we say that the standardized random variable (X—np)/
1 npq is asymptotically normal.

\subsection{Example 5.7.} 
Find the probability of getting between 3 and 6 heads inclusive in 10 tosses of a fair coin by using 
\begin{itemize}
\item[(a)] the binomial distribution 
\item[(b)] the normal approximation to the binomial distribution.
\end{itemize}
(a) Let X have the random variable giving the number of heads that
will turn up in 10 tosses. Then
P<»<=»=(‘§]£%f(;)¥§ P<»~»=(‘§1<a‘@11;'%z
5 5 6 4
P<X=»;i*§)@> a  P<»<:@>;i*:)(;> ta 
Then the required probability is
P(3§X56)=%+£+%+$=0.7734
(b) Treating the data as continuous, it follows that 3 to 6 heads can be
considered 25 to 6.5 heads. Also, the mean and the variance for the
binomial distribution is given by ,u=np=(l0)(%]=5 and
if
0' =1/npq =\/(10)[5X5) =1.5s.

%%-54 PROBABILITY AND STATISTICS

2.5 in standard units = 2'5 75 = —l .58
1,58
6.5 in standard units = 65-5 = 0.95
1.58
-I38 09$
Figure 5-6
Required probability = (area between z = —l.58 and z = 0.95)
= (area between z = —l .58 and z = 0)
+ (area between z = 0 and Z = 0.95)
= 0.4429 + 0.3289
= 0.7718
which compares very well with the true value 07734 obtained in part
(a), The accuracy is even better for larger values of n.

%==================================================================================%
\section{Relationships between Binomial and Poisson Distributions}
In the binomial distribution, if n is large while the probability p of
occurrence of an event is close to zero, so that q = l — p is close to one,
the event is called a rare event. In practice, We shall consider an event
as rare if the number of trials is at least 50 (n 2 50) while np is less than
5. For such cases, the binomial distribution is very closely approximat-
ed by the Poisson distribution with /I = np. This is to be expected on

%--CHAPTER 5; Examples of Random Variables 55
comparing the equations for t_he means and
variances for both distributions. By substituting /I = np, q = l and p = 0 into the equa-
tions for the mean and variance of a bino-
mial distribution, we get the results for the
mean and variance for the Poisson distribu-
tion.
%================================================================================================ %
\section{Relationships between Poisson and Normal Distributions}
Since there is a relationship between the binomial and normal distribu-
tions and between the binomial and Poisson distributions, we would
expect that there should be a relation between the Poisson and normal
distributions, This is in fact the case. We can show that if X is the fol-
lowing Poisson random variable
x —7t
f<x>=’l—x% <11)
and
X-/I
TT ('2)
is the corresponding standardized random variable, then
x /1 1 " ~
1' P _L_b =: *“"2d 13
K111 (a< W/I <1 27752 u ( )
i.e., the Poisson distribution approaches the normal distribution as
/I —> ><> or (X—/.1)/~/I is asymptotically normal.


%=============================================================================================================== %
%%-56 PROBABILITY AND STATISTICS
\section{Central Limit Theorem}
The similarities between the binomial, Poisson, and normal distributions naturally lead us to ask whether there are any other distributions besides the binomial and Poisson that have the normal distribution as
the limiting case. The following remarkable theorem reveals that actually a large class of distributions have this property.
Theorem 5-1: (Central Limit Theorem) Let X1, X2,..,, X" be inde-
pendent random variables that are identically distrib-
uted (i.e., all have the same probability function in the
discrete case or density function in the continuous
case) and have finite mean rt and variance ($2. Then if
S"=X1+XZ+---+XH (n=l,Z,..,),
b
lim P(aSS"%l/JSb)=%{e'“'/2 du (I4)
n—>><- 0' n \/21:
that is, the random variable (Sn —n/1)/0"/Z , which
is the standardized variable corresponding to S", is
asymptotically normal.
The theorem is also true under more general conditions; for exam-
ple, it holds when X1, X2,  X“ are independent random variables with
the same mean and the same variance but not necessarily identically
distributed.
\section{Law of Large Numbers}
Theorem 5-2: (Law of Large Numbers) Let xl, xz,  m be mutu-
ally independent random variables (discrete or con-
tinuous), each having finite mean tt and variance <52.
Then ifS" =X1 +X2+  +XH (n = 1, 2, ...),



CHAPTER 5; Examples of Random Variables 57
limP£S—”—p2ej=0 (15)
n—>><» ll
Since S"/n is the arithmetic mean of XI + X2 +  + X", this theo-
rem states that the probability of the arithmetic mean Sn/n differing
from its expected value |,t by more than 2 approaches zero as n —> <><>, A
stronger result, which we might expect to be true, is that lim 3" /n : /1,
Vl—>\><'
but that is actually false. However, We can prove that 11m Sn /H = y
Il~>>=
with probability one. This result is often called the strong law of large
numbers, and by contrast, that of Theorem 5-2 is called the weak law of
large numbers
%=============================================================================================================== %



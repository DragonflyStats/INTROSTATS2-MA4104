Bayesian analysis is the branch of statistics based on the idea that we have some knowledge in advance about the probabilities that we are interested in, so called a priori probabilities. This might be your degree of belief in a particular event, the results from previous studies, or a general agreed-upon starting value for a probability. The terminology "Bayesian" comes from the Bayesian rule or law, a law about conditional probabilities. The opposite of "Bayesian" is sometimes referred to as "Classical Statistics."
\end{frame}
%----------------------------------------------------------------- %
\begin{frame}
\frametitle{Example}
Consider a box with 3 coins, with probabilities of showing heads respectively 1/4, 1/2 and 3/4. We choose arbitrarily one of the coins. Hence we take 1/3 as the a priori probability P(C_1) of having chosen coin number 1. After 5 throws, in which X=4 times heads came up, it seems less likely that the coin is coin number 1. We calculate the a posteriori probability that the coin is coin number 1, as:

P(C_1|X=4)=\frac{P(X=4|C_1)P(C_1)}{P(X=4)}=\frac{P(X=4|C_1)P(C_1)}{P(X=4|C_1)P(C_1)+P(X=4|C_2)P(C_2)+P(X=4|C_3)P(C_3)}=
\frac{ {5 \choose 4} (\frac 14)^4 \frac 34 \frac 13}
{{5 \choose 4} (\frac 14)^4 \frac 34 \frac 13+{5 \choose 4} (\frac 12)^4 \frac 12 \frac 13+{5 \choose 4} (\frac 34)^4 \frac 14 \frac 13}=
In words:
\end{frame}
%----------------------------------------------------------------- %
\begin{frame}
The probability that the Coin is the first Coin, given that we know heads came up 4 times... Is equal to the probability that heads came up 4 times given we know it's the first coin, times the probability that the coin is the first coin. All divided by the probability that heads comes up 4 times (ignoring which of the three Coins is chosen). The binomial coefficients cancel out as well as all denominators when expanding 1/2 to 2/4. This results in

\[\frac{3}{3 + 32 + 81}=\frac{3}{116}\]
In the same way we find:
\[
P(C_2|X=4)=\frac{32}{3 + 32 + 81}=\frac{32}{116}\]
and
\[
P(C_3|X=4)=\frac{81}{3 + 32 + 81}=\frac{81}{116}.\]
\end{frame}
%----------------------------------------------------------------- %
\begin{frame}
This shows us that after examining the outcome of the five throws, it is most likely we did choose coin number 3.

Actually for a given result the denominator does not matter, only the relative Probabilities p(C_i) = P(C_i|X=4)/P(X=4) When the result is 3 times heads the Probabilities change in favor of Coin 2 and further as the following table shows:

Heads 	p(C_1)	p(C_2)	p(C_3)
5	& 1	32	243
4	& 3	32	81
3	& 9	32	27
2	& 27	32	9
1	& 81	32	3
0	& 243 &	32	1

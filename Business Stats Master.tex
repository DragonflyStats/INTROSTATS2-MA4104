\documentclass[a4paper,12pt]{article}
%%%%%%%%%%%%%%%%%%%%%%%%%%%%%%%%%%%%%%%%%%%%%%%%%%%%%%%%%%%%%%%%%%%%%%%%%%%%%%%%%%%%%%%%%%%%%%%%%%%%%%%%%%%%%%%%%%%%%%%%%%%%%%%%%%%%%%%%%%%%%%%%%%%%%%%%%%%%%%%%%%%%%%%%%%%%%%%%%%%%%%%%%%%%%%%%%%%%%%%%%%%%%%%%%%%%%%%%%%%%%%%%%%%%%%%%%%%%%%%%%%%%%%%%%%%%

\usepackage{vmargin}
\usepackage{amsmath}
\usepackage{graphics}
\usepackage{epsfig}
\usepackage{subfigure}
\usepackage{fancyhdr}
%\usepackage{listings}
\usepackage{framed}
\usepackage{graphicx}
\usepackage{amsmath}
\usepackage{amssymb}
\usepackage{multicol}
\setcounter{MaxMatrixCols}{10}


\pagestyle{fancy}
\setmarginsrb{20mm}{0mm}{20mm}{25mm}{12mm}{11mm}{0mm}{11mm}
\lhead{Maths-Resource.com} \rhead{Mr. Kevin O'Brien}
\chead{Business Statistics}


\begin{document}

\large

%-------------------------------------------------%

\section*{May 2012 Question 2 Correlation and Regression}
\begin{itemize}
\item The sample size $n$ = 10.
\item The \textbf{\textit{independent}} variable, usually denoted $x$, is the "cause variable" or "predictor variable".
\item The \textbf{\textit{dependent}} variable, usually denoted $y$, is the "effect variable".
\item Here the Maths achievement test score is the independent variable and the final grade in statistics is the dependent variable.
\item A big hint is given in the notation of the question.
\end{itemize}

% X = c(39,43,21,64,57,47,28,75,34,52)
% Y = c(65,78,52,82,92,89,73,98,56,75)

%---------------------------------------------------------- %
\subsection*{Sums of Squares Identities}
Before we do anything, We need to compute the following sums of squares identities
\begin{itemize}
\item $s_{xx}$
\item $s_{yy}$
\item $s_{xy}$
\end{itemize}
\subsection*{Calculation 1}
\[ s_{xx}  = \sum(x^2) - \frac{\sum(x)^2}{n} \]
\[ s_{xx}  = 23.634 - \frac{\sum(x)^2}{10} \]

\subsection*{Calculation 2}
\[ s_{yy}  = \sum(y^2) - \frac{\sum(y)^2}{n} \]
%\[ s_{xx}  = 23.634 - \frac{\sum(x)^2}{10} \]


\subsection*{Calculation 3}
\[ s_{xy}  = \sum(xy) - \frac{\sum(x)\times \sum{y}}{n} \]
%\[ s_{xx}  = 23.634 - \frac{\sum(x)^2}{10} \]

%------------------------------------------------------------%

\subsection*{Part iv - Prediction}
\begin{itemize}
	\item Suppose the regression equation is as follows
	\[ \hat{y} = 40.78424 + 0.76556 x \]
	\item If a student scored 5 marks on the achievement test (i.e. $x=5$), predict the students statistics grade.
	
	\[ \hat{y}_{(x=5)} = 40.78424 + (0.76556 \times 5) \]
	
	\item Solving using a calculator we get
	\[ \hat{y}_{(x=5)} = 44.61204 \]
	
	\item The score should be approximately 44.61.
\end{itemize}

%-------------------------------------------------%
\newpage

\section*{May 2013 Question 6b Correlation and Regression }
Calculate the correlation coefficient and interpret the value.
\begin{tabular}{|c|c|c|}
	\hline Residence	& X	  & Y \\ 
	\hline  &  &  \\ 
	\hline  &  &  \\ 
	\hline 
\end{tabular} 
\newpage
%-------------------------------------------------%


\section*{May 2012 Question 4 Normal Distribution / Theory}

\subsection*{Revision of Normal Distribution}
\subsection{Important rules for normal distribution}

\begin{itemize}

\item \textbf{Complement Rule}\\
For some value $A$, and for any continuous distribution $X$ (including any normal distribution and the Z distribution) we can say.
\[ P(X \leq a) = 1 - P(X \geq A) \]

\item \textbf{Symmetry Rule}\\
For the standard normal ($Z$) distribution only, we can say
 \[ P(Z \leq -A) = P(Z \geq A) \]

or conversely

\[ P(Z \geq -A) = P(Z \leq A) \]


\item \textbf{Interval Rule}\\

\noindent Suppose we have an interval for the random variable $X$ defined by the 
\begin{itemize}
\item the lower bound L
\item the upper bound U
\end{itemize}

\[ L \leq X \leq U \]

The probability of being inside this interval is the \textbf{complement} of being outside the interval.
The event of being outside the interval is the union of two disjoint events.

\begin{itemize}
\item The probability of $X$ being too low for the interval (i.e. less than the interval minimum L)
\[ P(X \leq L) \]
\item The probability of $X$ being too high for the interval (i.e. less than the interval maximum U)
\[ P(X \geq U ) \]
\end{itemize}


\[ P(U \leq X \leq L) = 1 - ( P(X \leq L) +  P(X \geq U ) ) \]
\end{itemize}

\newpage
\subsection*{Parameter Values}

Given the parameters of the normal distribution $X$ in the question.
\begin{itemize}
\item Normal Mean $\mu = 73$ points
\item Normal Standard Deviation $\sigma = 8$ points
\end{itemize}

\begin{itemize}
\item $P(X \leq 91)$
\item $P(65 \leq X \leq 89)$
\end{itemize}
Find the Z score for X = 91.

\[ Z = \frac{x- \mu}{ \sigma} = \frac{91 - 73}{8} =\frac{18}{8} = 2.25\]

Therefore we can say :\\ $P(X \leq 91)$ = $P(Z \leq 2.25)$ \\


From the tables $P(Z \leq 2.25) = 0.9877$
Therefore the probability of getting a grade lower than 91 is 0.9877 (i.e 98.77\%)


What is the probability of getting a score between 65 and 89.
Writing this mathematically:
\[ P(65 \leq X \leq 89) \]

\newpage
%================================================================%
\begin{itemize}
\item How many people get a score greater than 89? ($P(X\geq 89)$)
\item How many people get a score less than 65? ($P(X\leq 65)$)
\end{itemize}

To compute $P(X \geq 89)$ first compute the Z-score.

\[ Z = \frac{x - \mu}{\sigma} = \frac{89 - 73}{8} =\frac{16}{8} = 2 \]

$P(X \geq 89)$ = $P(Z \geq 2)$ = 0.0225.

To compute $P(X \leq 65)$ first compute the Z-score.

\[ Z = \frac{x - \mu}{\sigma} = \frac{65 - 73}{8} =\frac{-8}{8} = -1 \]

$P(X \leq 65)$ = $P(Z \leq -1)$ 

\begin{itemize}
\item We use the \textbf{symmetry rule}
\[ P(Z \leq -1) = P(Z \geq +1) \]
\item so we can say $P(X \leq 65)$ = $P(Z \geq +1)$ 
\item From the statistical tables $P(Z \geq +1)$ = 0.1583.
\end{itemize}

\subsection*{Theory Components}
\begin{itemize}
\item Distinguish between a bimodal distribution and a unimodal distribution
\item Compare and contrast interval and ordinal data.
\end{itemize}
%-------------------------------------------------%
\newpage
\section*{May 2012 Question 5 Normal Distribution }
Given
\begin{itemize}
\item $X$ is the variable of interest.
\item Normal Mean $\mu =25.5$ mpg
\item Normal Standard Deviation $\sigma =4.5$ mpg
\end{itemize}
\begin{itemize}
\item Find $x$ such that $P(X \geq x) = 0.30$
\end{itemize}
\subsection*{Solution}

From the Standard Normal Tables, find the value of $z$ that would give us
\[ P(Z \geq z) = 0.30 \]
Or if you are using the other type of tables 
\[ P(Z \leq z) = 0.70  \]
%-------------------------------------------------%
\newpage

\section*{May 2013 Question 3 Normal Distributions}

\subsection*{Important Information from the Question}
\begin{itemize}
\item Normal Mean $\mu$ = 1000 units
\item Normal Standard Deviation $\sigma$ = 200 units 
\end{itemize}

\subsection*{Objectives}
Compute the following : 
\begin{itemize}
\item $P(X \geq 1400 )$ More than 1400
\item $P(X \leq 500)$ Less than 500
\end{itemize}


\subsection*{Part 1 -  More than 1400}

Firstly compute the z score for 1400.

\[ Z_{1400} =  \frac{X - \mu}{\sigma} = \frac{1400 - 1000}{200} = \frac{400}{200} = 2  \]

So the \textbf{Z-score} in this case is 2.

This much we can say
\[P(X \geq 1400) = P(Z \geq 2)\]

$P(Z \geq 2)$ can be determined using statistical tables.
Depending on which statistical tables you are using, you will get one of the following answers. (Note the 
second and third statements are examples of complementary probabilities.)
\begin{itemize}
\item $P (0 \leq Z \leq 2)$ = 0.4775
\item $P ( Z \leq 2)$ = 0.9775
\item $P ( Z \geq 2)$ = 0.0225
\end{itemize}
The last expression is useful here. Recall that $P(X \geq 1400) = P(Z \geq 2)$. Therefore

\[P(X \geq 1400) = 0.0225\]

\subsection*{Return on Investment Question}

\begin{itemize}
\item The company needs to recover its investment in one year (i.e. make 50000).
\item As each product sells for 2 dollars profit, the company needs to sell 25,000 units to recover its investment.
\item we need to compute the probability of selling more than 25,000 units.
\[P(X \geq 25000) \]
\item We are told the normal mean for demand $\mu =20000$ and the normal standard deviation $\sigma = 4000$.
\item The first step is to compute the \textbf{\textit{z-score}}
\[ z = \frac{x - \mu}{\sigma}  = \frac{25000 - 20000}{4000} = \frac{5000}{4000} = 1.25\]
\end{itemize}

%-------------------------------------------------%
\newpage
\section*{May 2013 Question 4 Probability}
\begin{center}
\begin{tabular}{|c|c|c|c|}
\hline  & within $W$ & outside $O$ & Totals \\ 
\hline Correct time $C$ & 83 & 51 & 134 \\ 
\hline Delayed $D$ & 24 & 12 & 36 \\ 
\hline Totals & 107 & 63 & 170 \\ 
\hline 
\end{tabular} 
\end{center}\begin{itemize}
\item Probability of departing at correct time
\[P(C) = 107/170\]



\item Probability of being delayed and flying outside europe 
\[P(D  \mbox{ and } O) = 12/170\]
\item Probability of 
\[P() \]
\item Probability of 
\[P() \]
\item Probability of 
\[P() \]
\end{itemize}
%-------------------------------------------------%
\section*{May 2013 Question 5 Regression and Correlation}
\begin{center}
\begin{tabular}{|c|c|c|}
\hline  & Experience & No. of Rejects \\ 
\hline A & 4 & 22 \\ 
\hline B & 5 & 20 \\ 
\hline C & 7 & 18 \\ 
\hline D & 9 & 15 \\ 
\hline E & 9 & 16 \\ 
\hline F & 10 & 11 \\ 
\hline G & 14 & 10 \\ 
\hline 
\end{tabular} 
\end{center}

% X = c(4, 5, 7, 9, 9, 10, 14)
% Y = c(22, 20, 18, 15, 16, 11, 10)
% plot(X,Y,pch=16,col="red",cex=2,xlim=c(0,16),ylim=c(0,25),font.lab=2) 
% abline(coef(lm(Y~X)),col="blue",lty=2)
\begin{itemize}
\item $\sum x$ = 58
\item $\sum y$ = 118
\item $\sum x^2$ = 548
\item $\sum y^2$ = 1910
\item $\sum xy$ = 843
\end{itemize}

\begin{itemize}
\item $s_{xx}$ = 67.428
\item $s_{yy}$ = 118
\item $s_{xy}$ = -85
\end{itemize}
\subsection*{The correlation coefficient}
The correlation coefficient is $r = -0.9529$.

\[ r = \frac{s_{xy}}{\sqrt{(s_xx \times s_yy)}}= \frac{-85}{\sqrt{(67.428 \times 118)}}\]

\[r= \frac{-85}{sqrt{(7956.504)}} =  {-85 \over 89.199}  \]

\[r = -0.9529\]

\subsection*{The coefficient of determination}
The coefficient of determination $r^2$ is computed as the square of the correlation coefficient.
\[r^2 = (-0.9529)^2 = 0.90801\]
%-------------------------------------------------%
\section*{May 2013 Question 6a Sampling }

\begin{itemize}
	\item (purely) random samplings
	\item quota sampling
	\item stratified sampling
\end{itemize}

\subsection*{Stratified Sampling}
A stratified sample is a mini-reproduction of the population. Before sampling, the population is divided into characteristics of importance for the research. For example, by gender, social class, education level, religion, etc. Then the population is randomly sampled within each category or stratum. If 38\% of the population is college-educated, then 38\% of the sample is randomly selected from the college-educated population.
%-------------------------------------------------%
\newpage
\section*{May 2012 Question 3 Time Series Analysis}
\begin{center}
	\begin{tabular}{|c|c|c|c|c|}
		\hline  & $Q_1$ & $Q_2$ & $Q_3$ & $Q_4$ \\
		\hline 2007 & 20 & 35 & 26 & 18 \\  
		\hline 2008 & 18 & 36 & 24& 15 \\ 
		\hline 2009 & 14 & 34 & 25 & 14 \\ 
		\hline 2010 & 15 & 32 & 23 & 12 \\ 
		\hline 
	\end{tabular} 
\end{center}
\subsection*{Calculate the trend using the moving averages method}

\begin{tabular}{|c|c|c|c|c|c|} 
	\hline
	Year	&	Time Period	&	Y	&	Moving Total	&	Moving Average	&	Centered MA	\\	\hline
	(column 1)	&	(column 2)	&	(column 3)	&	(column 4)	&	(column 5)	&	(column 6)	\\	\hline
	2007	&	1	&	20	&		&		&		\\	\hline
	2007	&	2	&	35	&	99	&	24.75	&		\\	\hline
	2007	&	3	&	26	&	97	&	24.25	&	24.5	\\	\hline
	2007	&	4	&	18	&	98	&	24.5	&	24.375	\\	\hline
	2008	&	5	&	18	&	96	&	24	&	24.25	\\	\hline
	2008	&	6	&	36	&	93	&	23.25	&	23.625	\\	\hline
	2008	&	7	&	24	&	89	&	22.25	&	22.75	\\	\hline
	2008	&	8	&	15	&	87	&	21.75	&	22	\\	\hline
	2009	&	9	&	14	&	88	&	22	&	21.875	\\	\hline
	2009	&	10	&	34	&	87	&	21.75	&	21.875	\\	\hline
	2009	&	11	&	25	&	88	&	22	&	21.875	\\	\hline
	2009	&	12	&	14	&	86	&	21.5	&	21.75	\\	\hline
	2010	&	13	&	15	&	84	&	21	&	21.25	\\	\hline
	2010	&	14	&	32	&	82	&	20.5	&	20.75	\\	\hline
	2010	&	15	&	23	&		&		&		\\	\hline
	2010	&	16	&	12	&		&		&		\\	\hline
\end{tabular} 

% X = seq (1:13)
% Y1 = c(20,35,26, 18,18,36,24,15,14,34,25,14,15,32,23,12)
% Y2 = c(24.5,24.375,24.25,23.625,22.75,22,21.875,21.875,21.75,21.25,20.75)

\newpage
\subsection*{Calculate the seasonal variation for each Quarter}
Where trend values are present, subtract the trend values for the observed values $y-t$. Format them in a table like what is given belwo.

\begin{center}
	\begin{tabular}{|c|c|c|c|c|c|c|c|c|}
		\hline  & period & \phantom{spa} y-t \phantom{spa} & period & \phantom{spa} y-t \phantom{spa} & period & \phantom{spa} y-t \phantom{spa} & period & \phantom{spa} y-t \phantom{spa} \\ 
		\hline 2007  & 1 & $\ldots$ & 2 & $\ldots$ & 3 &  & 4 &  \\ 
		\hline 2008 & 5 &  & 6 &  & 7 &  & 8 &  \\ 
		\hline 2009 & 9 &  & 10 &  & 11 &  & 12 &  \\
		\hline 2010 & 9 &  & 10 &  & 11 & $\ldots$ & 12 & $\ldots$  \\ 
		\hline 
	\end{tabular} 
\end{center}

\subsection*{Outline three reasons why seasonal variations should be measures}
\begin{enumerate}
	\item
	\item
	\item
\end{enumerate}
%-------------------------------------------------%
\newpage
\section*{May 2013 Question 6 Time Series Analysis}
\begin{center}
\begin{tabular}{|c|c|c|c|c|}
\hline  & $Q_1$ & $Q_2$ & $Q_3$ & $Q_4$ \\ 
\hline 2010 & 14 & 16 & 9 & 14 \\ 
\hline 2011 & 16 & 17 & 12 & 17 \\ 
\hline 2012 & 18 & 20 & 13 & 18 \\ 
\hline 
\end{tabular} 
\end{center}
\subsection*{Calculate the trend using the moving averages method}
\begin{center}
\begin{tabular}{|c|c|c|c|c|c|}
\hline
Year	&	Time Period	&	Y	&	Moving Total	&	Moving Average	&	Centered MA	\\	\hline
(column 1)	&	(column 2)	&	(column 3)	&	(column 4)	&	(column 5)	&	(column 6)	\\	\hline
2008	&	1	&	14	&		&		&		\\	\hline
2008	&	2	&	16	&	53	&	13.25	&		\\	\hline
2008	&	3	&	9	&	55	&	13.75	&	13.5	\\	\hline
2008	&	4	&	14	&	56	&	14	&	13.875	\\	\hline
2009	&	5	&	16	&	59	&	14.75	&	14.375	\\	\hline
2009	&	6	&	17	&	62	&	15.5	&	15.125	\\	\hline
2009	&	7	&	12	&	64	&	16	&	15.75	\\	\hline
2009	&	8	&	17	&	67	&	16.75	&	16.375	\\	\hline
2010	&	9	&	18	&	68	&	17	&	16.875	\\	\hline
2010	&	10	&	20	&	69	&	17.25	&	17.125	\\	\hline
2010	&	11	&	13	&		&		&		\\	\hline
2010	&	12	&	18	&		&		&		\\	\hline
\end{tabular} 
\end{center}
\begin{itemize}
\item Observed values are re-arranged in column 3. An identifier for each observation is listed in column 2.
\item Compute the sum of each set of four observations.
\begin{itemize}
\item The first calculation is the sum of observations 1 to 4.(column 4) We then divide this by four to find the moving average (column 5)
\[\frac{14 + 16 + 9 + 14}{ 4} = \frac{53}{4} = 13.25 \]
\item The second calculation is the sum of observations 2 to 5.(column 4) We then divide this by four to find the moving average (column 5)
\[\frac{ 16 + 9 + 14 + 16}{ 4} = \frac{55}{4} = 13.75 \]
\item The second calculation is the sum of observations 3 to 6.(column 4) We then divide this by four to find the moving average (column 5)
\[\frac{9 + 14 + 16 + 17}{ 4} = \frac{56}{4} = 14 \]
\item We continue on in this fashion until we get to the last step.
\item The last calculation is the sum of observations 9 to 12.(column 4) We then divide this by four to find the moving average (column 5)
\[\frac{ 18 + 20 + 13 + 18}{ 4} = \frac{69}{4} = 17.25 \]
\end{itemize}
\item Columns 4 and 5 are now completed. 
\item The \textbf{\textit{trend}} is the average of each consecutive pairs of values in column 5. 
\begin{itemize}
\item The first value in the trend is the average of 13.25 and 13.75. i.e. 13.5
\item The second value in the trend is the average of 13.75 and 14. i.e. 13.875
\end{itemize}
\item Once we have complete this column we have the trend.
\end{itemize}
\subsection*{Calculate the seasonal variation for each Quarter}
\subsection*{Forecast the number of batteries made in each quarter of 2013}
%-------------------------------------------------%
\section*{May 2012 Question 7 Price Indices}

\subsection*{Some Theory}
The two most basic formulae used to calculate price indices are the \textbf{\textit{Paasche index}} (after the economist Hermann Paasche) 
and the \textbf{\textit{Laspeyres index}} (after the economist Etienne Laspeyres).

The Paasche index is computed as
\[  P_P=\frac{\sum (p_{n}\cdot q_{n})}{\sum (p_{0}\cdot q_{n})}  \]
while the Laspeyres index is computed as
\[  P_L=\frac{\sum (p_{n}\cdot q_{0})}{\sum (p_{0}\cdot q_{0})}  \]


\begin{itemize}
	\item $p_0$ and $q_0$ price and quantity at base period.
	\item $p_n$ and $q_0n$ price and quantity at period $n$.
\end{itemize}

A Laspeyres Index is known as a “base-weighted” or “fixed-weighted” index because the price increases are
weighted by the quantities in the base period.

\subsection*{Question}
\begin{center}
	\begin{tabular}{|c|c|c|c|c|}
		\hline \rule[-2ex]{0pt}{5.5ex}  & $P_0$ (2009) & $Q_0$ (2009) & $P_n$ (2010) & $Q_n$ (2010) \\ 
		\hline \rule[-2ex]{0pt}{5.5ex} A & 36 & 100 & 40 & 95 \\ 
		\hline \rule[-2ex]{0pt}{5.5ex} B & 80 & 12 & 90 & 10 \\ 
		\hline \rule[-2ex]{0pt}{5.5ex} C & 45 & 16 & 41 & 18 \\ 
		\hline \rule[-2ex]{0pt}{5.5ex} D & 5 & 1100 & 6 & 1200 \\ \hline 
	\end{tabular} 
\end{center}

\subsection*{Price Index Formulae}
\begin{itemize}
	\item \textbf{Laspeyres Price Index} 
	\[L_{PI} =  \frac{\sum  p_n \times q_0}{\sum  p_0 \times q_0}\]
	\item \textbf{Paasche Price Index}
	\[P_{PI} =  \frac{\sum  p_n \times q_n}{\sum  p_0 \times q_n}\]
\end{itemize}

\subsection*{Laspeyres Price Index - Calculation}
\begin{center}
	\begin{tabular}{|c|c|c|c|c|c|c|}
		\hline
		&	$p_0$	&	$q_0$	&	$p_n$	&	$q_n$	&	$p_0 \times q_0$	&	$p_n \times q_0$	\\	\hline
		A	&	36	&	100	&	40	&	94	&	3600	&	4000	\\	\hline
		B	&	80	&	12	&	90	&	10	&	960	&	1080	\\	\hline
		C	&	45	&	16	&	41	&	18	&	720	&	656	\\	\hline
		D	&	5	&	1100	&	6	&	1200	&	5500	&	6600	\\	\hline \hline
		&		&		&		& sum		&	10780	&	12336	\\	\hline
	\end{tabular} 
\end{center}

\[L_{PI} =  \frac{\sum  p_n \times q_0}{\sum  p_0 \times q_0} \times 100 = \frac{12336}{10780}\times 100 = 114.43\]

\subsection*{Paasche Price Index - Calculation}
\begin{center}
	\begin{tabular}{|c|c|c|c|c|c|c|}
		\hline
		&	$p_0$	&	$q_0$	&	$p_n$	&	$q_n$	&	$p_n \times q_n$	&	$p_0 \times q_n$	\\	\hline
		A	&	36	&	100	&	40	&	94	&	3760	&	3384	\\	\hline
		B	&	80	&	12	&	90	&	10	&	900	&	800	\\	\hline
		C	&	45	&	16	&	41	&	18	&	738	&	810	\\	\hline
		D	&	5	&	1100	&	6	&	1200	&	7200	&	6000	\\	\hline \hline
		&		&		&		& sum		&	12598	&	10994	\\	\hline
	\end{tabular} 
\end{center}

\[P_{PI} =  \frac{\sum  p_n \times q_n}{\sum  p_0 \times q_n} \times 100 = \frac{12598}{10994}\times 100 = 114.59\]
%--------------------------------------------------------%




%---------------------------------------%
\subsection*{Interpret both indices}

\subsection*{Explain why the Laspeyres Index is considered a "pure" index, but not the Paasche index}
\begin{itemize}
	\item Laspeyres is ``pure" in that it measures like with like from period to period whereas with Paasche's index, the weighting will change.
\end{itemize}
%-------------------------------------------------%
\section*{May 2013 Question 2 Price Indices}

\subsection*{Comparison of Paasche and Laspeyres Indices}

\textbf{which is likely to give a bigger answer, and why?}\\

\begin{itemize}
	\item Laspeyres is ``pure" in that it measures like with like from period to period whereas with Paasche's index, the weighting will change.
\end{itemize}
\subsection*{Calculation of Indices}
First - we are given the following information
\begin{center}
	\begin{tabular}{|c|c|c|c|c|}
		\hline \rule[-2ex]{0pt}{5.5ex}  & $P_0$ (2009) & $Q_0$ (2009) & $P_n$ (2010) & $Q_n$ (2010) \\ 
		\hline \rule[-2ex]{0pt}{5.5ex} A & 50 & 200 & 60 & 300 \\ 
		\hline \rule[-2ex]{0pt}{5.5ex} B & 80 & 100 & 100 & 200 \\ 
		\hline \rule[-2ex]{0pt}{5.5ex} C & 100 & 300 & 120 & 400 \\ 
		\hline 
	\end{tabular} 
\end{center}
%-------------------------------------------------%

\end{document}

Quartiles

For an ordered set of data, which contains $n$ items, the first and third quartiles can be identified as follows

$Q_1$ is the value of the $ {n+1\over 4}$th item


$Q_3$ is the value of the $ {3(n+1)\over 4}$th item


If ${n+1\over 4}$ is not a whole number, use the two items that it is between. The first quartile is the average of those two items.

If ${n+1\over 4}$ is not a whole number, use the two items that it is between. The third quartile is the average of those two items.







$Q_3$ is the value of the $ {3(n+1)\over 4}$th item






\[ IQR = Q_3 - Q_1 \]


Computing methods

There is no universal agreement on choosing the quartile values.


One standard formula for locating the position of the observation at a given percentile, y, 
with n data points sorted in ascending order is:

 
Case 1: If L is a whole number, then the value will be found halfway between positions L and L+1. 
Case 2: If L is a fraction, round to the nearest whole number. (for example, L = 1.2 becomes 1). 

The interquartile range is computed by subtracting the first quartile $Q_1$ from the third quartile $Q_3$.

\[ IQR = Q_3 - Q_1 \]


%---------------------------------------------------------------------------------%

The Geometric Mean

The Geometric mean is a specialized measure used to calculate the average proportional changes.

Geometric mean formula

$G = \sqrt[n]{(1+p_1) \times (1+p_1) +  \ldots + (1+p_n)}$

The price of a commodity changes by the following percentages over a period of four years.
Compute the average price change.


%-------------------------------
How to find the $n-th$ root using your calculator.

What is the n-th root of the number $X$

$\sqrt[n]{X} \;= \; X^{1 \over n} $

for example

$\sqrt[5]{11} \;= \; 11^{1 \over 5} \;= \; 11^{0.2} $



%---------------------------
Year 1 Year 2 Year 3 Year 4
Change 10% 5% -8% 12%


The four terms we are going to multiply are
1.10 , 1.05 0.92 and 1.12.


% 1.190112^{0.25}
% 1.044472





%--------------------
% section 4
In an election campaign, a campaign manager requests that a sample of votes be polled to determine public support for a candidate. In a sample of 150 votes 72 expressed plans to support the candidate.

What is the point estimate of the proportion of the voters who will support the candidate in the election?

contruct and interpret a 95\% confidence interval for the proportion of votes in the population that support the candidate.

given the confidennce interval, is the campaign manager justified in feeling confident that the candidate has at least 50% support

S.E. (\hat{P}) = \sqrt{{\hat{p}(1-\hat{p} \over n}}

%--------------------



%-----------------------

The formulae for geometric distribution is

%P(X=k) = (1-p)^{k-1} \times p^k%

P(X\leq 4 ) = ?

%P(X=k) = (1-0.2)^{4-1} \times 0.2^4%

%-----------------------

% The uniform distribution

What is the probability of an outcome less than 4?

This is equivalent to the probability of an outcome between 2 and 4.

so we let $L$ = 2 and $U$ = 4

${4-2 \over 8 - 2} = {2 \over 6} = 1/3
 

%-----------------------

Degress of freedmom n-1

For values between 31 and 40 we can use degrees freedom = 40

For samples sizes between 41 and 60, we can use degrees of freedom 60

For samples sizes between 61 and 120, we can use degrees of freedom 120

For samples larger than 120, we can use $\infty$




%
%--------------------


Discrete data has distinct whole number values with no intermediate points.

For example, the number of employees in a company is discrete data.


%----------------------------------------------------
Contingency Tables

The probability of throwing a total of 10 with 2 dices


Probability of $x$ \emph{given} that $y$ has occured.


There are 100 students in a firs year college intake. 36 are amles and are studying accounting
9 are male and ard studying economics
45 are female and studying accounting
13 are female and studying economics.

First, lets label this events.

$M$
$F$
$A$
$E$

Lets construct a table to handle this data.



%--------------------------------------------------------------

Inference Procedure

b)	The mean and variance of height in a sample of 25 Irish students are 174cm and 100cm2, respectively.

i)	Test the hypothesis that the mean height of all Irish students is 170cm at a significance level of 5%. 

%-------------------------------------------------------------

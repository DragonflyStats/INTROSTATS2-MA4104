\documentclass[a4]{beamer}
\usepackage{amssymb}
\usepackage{graphicx}
\usepackage{subfigure}
\usepackage{newlfont}
\usepackage{amsmath,amsthm,amsfonts}
%\usepackage{beamerthemesplit}
\usepackage{pgf,pgfarrows,pgfnodes,pgfautomata,pgfheaps,pgfshade}
\usepackage{mathptmx}  % Font Family
\usepackage{helvet}   % Font Family
\usepackage{color}

\mode<presentation> {
 \usetheme{Default} % was Frankfurt
 \useinnertheme{rounded}
 \useoutertheme{infolines}
 \usefonttheme{serif}
 %\usecolortheme{wolverine}
% \usecolortheme{rose}
\usefonttheme{structurebold}
}

\setbeamercovered{dynamic}

\title[MA4704]{MA4413 Statistics for Computing \\ {\normalsize MA4413 Lecture 6A : Continuous Distributions}}
\author[Kevin O'Brien]{Kevin O'Brien \\ {\scriptsize kevin.obrien@ul.ie}}
\date{Autumn 2011}
\institute[Maths \& Stats]{Dept. of Mathematics \& Statistics, \\ University \textit{of} Limerick}

\renewcommand{\arraystretch}{1.5}

%1. MA4104
%2. MA4704
%3. StatsLab
%4. Check that variance and standard deviations are specified correctly.
%--------------------------------------------------%
\begin{document}


%--------------------------------------------------%

\begin{frame}
\frametitle{Confidence Intervals for Sample Proportion}
\Large
\begin{itemize}
\item The general structure of confidence intervals is as follows
\[ \mbox{ Point Estimate } \pm \left[ \mbox{ Quantile } \times \mbox{ Standard Error } \right] \]
\item The structure of a confidence interval for sample proportion is
\[ \hat{p} \pm \left[ z_{(\alpha/2)} \times \mbox{S.E.}(\hat{p}) \right]\]

%\item The standard error, in the case of sample proportions, is
%\[ \mbox{S.E.}(\hat{p}) = \sqrt{\frac{\hat{p}\times (1-\hat{p})}{n}}\]

\end{itemize}
\end{frame}

%--------------------------------------------------%
\begin{frame}
\frametitle{Confidence Intervals for Sample Proportion}
\Large
\textbf{Point Estimate}
\begin{itemize}
\item The point estimate is the sample proportion, denoted $\hat{p}$.  
\item The sample proportion is calculated as the number of `successes' ($x$) divided by the total number of cases, in other words, the sample size $n$.
\[  \hat{p} = \frac{x}{n}  \]
\end{itemize}
\end{frame}

%--------------------------------------------------%
\begin{frame}
\frametitle{Confidence Intervals for Sample Proportion}
\textbf{Quantile}
\begin{itemize}
\item In the cases of large samples ($ n > 30$) , the standard normal ( `Z' ) distribution is used.

\end{itemize}
\end{frame}


%--------------------------------------------------%
\begin{frame}
\frametitle{Confidence Intervals for Sample Proportion}

\begin{itemize}

\item Complement Values:
\begin{itemize} \item When working in terms of proportions, for the the value $\hat{p} =0.40$, the complement value is $1-\hat{p} =0.60$.
\item When working in terms of percentages, for the value $\hat{p} = 40\%$, the complement value is $100-\hat{p} = 60\%$.
\end{itemize}
\end{itemize}
\end{frame}

%--------------------------------------------------%

\begin{frame}
\frametitle{ Sample Proportion : Example}


\begin{description}
\item[Point Estimate] The sample proportion is computed as follows
\[ \hat{p} = \frac{x}{n} = \frac{84}{120} = 0.70 \]
\item[Quantile] We are asked for a 95\% confidence interval. We have a large sample ($n=120$). The quantile is therefore 1.96.
\[ z_{\alpha/2} =1.96\]
\item[Standard Error] The standard error, with sample size n=120 is computed as follows
\[ \mbox{S.E.}(\hat{p}) = \sqrt{\frac{\hat{p} \times (1-\hat{p})}{n}} =  \sqrt{\frac{0.70 \times 0.30}{120}}\]

\end{description}
\end{frame}
%--------------------------------------------------%
\begin{frame}
\frametitle{Confidence Intervals for Sample Proportion}
\Large
\begin{itemize}

\item It is often easier to work in terms of percentages, rather than proportions.
If you are working in terms of percentages, remember to use the appropriate \textbf{\textit{complement value}} in the standard error formula (i.e. $100 - \hat{p}$)

\item The standard error, in the case of sample proportions, is
\[ \mbox{S.E.}(\hat{p}) = \sqrt{\frac{\hat{p}\times (100-\hat{p})}{n}}\]
\end{itemize}
\end{frame}

\end{document}
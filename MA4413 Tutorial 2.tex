\documentclass[]{article}

%opening
\title{MA4413 Statistics for Computing}
\author{Tutorial Sheet for week 3}

\begin{document}

\maketitle

(Time constraints may require continuation in Week 3.)
%\section{Probability Tutorial Sheet}

\section*{Question 1} %MOVED TO WEEK 2

A doctor treating a patient issues a prescription for antibiotics and provides for two repeat prescriptions. The probability that the infection will be cleared by the first prescription is p1 =0.6. 


The probability that successive treatments are successful, given that previous prescriptions were not successful are $p_2$ = 0.5, $p_3$ = 0.4. Calculate the probability that  

\begin{itemize}   
\item[(a.)] the patient is still infected after the third prescription,
\item[(b.)] the patient is cured by the second prescription,
\item[(c.)] the patient does not require a third prescription,
\item[(d.)] the patient is cured by the second prescription, given that the patient is eventually cured.
\end{itemize}
%--------------------------------------------------------------------- %
\section*{Question 2}
A driver passes through 3 traffic lights. The chance he/she will stop at the first is 1/2 , at the second 1/3 and at the third ¼ independently of what happens at any of the other lights. 

What is the probability that
\begin{itemize}
\item[(a)]. the driver makes the whole journey without being stopped at any of the lights
\item[(b)]. the driver is only stopped at the first and third lights
\item[(c)]. the driver is stopped at just one set of lights.
\item[(d)]. the driver stopped at the second set of lights, given he/she stopped at one set of lights.
\end{itemize}
%--------------------------------------------------------------------- %
\section*{Question 3}
The masses of 30 human males and 30 arabian stallions were observed. 
Their masses (in lbs) are given below

\textbf{Humans}
\begin{verbatim}
106, 120, 130, 138, 145, 151, 156, 161, 166, 171
176, 180, 185, 189, 194, 198, 203, 208, 212, 217
223, 228, 234, 240, 247, 255, 264, 276, 290, 313
\end{verbatim}

\textbf{Stallions}
\begin{verbatim}
808, 824, 835, 843, 851, 857, 862, 868, 872, 877
881, 886, 890, 894, 898, 902, 906, 910, 914, 919
923, 928, 932, 938, 943, 949, 957, 965, 976, 992
\end{verbatim}

\begin{itemize}
\item[a)] Draw histograms for these samples and compare them with respect to shape, centrality and relative dispersion. 
\item[b)] Calculate the medians of these samples (from the raw data).
\end{itemize}
%--------------------------------------------------------------------- %

\end{document}

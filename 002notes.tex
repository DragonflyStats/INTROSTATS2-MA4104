\documentclass[slidemain.tex]{subfiles} 
\begin{document} 


\begin{frame}
	
\begin{itemize}
	\item $>$  means `is greater than’
	\item $\geq$ means `is greater than or equal to’
	\item $<$ means `is less than’
	\item $\leq$ means `is less than or equal to’
	\item $\neq$ means `is not equal to’
	\item $\approx$ or $\simeq$ means `is approximately equal to’
\end{itemize}
\end{frame}


\begin{frame}
\frametitle{A simple data set}Suppose that we have a data set with $n$ observations. For each observation, a measure is recorded. Conventionally the measures are denoted $x$ unless a more suitable notation is available. A subscript can be used to indicate which observation the measure is for.
Hence we would write a data set as follows; $(x_{1}, x_{2},x_{3} , x_{1} \dots x_{n})$ (i.e. the first, second, third ... $n$th observation).

\end{frame}

\begin{frame}
\frametitle{Summation}
The summation sign $\sum$ is commonly used in most areas of statistics.
Given $x_1 = 3, x_2= 1, x_3 = 4, x_4 = 6, x_5= 8 $ find:

\[
(i) \displaystyle\sum_{i=1}^{i=n} x_{i}  \hspace{3cm}
(ii) \displaystyle\sum_{i=3}^{i=4} x_{i}^2
\]
\end{frame}
\begin{frame}
\[(i) \displaystyle\sum_{i=1}^{i=n} x_{i} = x_1 + x_2 +  x_3 +  x_4 + x_5 \]  \[= 3 +1 +4 +6 + 8  = \textbf{22} \]

\[ (ii) \displaystyle\sum_{i=1}^{i=n} x_{i}^2 = x_3^2 + x_4^2  = 9 + 16 = \textbf{25} \]

\noindent When all elements of a data set are used, a simple version of the summation notation can be used.
$\displaystyle\sum_{i=1}^{i=n} x_{i}$  can simply be written as $\sum x$

\end{frame}
%---------------------------------------------------------------------------- %
\begin{frame}
\frametitle{Example}
Given that $p_1= 1/4, p_2= 1/8, p_3= 1/8,p_4= 1/3, p_5 = 1/6$ find:

\begin{itemize}
	\item $\displaystyle\sum_{i=1}^{i=n} p_{i} \times x_{i}$
	\item $\displaystyle\sum_{i=1}^{i=n} p_{1} \times x_{i}^2$
\end{itemize}

\end{frame}
%---------------------------------------------------------------------------- %


\begin{frame}
	\frametitle{Joint probability tables}
	\begin{itemize}
	\item A joint probability table is a table in which all possible events (or outcomes) for one variable are listed as
	row headings, all possible events for a second variable are listed as column headings, and the value entered in
	each cell of the table is the probability of each joint occurrence. 
	
	\item Often, the probabilities in such a table are based
	on observed frequencies of occurrence for the various joint events. 
	\item The table
	of joint-occurrence frequencies which can serve as the basis for constructing a joint probability table is called a
	contingency table.
	\end{itemize}
\end{frame}

\end{document}

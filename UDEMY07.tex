Chapter 7
ESTIMATION
THEORY
IN T1-ns CHAPTER;
V Unbiased Estimates and
Efficient Estimates
l/ Point Estimates and Inten/al Estimates
V Confidence Interval Estimates of
Population Parameters
V Confidence Inten/als for Means
l/ Confidence Inten/a/s for Proportions
V Confidence Inten/a/s for Differences
and Sums
\section{Unbiased Estimates and Efficient Estimates}
As we remarked in Chapter 6, a statistic is called an unbiased estimator
of a population parameter if the mean or expectation of the statistic is
equal to the parameter. The corresponding value of the statistic is then
called an unbiased estimate of the parameter.
%=============================================================================================================== %
%%===76 PROBABILITY AND STATISTICS
If the sampling distribution of two statistics
have the same mean, the statistic with the smaller
variance is called a more eflicient estimator of the (
mean. The corresponding value of the efficient sta-  Q
tistic is then called an effieient estimate. Clearly
one would in practice prefer to have estimates that
are both efficient and unbiased, but this is not
always possible.
%=============================================================================================================== %

\section{Point Estimates and lnterval Estimates}
An estimate of a population parameter given by a Single number is
called a point estimate of the parameter. An estimate of a population
parameter given by two numbers between which the parameter may be
considered to lie is called an interval estimate of the parameter.
Example 7.1. If we say that a distance is 5.28 feet, we are giving a
point estimate. If, on the other hand, we say that the distance is
5.28 i 0.03 feet, i.e., the distance lies between 5.25 and 5.31 feet, we
are giving an interval estimate.
A statement of the error or precision of an estimate is often called
its reliability.
\section{Confidence Interval Estimatesof Population Parameters}
Let us and (SS be the mean and standard deviation (standard error) of the
sampling distribution of a statistic S. Then, if the sampling distribution
of S is approximately normal (which as we have seen is true for many
statistics if the sample size n 2 30), we can expect to find S lying in the
interval us — cs to us + cs, us — 20510 us + Zox or us — 36310 us + 365
about 68.27%, 95.45%, and 99.73% of the time, respectively.

%%-CHAPTER 7: Estimation Theory 77
Equivalently we can expect to find, or we
can be confident of finding us in the intervals S
—6St0S+<rS,S—2oSt0S+26S,orS—36Sto
S + 36$ about 68.27%, 95.45%, and 99.73% of 7 |||
the time, respectively. Because of this, We call
these respective intervals the 68.27%, 95.45%,
and 99.73% confidence intervals for estimating
us (i.e., for estimating the population parame-
ter, in this case of an unbiased S). The end numbers of these intervals
(S i 0'5, S i 20's, S i 36$) are then called the 68.37%, 95.45%, and
99.73% confidence limits.
Similarly, S i 1.9665 and S i 2.5865 are 95% and 99% (or 0.95 and
0.99) confidence limits for /.18. The percentage confidence is often called
the confidence level. The numbers 1.96, 2.58, etc., in the confidence
limits are called critical values, and are denoted by zc. From confidence
levels We can find critical values.
%=============================================================================================================== %

In Table 7.1 we give values of zc corresponding to various confi-
dence levels used in practice. For confidence levels not presented in the
table, the values of zc can be found from the normal curve area table in
Appendix B.
9'  W
‘
Table 7-1
Confidence Level I 99.73% I 99% I98’/0 96% I 95.45%
25 | 3.00 I 2.58 |2.33 2.os| 2.00
Confidence Levcll 95% I 90% ]s0%|as.27%| 50%
zq | 1.96 l 1.6451118] 1.00 10.6745
In cases where a statistic has a sampling distribution that is differ-
ent from the nom1al distribution, appropriate modifications to obtain
confidence intervals have to be made.



78 PROBABILITY AND STATISTICS
Confidence Intervals for Means
We shall see how to create confidence intervals for the mean of a pop-
ulation using two different cases. The first case shall be when we have a
large sample size (n 2 30), and the second case shall be when We
have a smaller sample (n < 30) and the underlying population is normal.
Large Samples (n 2 30)
%=============================================================================================================== %

If the statistic S is the sample mean X, then the 95% and 99% confi-
dence limits ior estimation of the population mean |1 are given by X i
1,966; and X i 2.58037, respectively. More generally, the confidence
limits are given by Y i ZCGX where Zr, which depends on the particular
level of confidence desired, can be read from Table 7.]. Using the vales
of 0'); obtained in Chapter Six, we see that the confidence limits for the
population mean are given by
7r1zC% (1)
in case sampling from an infinite population or if sampling is done With
replacement from a finite population, and by
_ N_
Xizcg, Ti (2)
if sampling is done without replacement from a population of finite size N.
In general, the population standard deviation 6 is unknown, so that
to obtain the above confidence limits, we use the estimator S or S.



%=============================================================================================================== %

CHAPTER 1: Estimation Theory 79
Example 7.2. Find a 95\% confidence interval estimating the mean height of the 1546 male students at XYZ University by taking a sample of size 100. (Assume the mean of the sample, x, is 67.45 and that the
standard deviation of the sample, 5, is 2.93 inches.)

The 95\% confidence limits are 1Ti1.96i 4
\/Z
Using 1? = 6745 inches and .€ = 293 inches as an estimate of 6, the
confidence limits are
2.93
67.45il.96 — ' h.
K /TOOJIHC es
or
67.45 i 0.57 inches
Then the 95% confidence interval for the population mean pt is 66.88 to 68.02 inches, which can be denoted by 66.88 < ,u < 68.02. We can therefore say that the probability that the population mean
height lies between 66.88 and 68.02 inches is about 95% or 0.95. In
symbols, We write P(66.88 < /.1 < 68.02) = 0.95. This is equivalent to
saying that We are 95% cunfidertt that the population mean (or true
mean) lies between 66.88 and 6802 inches.
Small Samples (n < 30) and Population Normal
In this case we use the t distribution (see Chapter Ten) to obtain confi-
dence levels. For example, if —tlm5 and IONS are the values of T for
which 2.5% of the area lies in each tail of the tdistribution, then at 95%
confidence interval for Tis given by



80 PROBABILITY AND STATISTICS
<>?—#M
“$0.975 < X < 70.975 (3)
S
from which we can see that tl can be estimated to lie in the interval
e s e s
X -2 7 < < X +1 7 (4)
0.975 V? F1 0.975 ‘/Z
with 95% confidence. In general the confidence limits for population
means are given by
_ 5
X it 7 5)
C x/Z (
where the t‘ values can be read from Appendix C.
A comparison of (5) with (1) shows that for small samples We
replace zc, by Z‘, For n > 30, zt‘ and tr are practically equal. It should be
noted that an advantage of the small sampling theory (Which can of
course be used for large samples as well, i.e., it is exact) in that S appears
in (5) so that the sample standard deviation can be used instead of the
population standard deviation (Which is usually unknown) as in (1).
Sample size is very important! We con- I struct different confidence inten/als I //1
based on sample size, so make sure *3-_,$ £3’
you know which procedure to use.



CHAPTER 1: Estimation Theory 81
Confidence Intervals for Proportions
Suppose that the statistic S is the proportion of “successes” in a sample
of size n 2 30 drawn from a binomial population in which p is the pro-
portion of successes (i.e., the probability of success). Then the confi-
dence limits for p are given by P i ZCUP, Where P denotes the propor-
tion of success in the sample of size n. Using the values of 0'], obtained
in Chapter Six, we see that the confidence limits for the population pro-
portion are given by
‘el
Pizc\»—q=PizCV@ (6)
Vl
3
in case sampling from an infinite population or if sampling is with
replacement from a finite population, Similarly, the confidence limits
€1I'€
pq N—n
Pi Zr 7 E 
if sampling is without replacement from a population of finite size N.
Note that these results are obtained from (1) and (2) on replacing X by
Pandoby 1/pq.
To compute the above confidence limits, we use the sample esti-
mate P for p.
Example 7.3. A sample poll of 100 voters chosen at random from
all voters in a given district indicate that 55% of them were in favor of
a particular candidate. Find the 99% confidence limits for the proportion
of all voters in favor of this candidate.
The 99% confidence limits for the population p are



82 PROBABILITY AND STATISTICS
Pil.58o',, = P1258 Y
Yl
= 055+258  
' " 100
= 0.55 i 0413
Confidence Intervals for
Differences and Sums
If S] and S2 are two sample statistics with approximately normal sam-
pling distributions, confidence limits for the differences of the popula-
tion parameters corresponding to S1 and S2 are given by
s, —s2 izcosfisz = sl —s2 izcvoé +a§1 (8)
while confidence limits for the sum of the population parameters are
given by
2 2
S|+S2izL.0‘S|+S2 =S1+S2iz,1 051 +052 (9)
provided that the samples are independent,
For example, confidence limits for the difference of two population
means, in the case where the populations are infinite and have known
standard deviations <5], 62, are given by
2 2
— — — — 0 0'
X,-X21401? =X|—X2iz(, —1+—2 (10)
' 1 "| "2
where XI, nl and X2, n2 are the respective means and sizes of the two
samples drawn from the populations.



CHAPTER 1: Estimation Theory 83
Similarly, confidence limits for the difference of two population
proportions, where the populations are infinite, are given by
P l—P P l—P
P]_p2iZ( $4.? (11)
1 2
where PI and P2 are the two sample proportions and rt, and n2 are the
sizes of the two samples drawn from the populations.
Remember
The variance for the difference of
means is the same as the variance
for the sum of means! In other words,
2
X+Y X—Y
02 =0
Example 7.4. In a random sample of 400 adults and 600 teenagers
who watched a certain television program, 100 adults and 300 teenagers
indicated that they liked it, Construct the 99.73% confidence limits for
the difference in proportions of all adults and all teenagers who watched
the program and liked it.
Confidence limits for the difference in proportions of the two
groups are given by (I1), where subscripts l and 2 refer to teenagers and
adults, respectively, and Q1 = l — Pl, Q2 = l — P1. Here P1 = 300/600 =
O50 and P2 = 100/400 : 0.25 are, respectively, the proportions of
teenagers and adults who liked the program. The 99.73% confidence
limits are given by

%%-84 PROBABILITY AND STATISTICS
0.50-0.25 1 3   = or25¢0.09 (12)
600 400
Therefore, We can be 99.73% confident that the true difference in
proportions lies between 0.16 and 0.34.


%==============================================================================%
\end{document}

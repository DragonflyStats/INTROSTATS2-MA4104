Integral calculus, also known as integration, is one of the two branches of calculus, with the other being differentiation. Differentiation describes how the value of a function changes with respect to its variables. Integration is the inverse, in that it gives the exact summation of a function between two values. Integral calculus provides an exact means of calculating the area under the curve of a mathematical function. Integration has a wide range of applications in physics and engineering.

The two pioneers of calculus were the 17th century scientists Isaac Newton and Gottfried Leibniz. The mathematical notation used today is based on the work of Leibniz. Although undoubtedly a great scientist, Newton had a reputation for being very competitive and vindictive, and he was unwilling to share the credit with his German contemporary. Newton used his considerable influence at the Royal Society in London to directly and indirectly accuse Leibniz of plagiarism. The validity of these accusations has never been verified, but the controversy destroyed Leibniz's reputation.

Integration is best described in terms of the area under the curve of a mathematical function. This area can be thought of as the sum of vertical strips of equal width. A few wide strips will give an approximate value for the area; increasing the number of strips decreasing their width will give an ever more accurate value for this area. Integral calculus works by considering when the width of these strips approaches 0, and therefore the number of strips approaches infinity. The summation of an infinite number of infinitesimally small strips gives the exact value for the area.

Calculus is used to describe how a function (f) changes in relation to time (t). If the velocity (v) of a particle is defined by the function v = f(t), then how far it has traveled can be worked out using integration, because this is equal the area under the curve. The distance traveled between two distinct points can be found using a definite integral.

There are many other applications of integral calculus — so many that to make an exhaustive list would be impossible. In physics, it can be used to calculate the work done by a body moving in simple harmonic motion or to derive equations describing the behavior of gases. Civil or mechanical engineers can use integral calculus to analyze the motions of fluids or the stress distributions of the pipes carrying these fluids. Electrical engineers use integral calculus to analyze electromagnetic waveforms.


%=================================================================================%
Chapter 2
DESCRIPTIVE STATISTICS
IN THIS CHAPTER2
V Descriptive Statistics
V Measures of Central Tendency
V Mean
V Median
V Mode
V Measures of Dispersion
V Variance and Standard Deviation
V Percentiles
V Interquartile Range
V Skewness
%=============================================================================================================== %

\subsection*{Descriptive Statistics}
When giving a report on a data set, it is useful to describe the data set
with terms familiar to most people. Therefore, we shall develop widely
accepted terms that can help describe a data set, We shall discuss Ways
to describe the center, spread, and shape of a given data set,
% % End of Page 14
%=============================================================================================================== %

%%--- CHAPTER 2: Descriptive Statistics 15
\subsection*{Measures of Central Tendency}

A measure of central tendency gives 21 single value that acts as a repre-
sentative nr average of the values of all the outcomes of your experi-
ment. The main measure of central tendency we will use is the arith-
metic mean. While the mean is used the most, two other measures of
central tendency are also employed. These are the median and the mode.
* Note!
There are many ways to measure the central tendency of a
data set, with the most common being the arithmetic mean,
the median, and the mode. Each has advantages and dis-
advantages, depending on the data and the intended pur-
pose.
Mean
If we are given a set of rt numbers, say xi, x2,  , m, then the mean, usu-
ally denoted by Z or u , is given by
_ + +I4I
Fm (1,
Fl
Example 2.1. Consider the following set of integers:
S= [1,2,3,4,5, 6,7, 8,9]
The mean, i , of the set S is



16 PROBABILITY AND STATISTICS
X_ l+2+3+4+5+6+7+8+9 _5
9
Medmn
The median is that value x for which P(X < x) fig and P(X > X) S % .
ln other words, the median is the value where half of the values of xi,
xz,  ,x” are larger than the median, and half of the values of XI, x2,  ,
x" are smaller than the median.
Example 2.2. Consider the following set of integers:
S= {1,6,3, 8, 2,4,9)
If we want to find the median, we need to find the value, x, where
half the values are above x and half the values are below x. Begin by
ordering the list:
s=uaaaa&w
%=============================================================================================================== %
Notice that the value 4 has three scores below it and three
scores above it. Therefore, the median, in this example, is 4.
In some instances, it is quite possible that the value of the
median will not be one of your observed values.
Example 2.3. Consider the following set of integers:
\[ S= \{L2, 3,4, 6, 8, 9, 12\} \]
Since the set is already ordered, we can
skip that step, but if you notice, we don’t
havejust one value in the middle of the list.
Instead, we have two values, namely 4 and
6. Therefore, the median can be any number


%=============================================================================================================== %
CHAPTER 2: Descriptive Statistics 17
between 4 and 6. In most cases, the average of the two numbers is
reported. So, the median for this set of integers is
4+6:5
2
In general, if we have n ordered data points, and n is an odd
number, then the median is the data point located exactly in the middle
. . . +1 .
of the set. This can be found in location "T of your set. If n is an
even number, then the median is the average of the two middle terms of
Vl
the ordered set. These can be found in locations g and 5+1.
%=============================================================================================================== %

\subsection{Mode}
The mode of a data set is the value that occurs most often, or in other
words, has the most probability of occurring. Sometimes we can have
two, three, or more values that have relatively large probabilities of
occurrence. In such cases, we say that the distribution is bimodal, tri-
mudal, or multimodal, respectively.

%=======================================================================================%
Example 2.4. Consider the following rolls of a ten-sided die:
\[R: \{2, 8, 1, 9, 5, 2, 7, 2, 7, 9, 4, 7, l, 5, 2\}\]
The number that appears the most is the number 2. It appears four times. Therefore, the mode for the set R is the number 2.

Note that if the number 7 had appeared one more time, it would have been present four times as well. In this case, we would have had a bimodal distribution, with 2 and 7 as the modes.
%=======================================================================================%
\section{Measures of Dispersion}

Consider the following two sets of integers:
S = (5, 5, 5, 5, 5, 5} and R: {O, O, 0, 10, IO. IO)

If we calculated the mean for both S and R, we
would get the number 5 both times. However, these are two vastly different data sets. Therefore we need another
descriptive statistic besides a measure of central tendency, which we shall call a measure ofdispersion. We shall 
measure the dispersion or scatter of the values of our a
ll .. I
data set about the mean of the data set. If the values tend i
to be concentrated near the mean, then this measure shall  W
be small, while if the values of the data set tend to be distributed far from the mean, then the measure will be
large. The two measures of dispersions that are usually
used are called the variance and standard deviation.
Variance and Standard Deviation
A quantity of great importance in probability and statistics is called the
variance. The variance, denoted by 62, for a set of n numbers x’, x2,  ,
m, is given by
[(161-.U)2+(X —i1>’+---+<x -A02]
a’ =  (2)
The variance is a nonnegative number. The positive square root of the variance is called the standard deviation.
Example 2.5. Find the variance and standard deviation for the fol»
lowing set of test scores:
T = (75, 80, 82, 87,96}

%=============================================================================================================== %

CHAPTER 2: Descriptive Statistics 19
Since We are measuring dispersion about the mean, we will need to
find the mean for this data set.
75+80+82+87+96
‘u= =84
Using the mean, We can now find the variance.
[<75 -84)’ +(s0-s4>2 +(s2- 34f + (87-$4)’ +(96 - 84$]
5
02:
Which leads to the following;
0,2 I t<81>+<16>+<4>+<9>+<144>t :50 8
5 .
Therefore, the variance for this set of test scores is 50.8. To get the
standard deviation, denoted by 0', simply take the square root of the
variance.
0' = \/<7 = \/50.8 = 7.1274118
%=============================================================================================================== %

The variance and standard deviation are generally the most used quantities to report the measure of dispersion. However, there are other quantities that can also be reported.

You Need to Know I

It is also widely accepted to divide the variance by (n - 1) as opposed to n. While this leads to a different result, as
n gets large, the difference becomes minimal.

%=============================================================================================================== %

%%--- 20 PROBABILITY AND sr/msrtcs
\subsection*{Percentiles}
It is often convenient to subdivide your ordered data set by use of ordi-
nates so that the amount of data points less than the ordinate is some
percentage of the total amount of observations. 

%=======================================================================================%
The values corresponding to such areas are called percentile values, or briefly, percentiles,
Thus, for example, the percentage of scores that fall below the ordinate
at xu is ot, For instance, the amount of scores less than xo 10 would be
0.10 or 10\%, and xom would be called the 10th percentile. 

Another example is the median, Since half the data points fall below the median, it is the 50th percentile (or fifih decile), and can be denoted by xfi 50 ,

The 25th percentile is often thought of as the median of the scores below the median, and the 75th percentile is often thought of as the median of the scores above the median. The 25th percentile is called the first quartile, while the 75th percentile is called the third quartile. As you can imagine, the median is also known as the second quartile,
\subsection*{Interquartile Range}
Another measure of dispersion is the interquarlile range. The interquar-
tile range is defined to be the first quartile subtracted from the third
quartile. In other words, m 75 — m 25

\subsubsection*{Example 2.6.}
Find the interquartile range from the following set of
golf scores:
\[S = \{67, 69, 70, 71, 74-, 77, 78, 82, 89\} \]
\begin{itemize}

\item Since we have nine data points, and the set is ordered, the median is
located in position % , or the 5th position. That means that the medi-
an for this set is 74.

\item The first quartile, x015, is the median of the scores below the fifth



% %- CHAPTER 2: Descriptive Statistics 21
position. 
\item Since we have four scores, the median is the average of the
second and third score, which leads us to x025 = 69.5.
\item The third quartile, x015, is the median of the scores above the fifth
position. Since we have four scores, the median is the average of the
seventh and eighth score, which leads us to xms = 80.
\item Finally, the interquartile range is xO_75 — X015 = 80 — 69.5 = 11.5.
\item One final measure of dispersion that is worth mentioning is the
semiinterquarrile range. As the name suggests, this is simply half of the
interquartile range.
\end{itemize}

Example 2.7. Find the semiinterquartile range for the previous data
set.
%(x0_75 — 1:015) = %(80 — 69.5) = 5t75
%=============================================================================================================== %
\newpage
\subsection*{Skewness}
The final descriptive statistics we will address in this section deals with
the distribution of scores in your data set. For instance, you might have
a symmetrical data set, or a data set that is evenly distributed, or a data
set with more high values than low values.
Often a distribution is not symmetric about any value, but
instead has a few more higher values, or a few more lower values. If the
data set has a few more higher values, then it is said to be skewed to the
right.
Skewed to
tn»: "gm

Figure 2-1
Skewed to the right.

%=============================================================================================================== %


22 PROBABILITY AND STATISTICS
If the data set has 2i few more lower values, then it is said to be skewed
to the left.
Skcwcd to
lbc Icfl
Figure 2-2
Skewed to the left.
* Important!
If a data set is skewed to the right or to the left, then there
is a greater chance that an outlier may be in your data set.
Outliers can greatly affect the mean and standard deviation
of a data set. So, if your data set is skewed, you might want
to think about using different measures of central tendency
and dispersion!



\section{One Way ANOVA}

A One-Way Analysis of Variance is a way to test the equality of three or more means at one time by using variances.

\subsection{Assumptions}
\begin{itemize}\item The populations from which the samples were obtained must be normally or approximately normally distributed.
\item The samples must be independent.
\item The variances of the populations must be equal.
\end{itemize}
\subsection{Hypotheses}
The null hypothesis will be that all population means are equal, the alternative hypothesis is that at least one mean is different.

 Commonly lower case letters apply to the individual samples and capital letters apply to the entire set collectively. That is, n is one of many sample sizes, but N is the total sample size.

\subsection{Decision Rule}
The decision will be to reject the null hypothesis if the test statistic from the table is greater than the F critical value with $k-1$ numerator and $N-k$ denominator degrees of freedom.

If the decision is to reject the null, then at least one of the means is different. However, the ANOVA does not tell you where the difference lies.


\newpage

\documentclass{beamer}

\usepackage{graphics}

\begin{document}


%----------------------------------%
\begin{frame}

{
\Huge
\[\mbox{Coefficient of Variation} \]
}
{
\Large

\[\mbox{www.stats-lab.com} \]
\[ \mbox{twitter: @statslabdublin} \] 

}
\end{frame}

%----------------------------------%

\begin{frame}
\frametitle{Coefficient of Variation}
\LARGE
\vspace{-1cm}
\begin{itemize}
\item The Coefficient of Variation is a useful statistic for comparing the \textbf{dispersion} from multiple sample datasets, particularly when the sample means are quite different from each other.
\item
When comparing variance in two samples, the sample with the higher coefficient of variation is the sample with higher relative variability.
\end{itemize}
\end{frame}
\begin{frame}
\frametitle{Coefficient of Variation}
\LARGE
\vspace{-1cm}
The Coefficient of Variation is the sample standard deviation ($s$) divided by the sample mean ($\bar{x}$), expressed as a percentage.

\[ \mbox{CV}  = \frac{s}{\bar{x}}  \times 100 \% \]
\end{frame}
%----------------------------------%
\begin{frame}
\frametitle{Coefficient of Variation : Example}
\LARGE
\vspace{-2.5cm}
\begin{center}
\begin{tabular}{|c|c|c|}
\hline  & Company 1 & Company 2  \\ 
\hline Mean $\bar{x}$ & 100 & 200 \\ 
\hline Std. Deviation $s$& 20 & 30  \\ 
\hline 
\end{tabular} 
\end{center}
\end{frame}
\begin{frame}
\frametitle{Coefficient of Variation : Example}
\end{frame}
%----------------------------------%


\end{document}
\documentclass{beamer}

\usepackage{default}

\begin{document}
	
	
	Exam : Wednesday March 16th at 9pm
	
	45 minutes long
	Worth 15% of the overall module grade
	No calculators allowed.
	15 multiple choice questions
	
	
	Qualitative Data
	Quantitative Data
	
	Discrete v Continuous
	
\end{frame}
%============================================================ %
\begin{frame}
	An experiment is
	
	An observational study is
	
	This is an experiment because the outcomes are a response to conditions set by the study.
	
\end{frame}
%============================================================ %
\begin{frame} 
	
	Ratio v Interval
	
	The difference between ratio and interval is most evident in how a value of zero is treated.
	
	For Ratio values – a value of zero means the complete absence.
	For interval values
	
	Consider that 0 degrees centigrade is 32 degrees Fahrenheit.
	
	ordinal
	nominal (Categorical)
	
	For example consider a variable”origin” where the outcomes are Limerick /Rest of Munster/Rest of Ireland/Rest of World.
	
	ordinal – contains a hierarchy
	
	Discrete:
	
	Number of Children in a class
	Number of faulty components per batch
	
	Continuous
	
	Length of time between cars arriving at a petrol station pump.
	
	
	It  is not possible have 24.5 children in any class, but it is possible to have an average of 24.5 per class.
\end{frame}
%============================================================ %
\begin{frame}
	
	Definitions
	Population
	Sample
	Statistic: value based on a sample. Commonly used as an estimate for the parameter value.
	
	Parameter value
	Sampling Frame:
	Databases
	
\end{frame}
%============================================================ %
\begin{frame}
	
	Section 2 Graphical methods
	
	Pie Charts – hard to make out what is going on.
	Some aesthetic use i.e. breaking up a long page of text.

\end{frame}
%============================================================ %
\section{Probability}
\begin{frame}
	
\begin{itemize}
\item Law of total probability
	
\item Independent events
	
\item Conditional probability
\end{itemize}	

\end{frame}
%============================================================ %
\begin{frame}
	
	Given that event B has already occurred ,   what is the probability of event A
	
	P( A | B)  “probability of A given B”
	
	P( A and B) “ probability of A and B”
	
	\textbf{Complement event}\\
	
	What is the probability of A not happening
	What is the probability of outcomes not included in A.
	
\end{frame}
%============================================================ %
\begin{frame}
	
	Section 4: Probability Distributions
	
	Expected value of the outcome
	
	E[X]
	
	By definition the expected value is the value that 50% of the the outcomes are greater than.By extension 50% of values are less than the Expected value.
	\[
	P( X \geq E[X])  = 0.50 \]
	\[ P(X \leq E[X]) = 0.50\]
\end{frame}
%============================================================ %
\section{Binomial Distribution}
\begin{frame}	

	Number of independent trials
	
	A coin is tossed eight times. – the number of trials is therefore 8.
	
	A group of people or a batch of items can also be considered as a series of independent trials.
\end{frame}
%============================================================ %
\begin{frame} 
	Probability of a success
	
	A “success” is dependent on how the question is framed, or what is being estimated.

\end{frame}
%============================================================ %
\begin{frame}	
	Continuous Uniform distribution
	
\begin{description}
\item[L] lower bound of an interval
\item[U] upper bound of an interval
\end{description}
	Probability of an outcome being between lower bound L and upper bound U
	\[P( L \leq X \leq U)  =  { U - L \over  b – a }\]
\end{frame}
%============================================================ %
\begin{frame} 
	
	Reminder
	
	" $\leq$" is less than or equal to
	
	" $\geq$" is greater than or equal to
	
	
	\[L \leq X \leq U\]
	
	simply states that X is between L and U inclusively.
	
	("inclusively" mean that X could be exactly L or U also, although the probability of this is extremely low)
	
\end{frame}
%============================================================ %
\begin{frame} 
	The probability density function is given as
	
	\[f(x) = {1 \over b-a} for a \leq x \leq b\]
	
	For any value "c" between the minimum value a and the maximum value b
	
	\[P(X \ geq c) = {b-c \over b-a}\]
	
	here b is the upper bound while c is the lower bound
	
	
	\[P(X \ leq c) = {c-a \over b-a}\]
	
	here c is the upper bound while a is the lower bound
	
\end{frame}
%============================================================ %
\begin{frame}
	
	Exercises on Murdoch Barnes table 6
	
	Find the following
	
	$e^{-0.6}$
	
	$e^{-1.4}$
	
	$e^{-3.2}$
	
\end{frame}
Sampling
Census and sample survey design. Target and study populations, uses and limitations of non-probability sampling methods, sampling frames, sampling fraction.
Revision and extension of basic concepts from Higher Certificate.
Simple random sampling. Estimators of totals, means and proportions; bias. Estimated standard errors, confidence intervals and precision. Sampling fraction and finite population correction. Ratio and regression estimators.
Examples of practical use in various contexts.
Use of supplementary information.
Stratified random sampling. Estimators of totals, means and proportions; bias. Estimated standard errors, confidence intervals and precision. Cost functions. Proportional and optimal allocations. Limitations of stratified sampling.
Examples of practical use in various contexts.
Minimisation of cost × variance.
One-stage cluster sampling. Estimators for totals, means and proportions with equal cluster sizes and with different cluster sizes. Estimated standard errors, confidence intervals and precision. Link with systematic sampling. Description of two-stage sampling and of multi-stage sampling. Limitations.
Examples of practical use in various contexts.

\end{document}

\documentclass[12pt]{article}

%opening
\title{}
\author{}

\begin{document}
% http://www.yale.edu/ynhti/curriculum/units/1986/5/86.05.03.x.html
\maketitle
\tableofcontents

\newpage
\section{Introduction to Statistics}
This chapter contains two separate but related themes, both to do with the
understanding of data. The first idea is to find graphical representations for the data,
which allow one to easily see its most important characteristics. The second idea is to
find simple numbers, like mean or inter-quartile range, which will summarise those
characteristics.

\begin{description}
\item[	Lecture	1	]	Part A	About descriptive statistics
\item[	Lecture	1	]	Part B	About inferential statistics
\item[	Lecture	1	]	Part C	Variables
\item[	Lecture	2	]	Part A	Parameters and Statistics
\item[	Lecture	2	]	Part B	Categorical Data
\item[	Lecture	2	]	Part C	Summation notation
\item[	Lecture	3	]	Part A	Measurement scales
\item[	Lecture	3	]	Part B	Computing the sample mean
\item[	Lecture	3	]	Part C	Computing the sample median





%----------------------------------------------------------------------- %
%http://www.eumetcal.org/resources/ukmeteocal/verification/www/english/msg/ver_categ_forec/uos1/uos1_ko1.htm
\newpage
\section{Graphical Methods}
\end{description}
\begin{description}
\item[	Lecture	4	]	Part A	Pie-Charts (?)
\item[	Lecture	4	]	Part B	Relative Frequency and Cumulative Frequency
\item[	Lecture	4	]	Part C	Histograms
\item[	Lecture	5	]	Part A	Ogives
\item[	Lecture	5	]	Part B	Stem and Leaf Plots
\item[	Lecture	5	]	Part C	Boxplots
\end{description}


 

%---------------------------------------------------------------------- %
\newpage
\begin{description}
\section{Descriptive Statistics}
\item[	Lecture	6	]	Part A	Interquartile Range		
\item[	Lecture	6	]	Part B			
\item[	Lecture	6	]	Part C			
\item[	Lecture	7	]	Part A	Types of Data		
\item[	Lecture	7	]	Part B	Range			
\item[	Lecture	7	]	Part C	Mean		
\item[	Lecture	8	]	Part A	Median and Mode	
\item[	Lecture	8	]	Part B	Outliers
\item[	Lecture	8	]	Part C	Trimean and Trimmed Mean (Outliers)
\item[	Lecture	9	]	Part A  Skew and Kurtosis
\item[	Lecture	9	]	Part B	Summary 		
\item[	Lecture	9	]	Part C	Spread		
\item[	Lecture	10	]	Part A		
\item[	Lecture	10	]	Part B	Semi-Interquartile Range		
\item[	Lecture	10	]	Part C	Variance		
\end{description}




\newpage




%----------------------------------------------------------------- %





\section{Grouped Data}
\begin{description}

\item[	Lecture	10	]	Part A	
\item[	Lecture	10	]	Part B	
\item[	Lecture	10	]	Part C	
\end{description}
\newpage

\section{Probability }
\begin{description}

\item[	Lecture	11	]	Part A	Introduction to Probability
\item[	Lecture	11	]	Part B	Counting and Permutations
\item[	Lecture	11	]	Part C	Permuations without Repetition
\item[	Lecture	12	]	Part A	
\item[	Lecture	12	]	Part B	Axioms of Probability
\item[	Lecture	12	]	Part C	Conditional Probability and Independent Events
\end{description}

\subsection*{Random experiment}
\begin{itemize}
\item \textbf{Sample Space, S}. For a given experiment the sample space, S, is the set of all
possible outcomes.
\item \textbf{Event, E}. This is a subset of S. If an event E occurs, the outcome of the
experiment is contained in E.
\end{itemize}

%------------------------------------------------------------ %
\newpage
\section{Probability Distributions}
\begin{description}

\item[	Lecture	13	]	Part A	
\item[	Lecture	13	]	Part B	
\item[	Lecture	13	]	Part C	 The Geometric Distribution
\item[	Lecture	14	]	Part A	Poisson Approximation of the Binomial Distribution
\item[	Lecture	14	]	Part B	
\item[	Lecture	14	]	Part C	
\end{description}





\section{The Normal Distribution}
\begin{description}
\item[	Lecture	15	]	Part A	
\item[	Lecture	15	]	Part B	
\item[	Lecture	15	]	Part C	Using the Statistical tables
\item[	Lecture	16	]	Part A	
\item[	Lecture	16	]	Part B	
\item[	Lecture	16	]	Part C	
\item[	Lecture	17	]	Part A	
\item[	Lecture	17	]	Part B	
\item[	Lecture	17	]	Part C	Factorial And Choose (Pascal Traingle)
\item[	Lecture	18	]	Part A	Binomial 
\item[	Lecture	18	]	Part B	Geometric
\item[	Lecture	18	]	Part C	Poisson
\item[	Lecture	19	]	Part A	Normal Distribution
\item[	Lecture	19	]	Part B	Standard Normal Distribution
\item[	Lecture	19	]	Part C	
\item[	Lecture	20	]	Part A	
\item[	Lecture	20	]	Part B	
\item[	Lecture	20	]	Part C	
\end{description}


\newpage
%\chapter{Revision}
\section{Relational operators}
\begin{itemize}
\item $>$  means `is greater than’
\item $\geq$ means `is greater than or equal to’
\item $<$ means `is less than’
\item $\leq$ means `is less than or equal to’
\item $\neq$ means `is not equal to’
\item $\approx$ or $\simeq$ means `is approximately equal to’
\end{itemize}




\section{A simple data set}Suppose that we have a data set with $n$ observations. For each observation, a measure is recorded. Conventionally the measures are denoted $x$ unless a more suitable notation is available. A subscript can be used to indicate which observation the measure is for.
Hence we would write a data set as follows; $(x_{1}, x_{2},x_{3} , x_{1} \dots x_{n})$ (i.e. the first, second, third ... $n$th observation).


\section{Summation}
The summation sign $\sum$ is commonly used in most areas of statistics.
Given $x_1 = 3, x_2= 1, x_3 = 4, x_4 = 6, x_5= 8 $ find:

\[
(i) \displaystyle\sum_{i=1}^{i=n} x_{i}  \hspace{3cm}
(ii) \displaystyle\sum_{i=3}^{i=4} x_{i}^2
\]
\begin{eqnarray*}(i) \displaystyle\sum_{i=1}^{i=n} x_{i} &=& x_1 + x_2 +  x_3 +  x_4 + x_5 \\  &=& 3 +1 +4 +6 + 8 \\ &=& \textbf{22} \end{eqnarray*}

\[ (ii) \displaystyle\sum_{i=1}^{i=n} x_{i}^2 = x_3^2 + x_4^2  = 9 + 16 = \textbf{25} \]

\noindent When all elements of a data set are used, a simple version of the summation notation can be used.
$\displaystyle\sum_{i=1}^{i=n} x_{i}$  can simply be written as $\sum x$


\subsection*{Example}
Given that $p_1= 1/4, p_2= 1/8, p_3= 1/8,p_4= 1/3, p_5 = 1/6$ find:

\begin{itemize}
\item $\displaystyle\sum_{i=1}^{i=n} p_{i} \times x_{i}$
\item $\displaystyle\sum_{i=1}^{i=n} p_{1} \times x_{i}^2$
\end{itemize}

\section{Arithmetic mean} One of the basic quantities is the arithmetic mean (it is sometimes
called the `average’ but there are in fact other measures of average apart from the
mean ). The arithmetic mean is calculated by
adding the measures of the number of observations in which you are interested and
dividing by the number of observations.

\[ \bar{x} =  \frac{\sum x}{n}  \]

\noindent For our data set $\bar{x} = \frac{22}{5}  = \textbf{4.4}$.



\newpage
\section{Medians and modes}
The median ($\tilde{x}$) is the value that separates a sample into two groups; 50\% of observations are greater
than the median and 50\% are less than it.
\\ \\\noindent The set of n numbers is arranged in ascending order, say as $x_(1), x_(2), x_(3), \dots x_(n)$, where $x_(1)$ is the smallest of the observations and $x_(n)$ is the largest.
\\ \\\noindent Computation of the median differs for samples that have an odd number size, and samples with an even number size. If sample size $n$ is odd
\[ \tilde{x} =  x_{(\frac{n+1}{2})} ,\]
or if $n$ is even 
\[ \tilde{x} =  \frac{  x_{(\frac{n}{2})}  + x_{(\frac{n+1}{2})} }{2}. \]
\\ \\ \noindent A visual inspection of the ordered data set will be useful for a quick determination of the median.
For example, the median of the numbers 1, 3, 4, 5 and 7 is 4 (check this by
rearranging the numbers in order!) and for 1, 3, 4 and 5, the median is (3+4)/2, that is
3.5.
\\ \\ \noindent The mode is the most frequently occurring value. There is not necessarily only one
such value. For example, the figures 1, 2, 2, 3, 5, 9, 9, 11 have two modes: the
numbers 2 and 9.




\section{Measures of dispersion}
It is unlikely we will only be interested in the average value of our data, we will want
to know how large the spread or dispersion of values is about it. 

\noindent The simplest measure of dispersion is the range.
The range is simply the difference of the lowest and highest values. The range is
another easy-to-understand measure, but it will clearly be very affected by a few
extreme values.\\\\
\noindent Aside from the range, the most common measures of dispersion are:
\begin{itemize}
\item Variance
\item Standard deviation
\item Mean Absolute Deviation (MAD)
\item Inter-quartile range.
\end{itemize}
\noindent The first three are related to the use of the arithmetic mean, and are computed using deviations of each observation from the mean.\\\\\noindent The mean absolute deviation (MAD) uses the absolute values of the deviations from
the mean and perhaps gives us a more intuitively understandable measure of deviation than
variance and standard deviation.

\section{Statistics for grouped data}
grouped data refers to the arrangement of raw data with a wide range of values into groups. This process makes the data more manageable. Graphs and frequency diagrams can then be drawn showing the class intervals chosen instead of individual values.


\noindent An estimate, $\bar{x}$, of the mean of the population from which the data are drawn can be calculated from the grouped data as:
\[ \bar{x} = \frac{\sum f x }{\sum f}\]
In this formula, $x$ refers to the mid-point of the class intervals, and $f$ is the class frequency. Note that the result of this will be different from the sample mean of the ungrouped data.

\newpage

\begin{tabular}{|c|c|c|}
\hline
Class limits& Class midpoint & frequency \\
  \hline
\$240 - 259.99 & \$250 &7\\
\$260 - 279.99 & \$270 &20\\
\$280 - 299.99 & \$290 &33\\
\$300 - 319.99 & \$310 &25\\
\$320 - 339.99 & \$330 &11\\
\$340 - 359.99 & \$350 &4\\
\hline
& & Total = 100\\
  \hline
\end{tabular}

\section{Cumulative frequency}
The graph of a cumulative frequency distribution is called an ogive (pronounced ``o-jive"). For the less-than
type of cumulative distribution, this graph indicates the cumulative frequency below each exact class limit of
the frequency distribution. When such a line graph is smoothed, it is called an ogive curve.


\section{Relative frequency}
A relative frequency distribution is one in which the number of observations associated with each class has
been converted into a relative frequency by dividing by the total number of observations in the entire
distribution. Each relative frequency is thus a proportion, and can be converted into a percentage by multiplying
by 100.\\

\noindent One of the advantages associated with preparing a relative frequency distribution is that the cumulative
distribution and the ogive for such a distribution indicate the cumulative proportion (or percentage) of
observations up to the various possible values of the variable. A percentile value is the cumulative percentage of
observations up to a designated value of a variable.

\section{Coefficient of variation}
The coefficient of variation, CV, indicates the relative magnitude of the standard deviation as compared
with the mean of the distribution of measurements, as a percentage. Thus, the formulas are
\begin{eqnarray*}
\mbox{ Population : } CV = \frac{\sigma}{\mu } \times 100 \\
\mbox{ Sample : } CV = \frac{s}{\bar{x}} \times 100
\end{eqnarray*}

The coefficient of variation is useful when we wish to compare the variability of two data sets relative to the
general level of values (and thus relative to the mean) in each set.

\section{Basic definitions of probability}

The symbol P is used to designate the probability of an event. Thus P(A) denotes the probability that event
A will occur in a single observation or experiment.
\\
\\
The smallest value that a probability statement can have is 0 (indicating the event is impossible) and the
largest value it can have is 1 (indicating the event is certain to occur). Thus, in general:
$0 	\leq P(A) \leq 1$
\\
\\
In a given observation or experiment, an event must either occur or not occur. Therefore, the sum of the
probability of occurrence plus the probability of nonoccurrence always equals 1. Thus, where $A^{\prime}$ indicates the nonoccurrence of event A, we have
$P(A) + P(A^{\prime}) =  1$

\section*{Mutually exclusive events}
Two or more events are mutually exclusive, or disjoint, if they cannot occur together. That is, the occurrence
of one event automatically precludes the occurrence of the other event (or events). For instance, suppose we
consider the two possible events ``ace" and ``king" with respect to a card being drawn from a deck of playing
cards. These two events are mutually exclusive, because any given card cannot be both an ace and a king.
Two or more events are nonexclusive when it is possible for them to occur together. 
\\
Note that this definition does not indicate that such events must necessarily always occur jointly. For instance, suppose we consider the two possible events ``ace" and ``spade". These events are not mutually exclusive, because a given card can be both an ace and a spade; however, it does not follow that every ace is a spade or every spade is an ace.
\section*{General rule of addition}
For events that are not mutually exclusive, the probability of the joint occurrence of the two events is
subtracted from the sum of the simple probabilities of the two events. We can represent the probability of joint
occurrence by P(A and B). In the language of set theory this is called the intersection of A and B and the
probability is designated by P(A and B).  Thus, the rule of addition for events that are not mutually exclusive is
\[ P(A \mbox{ or }B) = P(A)+ P(B) - P(A \mbox{ and }B)\]

\subsection*{Example}

When drawing a card from a deck of playing cards, the events ``ace" and ``spade" are not mutually
exclusive. The probability of drawing an ace (A) or spade (S) (or both) in a single draw is
\begin{eqnarray*} P(A \mbox{ or }B) &=& P(A) + P(S) - P(A \mbox{ and }B)\\ &=& 4/52 + 13/52 -1/52 \\&=& 16/52 \\
&=& \textbf{4/13} 
\end{eqnarray*}

\section*{Independent events}
Two events are independent when the occurrence or nonoccurrence of one event has no effect on the
probability of occurrence of the other event. Two events are dependent when the occurrence or nonoccurrence
of one event does affect the probability of occurrence of the other event.



\section*{Conceptual approaches}
Historically, three different conceptual approaches have been developed for defining probability and for
determining probability values: the classical, relative frequency, and subjective approaches.\\\noindent If N(A) possible elementary outcomes are favorable to event A,
N(S) possible outcomes are included in the sample space, and all the elementary outcomes are equally likely and
mutually exclusive, then the probability that event A will occur is
\[P(A) = \frac{N(A)}{N(S)}\]

\subsection*{Examples}
When a fair dice is thrown, what are the possible outcomes? There are 6 possible outcomes. The dice can role any number between one and six. Each outcome is equally likely. The probability of each outcome is 1/6.


In a well-shuffled deck of cards which contains 4 aces and 48 other cards, the probability of an ace (A)
being obtained on a single draw is;
\[ P(A)= N(A)/ N(S) = 4/52 = 1/13 \]

\section{Bayes’ theorem}
In its simplest algebraic form, Bayes’ theorem is concerned with determining the conditional probability of
event A given that event B has occurred. The general form of Bayes’ theorem is
\[ P(A|B) =
\frac{P(A \mbox{ and }B)}{P(B)} \]

\section{Joint probability tables}
A joint probability table is a table in which all possible events (or outcomes) for one variable are listed as
row headings, all possible events for a second variable are listed as column headings, and the value entered in
each cell of the table is the probability of each joint occurrence. 
\\
\\
\noindent Often, the probabilities in such a table are based
on observed frequencies of occurrence for the various joint events. The table
of joint-occurrence frequencies which can serve as the basis for constructing a joint probability table is called a
contingency table.

\section{random variables}
A random variable is defined as a numerical event whose value is determined by a chance process.
When probability values are assigned to all possible numerical values of a random variable X, either by a listing
or by a mathematical function, the result is a probability distribution. \\
\\
The sum of the probabilities for all the possible numerical outcomes must equal 1.0. Individual probability values may be denoted by the symbol f (x),
which indicates that a mathematical function is involved, by P(x = X), which recognizes that the random
variable can have various specific values, or simply by P(X).


\section{solutions 1}

\begin{enumerate}

\item Assume that the number of weekly study hours for students at a certain university
is approximately normally distributed with a mean of 22 and a standard deviation
of 6.
\begin{enumerate}
\item Find the probability that a randomly chosen student studies less than 12
hours.
\item Estimate the percentage of students that study more than 37 hours.
\end{enumerate}


$X \sim \mathcal(22,6^2)$\\
$P(X \leq 12)$\\
$P(X \geq 37)$\\
$Z_1 = \frac{12 - 22}{6} = \frac{-10}{6} = -1.66 $\\
$Z_2 = \frac{37 - 22}{6} = \frac{15}{6} = 2.5 $


\end{enumerate}


\section{Normal - example}

In an examination the scores of students who attend schools of type A are
normally distributed about a mean of 55 with a standard deviation of 6. The
scores of students who attend type B schools are normally distributed about a
mean of 60 with a standard deviation of 5.

Which type of school would have a higher proportion of students with marks above 70?

\begin{itemize}
\item $\mu_A$ = 55
\item $\sigma_A$ = 8
\item $\mu_B$ = 60
\item $\sigma_B$ = 5
\end{itemize}

We have to fins $P(X_A \geq 70)$
and $P(X_B \geq 70)$.


using the standardisation formula
$Z_A = \frac{70 - 55}{6} = \frac{15}{6} = 2.5 $

$Z_B = \frac{70 - 60}{5} = \frac{10}{5} = 2 $

\newpage

\section*{Confidence intervals: example}

\noindent A random sample of 25 female students is chosen from
students at higher education establishments in a particular area
of a country, and it is found that their mean height is 165
centimeters with sample variance of 81.
\\
\\
Assuming that the distribution of the heights of the students may
be regarded as normally distributed, calculate a 95\% confidence
interval for the mean height of female students.
\\
\\
You are asked to obtain a 98\% confidence interval for the mean
height of width 3 centimeters. What sample size would be needed in
order to achieve that degree of accuracy?

\subsection*{Solution 1}

\begin{itemize}
\item Confidence interval formula: $\bar{X} \pm t_{(\alpha/2,df)}
\frac{s}{\sqrt{n}}$.

\item Small sample, therefore degrees of freedom = 24 (i.e. n-1).

\item 95\% confidence, therefore $\alpha = 5\% (i.e. 0.05)$

\item Correct t value from tables: $t_{(\alpha /2,n-1)} =
t_{0.025,24} $

\item Interval computed: $165 \pm 1.96 \frac{9}{\sqrt{25}}  =
(160:09; 169:91).$
\end{itemize}


\subsection*{Solution 2}

\begin{itemize}
\item Confidence interval width is 3, so half-width is 1.5

\item Seek n such that $1.96 \times \frac{9}{\sqrt{n}} = 1.5$

\item Divide both sides by $1.96 \times 9$ \\
\[\frac{1}{\sqrt{n}} = \frac{1.5}{1.96 \times 9} =\]


\item invert and square boths sides.
\end{itemize}
\end{document}

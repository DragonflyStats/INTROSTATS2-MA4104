\begin{document}
%----------------------------%

\begin{frame}
\huge
\[ \mbox{Probability Distributions}\]
\[ \mbox{The Pareto Distribution}  \]
\Large
\[ \mbox{www.Stats-Lab.com}  \]
\[ \mbox{Twitter: @StatsLabDublin}  \]

\end{frame}
%----------------------------%
%----------------------------%

\begin{frame}

\frametitle{The Pareto Distribution}




The General Pareto Distribution

As with many other distributions, the Pareto distribution is often generalized by adding a scale parameter. Thus, suppose that Z has the basic Pareto distribution with shape parameter a>0. If b>0, the random variable X=bZ has the Pareto distribution with shape parameter a and scale parameter b. Note that X takes values in the interval [b, \infty).

Analogies of the results given above follow easily from basic properties of the scale transformation.

\end{frame}
%----------------------------%
%----------------------------%

\begin{frame}
\frametitle{The Pareto Distribution}

The probability density function is
f(x)=abaxa+1,b\leq x< \infty
The distribution function is
F(x)=1-(bx)a,b\leq x< \infty
\end{frame}
%----------------------------%
%----------------------------%

\begin{frame}
\frametitle{The Pareto Distribution}



The quantile function is
\[F-1(p)=b(1-p)1/a,0\leq p<1\]


\end{frame}
%----------------------------%
%----------------------------%

\begin{frame}
\begin{itemize}

\frametitel{Cumulative distribution function[edit]}
From the definition, the cumulative distribution function of a Pareto random variable with parameters \alpha and xm is
\[F_X(x) = \begin{cases}
1-\left(\frac{x_\mathrm{m}}{x}\right)^\alpha & \text{for } x \ge x_\mathrm{m}, \\
0 & \text{for }x < x_\mathrm{m}.
\end{cases}
\]
\end{itemize}
\end{frame}
%----------------------------%
%----------------------------%

\begin{frame}
\begin{itemize}
When plotted on linear axes, the distribution assumes the familiar J-shaped curve which approaches each of the orthogonal axes asymptotically. All segments of the curve are self-similar (subject to appropriate scaling factors). When plotted in a log-log plot, the distribution is represented by a straight line.
Probability density function[edit]
It follows (by differentiation) that the probability density function is
\[
f_X(x)= \begin{cases} \alpha\,\dfrac{x_\mathrm{m}^\alpha}{x^{\alpha+1}} & \text{for }x \ge x_\mathrm{m}, \\[12pt] 0 & \text{for } x < x_\mathrm{m}. \end{cases} 
\]
\end{itemize}
\end{frame}
%----------------------------%
%----------------------------%

\begin{frame}
\begin{itemize}
% Moments and characteristic function[edit]
The expected value of a random variable following a Pareto distribution is
\[
E(X)= \begin{cases} \infty & \text{if }\alpha\le 1, \\ \frac{\alpha x_\mathrm{m}}{\alpha-1} & \text{if }\alpha>1. \end{cases}
\]
\end{itemize}
\end{frame}
%----------------------------%
%----------------------------%

\begin{frame}
\begin{itemize}

\item The variance of a random variable following a Pareto distribution is
\[
\mathrm{Var}(X)= \begin{cases} \infty & \text{if }\alpha\in(1,2], \\ \left(\frac{x_\mathrm{m}}{\alpha-1}\right)^2 \frac{\alpha}{\alpha-2} & \text{if }\alpha>2. \end{cases}
(If \alpha\le 1, the variance does not exist.)
\]


\end{itemize}
\end{frame}
%----------------------------%
%----------------------------%

\begin{frame}
\begin{itemize}


The Pareto distribution is 
a continuous distribution with the probability density function (pdf):
\[
f(x; \alpha, \beta) = \alpha\beta\alpha / x\alpha+ 1
\]
For shape parameter $\alpha > 0$, and location parameter \beta > 0, and \alpha > 0.
\end{itemize}
\end{frame}
%----------------------------%
%----------------------------%

\begin{frame}
\begin{itemize}
The Pareto distribution often 
describes the larger compared
 to the smaller. 
A classic example is that 
80\% of the wealth is owned by 20% of the population.

% The following graph illustrates how the PDF varies with the location parameter \beta:
\end{itemize}
\end{frame}
%----------------------------%
\end{document}

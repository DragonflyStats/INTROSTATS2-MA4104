

%===================================================================================%

The Gamma distribution is very important for technical reasons, since it is the parent of the exponential distribution and can explain many other distributions.

The probability distribution function is:


f_x (x) =
\begin{cases}
\frac{1}{a^p \Gamma (p)} x^{p-1} e^{-x/a}, & \mbox{if } x \ge 0 \\
0, & \mbox{if } x < 0
\end{cases}\quad a,p >0

Where  \Gamma(p) = \int_0^\infty  t^{p-1} e^{-t}\,dt\, is the Gamma function. The cumulative distribution function cannot be found unless p=1, in which case the Gamma distribution becomes the exponential distribution. The Gamma distribution of the stochastic variable X is denoted as  X \in \Gamma (p,a) .

Alternatively, the gamma distribution can be parameterized in terms of a shape parameter \alpha = k and an inverse scale parameter \beta = 1/\theta, called a rate parameter:

 g(x;\alpha,\beta) = K x^{\alpha-1}  e^{-\beta\,x}   \ \mathrm{for}\ x > 0 \,\!.
where the K constant can be calculated setting the integral of the density function as 1:


\int_{-\infty}^{+\infty}g(x;\alpha,\beta) \mathrm{d}t \, = \int_{0}^{+\infty} K x^{\alpha-1}  e^{-\beta\,x}  \mathrm{d}x \, = 1
following:


K \int_{0}^{+\infty} x^{\alpha-1}  e^{-\beta\,x}  \mathrm{d}x \, = 1

K = \frac{1}{\int_{0}^{+\infty} x^{\alpha-1}  e^{-\beta\,x}  \mathrm{d}x}
and, with change of variable  y = \beta x  :


\begin{align}
K &= \frac{1}{\int_{0}^{+\infty} \frac{y^{\alpha-1}}{\beta^{\alpha - 1}}  e^{-y}  \frac{\mathrm{d}y}{\beta}} \\

&= \frac{1}{\frac{1}{\beta^{\alpha}}\int_{0}^{+\infty} y^{\alpha-1}  e^{-y} \mathrm{d}y} \\

&= \frac{\beta^{\alpha}}{\int_{0}^{+\infty} y^{\alpha-1}  e^{-y}  \mathrm{d}y} \\

&= \frac{\beta^{\alpha}}{\Gamma(\alpha)}
\end{align}

following:

 g(x;\alpha,\beta) = x^{\alpha-1}  \frac{\beta^{\alpha} \, e^{-\beta\,x} }{\Gamma(\alpha)}  \ \mathrm{for}\ x > 0 \,\!.

%===================================================================================%
Probability Density Function[edit]
We first check that the total integral of the probability density function is 1.

\int^\infin_{-\infin}\frac{1}{a^p \Gamma (p)} x^{p-1} e^{-x/a}dx
Now we let y=x/a which means that dy=dx/a

\frac{1}{ \Gamma (p)} \int^\infin_{0} y^{p-1} e^{-y}dy
\frac{1}{ \Gamma (p)} \Gamma (p)=1
%===================================================================================%
Mean[edit]
\operatorname{E}[X]=\int^\infin_{-\infin}x \cdot \frac{1}{a^p \Gamma (p)} x^{p-1} e^{-x/a}dx
Now we let y=x/a which means that dy=dx/a.

\operatorname{E}[X]=\int^\infin_{0}ay \cdot \frac{1}{\Gamma (p)} y^{p-1} e^{-y}dy
\operatorname{E}[X]=\frac{a}{\Gamma (p)}\int^\infin_{0}y^{p} e^{-y}dy
\operatorname{E}[X]=\frac{a}{\Gamma (p)}\Gamma (p+1)
We now use the fact that \Gamma (z+1)=z\Gamma (z)

\operatorname{E}[X]=\frac{a}{\Gamma (p)}p\Gamma (p)=ap

%===================================================================================%
Variance[edit]
We first calculate E[X^2]

\operatorname{E}[X^2]=\int^\infin_{-\infin}x^2 \cdot \frac{1}{a^p \Gamma (p)} x^{p-1} e^{-x/a}dx
Now we let y=x/a which means that dy=dx/a.

\operatorname{E}[X^2]=\int^\infin_0 a^2 y^2 \cdot \frac{1}{a \Gamma (p)} y^{p-1} e^{-y}ady
\operatorname{E}[X^2]=\frac{a^2}{ \Gamma (p)}\int^\infin_0  y^{p+1} e^{-y}dy
\operatorname{E}[X^2]=\frac{a^2}{ \Gamma (p)}\Gamma (p+2) =pa^2(p+1)
Now we use calculate the variance

\operatorname{Var}(X)=\operatorname{E}[X^2]-(\operatorname{E}[X])^2
\operatorname{Var}(X)=pa^2(p+1)-(ap)^2=pa^2


Gamma Distribution[edit]
Gamma
Probability density function
Probability density plots of gamma distributions
Cumulative distribution function
Cumulative distribution plots of gamma distributions
Parameters	
\scriptstyle k \;>\; 0 shape
\scriptstyle \theta \;>\; 0\, scale
Support	\scriptstyle x \;\in\; (0,\, \infty)\!
PDF	\scriptstyle \frac{1}{\Gamma(k) \theta^k} x^{k \,-\, 1} e^{-\frac{x}{\theta}}\,\!
CDF	\scriptstyle \frac{1}{\Gamma(k)} \gamma\left(k,\, \frac{x}{\theta}\right)\!
Mean	\scriptstyle \operatorname{E}[ X] = k \theta \!
\scriptstyle \operatorname{E}[\ln X] = \psi(k) +\ln(\theta)\!
(see digamma function)
Median	No simple closed form
Mode	\scriptstyle (k \,-\, 1)\theta \text{ for } k \;>\; 1\,\!
Variance	\scriptstyle\operatorname{Var}[ X] = k \theta^2\,\!
\scriptstyle\operatorname{Var}[\ln X] = \psi_1(k)\!
(see trigamma function )
Skewness	\scriptstyle \frac{2}{\sqrt{k}}\,\!
Ex. kurtosis	\scriptstyle \frac{6}{k}\,\!
Entropy	\scriptstyle \begin{align}
                      \scriptstyle k &\scriptstyle \,+\, \ln\theta \,+\, \ln[\Gamma(k)]\\
                      \scriptstyle   &\scriptstyle \,+\, (1 \,-\, k)\psi(k)
                    \end{align}

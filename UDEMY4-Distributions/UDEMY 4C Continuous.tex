
% Udemy Introductory Statistcs
% Probability Distribution


% See "book", "report", "letter" for other types of document.

\documentclass[11pt]{beamer} % use larger type; default would be 10pt
\usepackage{amsmath}
\usepackage{amssymb}
\usepackage{graphicx} % support the \includegraphics command and options



\begin{document}

%-----------------------------------------%
\begin{frame}
\frametitle{ Continuous Distributions}
\begin{itemize}
\item[(a)] The continuous uniform distribution
\item[(b)] The exponential distribution
\item[(c)] The Weibull distribution
\item[(d)] The Pareto distribution
\end{itemize}
\end{frame}




\section{Other useful Continuous Distributions}
%http://www.computing.dcu.ie/~jhorgan/chapter16slides.pdf

%Related to positive real-valued quantities that grow exponentially (e.g. prices, incomes, populations)[edit]
%---------------------------------------------------------------------------------%
\begin{frame}
\frametitle{Other useful Continuous Distributions}
\begin{description}
\item [Pareto distribution] for a single such quantity whose log is exponentially distributed; 
the prototypical power law distribution
\item [Log-normal distribution] for a single such quantity whose log is normally distributed
\item [Weibull distribution]
\end{description}
\end{frame}
%---------------------------------------------------------------------------------%

%\subsection{Pareto Distribution}
%\subsection{Log-Normal Distribution}
\begin{frame}
\frametitle{Weibull Distribution}
The Weibull distribution is used
\begin{itemize}
\item in survival analysis
\item in reliability engineering and failure analysis
\item in industrial engineering to represent manufacturing and delivery times
\item in extreme value theory
\item in weather forecasting
\end{itemize}
\end{frame}
\begin{frame}
The probability density function of a Weibull random variable x is:

\[
f(x;\lambda,k) =
\begin{cases}
\frac{k}{\lambda}\left(\frac{x}{\lambda}\right)^{k-1}e^{-(x/\lambda)^{k}} & x\geq 0 ,\\
0 & x<0,
\end{cases}
\]
where $k > 0$ is the shape parameter and $\lambda > 0$ is the scale parameter of the distribution.
%---------------------------------------------------------------------------------%


%----------------------------------------------------------------------------------%
\end{frame}


%-----------------------------------------%
\begin{frame}
\frametitle{Important Characteristics of a Continuous Probability Distribution }
\begin{itemize}
\item Expected value $E(X)$
\item Variance
\item probability density function ($f(X)$)
\item cumulative distribution function ($F(X)$)
\end{itemize}
\end{frame}
%----------------------------------------- %
\begin{frame}
\frametitle{Exponential Distribution : Expected Value}
Expected Value of an exponentially distributed random variable $X$, specifed with the \textbf{rate parameter} $\lambda$
\[ X \sim exp(\lambda)  \]
is computed using the following formula
\[ E(X) = \frac{1}{\lambda} \]

\end{frame}

%-----------------------------------------%
\begin{frame}
\frametitle{Uniform Distribution : Expected Value}
Expected Value of a uniformly distributed random variable $X$, specifed with (with maximum value $b$ and minimum value $a$ , i.e.
\[ X \sim U(a,b) \]
is computed using the following formula
\[ E(X) = \frac{a+b}{2} \]

\end{frame}


\end{document}
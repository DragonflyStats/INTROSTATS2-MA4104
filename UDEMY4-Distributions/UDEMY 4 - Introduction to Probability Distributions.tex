\documentclass{beamer}

\usepackage{amsmath}
\usepackage{amssymb}

\begin{document}

\section{Introduction to Distributions}
\begin{frame}
\frametitle{Random Variables}
\begin{itemize}
\item A random variable is a variable whose value is determined by the outcome of a random phenomenon.
The statistical techniques we've learned so far deal with variables, not events, so we need to define a
variable in order to analyze a random phenomenon.
\item A discrete random variable has a finite number of possible values, while a continuous random variable
can take all values in a range of numbers.
\end{itemize}
\end{frame}
%------------------------------------------------------------------------------------------ %
\begin{frame}

\begin{itemize}
\item The probability distribution of a random variable tells us the possible values of the variable and how
probabilities are assigned to those values.
\item The probability distribution of a discrete random variable is typically described by a list of the
values and their probabilities. Each probability is a number between 0 and 1, and the sum of the
probabilities must be equal to 1.
\end{itemize}
\end{frame}
%------------------------------------------------------------------------------------------ %

\begin{frame}
\Large
\begin{itemize}
\item The probability distribution of a continuous random variable is typically described by a density
curve. \item The curve is defined so that the probability of any event is equal to the area under the
curve for the values that make up the event, and the total area under the curve is equal to 1. \item One
example of a continuous probability distribution is the normal distribution.
\end{itemize}
\end{frame}
%------------------------------------------------------------------------------------------ %

\begin{frame}
\Large
\begin{itemize}
\item We use the term parameter to refer to a number that describes some characteristic of a population. \item  We
rarely know the true parameters of a population, and instead estimate them with statistics. \item Statistics
are numbers that we can calculate purely from a sample. \item The value of a statistic is random, and will
depend on the specific observations included in the sample.
\end{itemize}
\end{frame}
%------------------------------------------------------------------------------------------ %
\begin{frame}
\begin{itemize}
\item The law of large numbers states that if we draw a bunch of numbers from a population with mean ¹,
we can expect the mean of the numbers $\bar{y}$ to be closer to $\mu$ as we increase the number we draw. This
means that we can estimate the mean of a population by taking the average of a set of observations
drawn from that population.
\end{itemize}
\end{frame}
\begin{frame}
\frametitle{Discrete random variables}
\begin{itemize}
\item For a discrete random variable observed values can occur only at isolated points along a scale of values. \item For a six sided dice, the only possible observed values are 1, 2, 3, 4, 5 and 6. It is not possible to observe values such as 5.732.
\end{itemize}
\end{frame}
%------------------------------------------------------------------------------------------ %
\begin{frame}
\frametitle{Discrete Random Variables}
\Large
\vspace{-1cm}
\begin{itemize}
\item Therefore, it is possible that all numerical values for the variable can be listed in a table with accompanying
probabilities. There are several standard probability distributions that can serve as models for a wide variety of discrete random variables involved in business applications. \item  The standard models described in this course are
the binomial, hypergeometric, and Poisson probability distributions.
\end{itemize}
\end{frame}
%------------------------------------------------------------------------------------------ %
\begin{frame}
\frametitle{Discrete Random Variables}
\Large
\vspace{-1cm}
\begin{itemize}
\item For a continuous random variable all possible fractional values of the variable cannot be listed, and
therefore the probabilities that are determined by a mathematical function are portrayed graphically by a
probability density function, or probability curve.
\end{itemize}
\end{frame}
%------------------------------------------------------------------------------------------ %

\end{document}
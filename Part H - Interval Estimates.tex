
Confidence Interval Estimation
In statistics one often would like to estimate unknown parameters for a known distribution. For example, you may think that your parent population is normal, but the mean is unknown, or both the mean and standard deviation are unknown. From a data set you can't hope to know the exact values of the parameters, but the data should give you a good idea what they are. For the mean, we expect that the sample mean or average of our data will be a good choice for the population mean, and intuitively, we understand that the more data we have the better this should be. How do we quantify this?

Statistical theory is based on knowing the sampling distribution of some statistic such as the mean. This allows us to make probability statements about the value of the parameters, such as we are 95 per cent certain the parameter is in some range of values.

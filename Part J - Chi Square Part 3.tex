
\documentclass[a4]{beamer}
\usepackage{amssymb}
\usepackage{graphicx}
\usepackage{subfigure}
\usepackage{newlfont}
\usepackage{amsmath,amsthm,amsfonts}
\usepackage{beamerthemesplit}
\usepackage{pgf,pgfarrows,pgfnodes,pgfautomata,pgfheaps,pgfshade}
\usepackage{mathptmx}  % Font Family
\usepackage{helvet}   % Font Family
\usepackage{color}

\mode<presentation> {
 \usetheme{Default} % was
 \useinnertheme{rounded}
 \useoutertheme{infolines}
 \usefonttheme{serif}
 %\usecolortheme{wolverine}
% \usecolortheme{rose}
\usefonttheme{structurebold}
}

\setbeamercovered{dynamic}

\title[Stats-Lab.com]{\LARGE Introduction to Statistics and Probability \\ {\Large Chi-Square Test : }}
\author[Kevin O'Brien]{Kevin O'Brien}
\date{Spring 2014}


\renewcommand{\arraystretch}{1.5}

\begin{document}


\begin{frame}
\titlepage
\end{frame}
\begin{frame}
\frametitle{Chi-Square Test of Association}

\Large
\textbf{Critical Value}

\begin{itemize}
\item Signifiance Level $\alpha$
\item Degrees of freedom $\nu$ (also referred to as $d.f.$)
\end{itemize}

\end{frame}
%---------------------------------------------%
\begin{frame}
\frametitle{Chi-Square Test of Association}
\Large
\vspace{-1cm}
The degrees of freedom $\nu$

\[ \nu  = (r-1) \times (c-1) \]


\begin{itemize}
\item $r$ number of rows in frequency table
\item $c$ number of columns in frequency table
\end{itemize}

\end{frame}
%---------------------------------------------%
\end{document}
\documentclass{beamer}

\usepackage{amsmath}
\usepackage{amssymb}

\begin{document}


\begin{frame}
\frametitle{Permutations}
The number of permutations of n objects is the number of ways in which the objects can be arranged in
terms of order: \\
Permutations of n objects = $ n! = (n) \times (n - 1) \times (n-2) dots  \times (2) \times (1)$
\\

The symbol n! is read ``n factorial". In permutations and combinations problems, n is always positive.
Also, note that by definition $0! = 1$ in mathematics.
\end{frame}

\section{Combinations}
\begin{frame}
In the case of permutations, the order in which the objects are arranged is important. In the case of
combinations, we are concerned with the number of different groupings of objects that can occur without regard
to their order. Therefore, an interest in combinations always concerns the number of different subgroups that
can be taken from n objects. The number of combinations of n objects taken r at a time is

\end{frame}
\end{document}
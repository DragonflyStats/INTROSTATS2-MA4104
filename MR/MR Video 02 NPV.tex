\documentclass{beamer}

\usepackage{amsmath}
\usepackage{amssymb}

\begin{document}


%------------------------------------------------%
\begin{frame}{MathsResource.com}{Net Present Value}
\Large
\begin{itemize}
\item Each cash inflow/outflow is discounted back to its present value (PV). 
\item Then they are summed. Therefore NPV is the sum of all terms,
\[ \frac{R_t}{(1+i)^{t}} \]
where
\begin{description}
\item[t] - the time of the cash flow
\item[i] - the discount rate (the rate of return that could be earned on an investment in the financial markets with similar risk.); 
\end{description}
\end{frame}

%------------------------------------------------%
\begin{frame}{MathsResource.com}{Net Present Value}
\Large
\begin{itemize}
\item the opportunity cost of capital
$R_t$ - the net cash flow i.e. cash inflow – cash outflow, at time t . 
\end{itemize}
\end{frame}
%------------------------------------------------%
\begin{frame}{MathsResource.com}{Net Present Value}
\Large
\begin{itemize}
\item For educational purposes, $R_0$ is commonly placed to the left of the sum to emphasize its role as (minus) the investment.
\item The result of this formula is multiplied with the Annual Net cash in-flows and reduced by Initial Cash outlay the present value but in cases where the cash flows are not equal in amount, then the previous formula will be used to determine the present value of each cash flow separately. 
\end{itemize}

\end{frame}

%------------------------------------------------%
\begin{frame}{MathsResource.com}{Net Present Value}
\Large
\begin{itemize}
\item Any cash flow within 12 months will not be discounted for NPV purpose, nevertheless the usual initial investments during the first year R0 are summed up a negative cash flow.

\item Given the (period, cash flow) pairs $(t, R_t)$ where N is the total number of periods, the net present value $\mathrm{NPV}$ is given by:
\[\mathrm{NPV}(i, N) = \sum_{t=0}^{N} \frac{R_t}{(1+i)^{t}}\]
\end{itemize}
\end{frame}

%------------------------------------------------%
\begin{frame}
\frametitle{Net Present Value}
\Large
\begin{centering}
\begin{tabular}{|c|l|} \hline
$E_d = 0$ & Perfectly inelastic demand \\
$- 1 < E_d < 0 $ &Inelastic or relatively inelastic demand\\
$E_d=  - 1 $& Unit elastic\\
$ - \infty < E_d < - 1$ & 
Elastic or relatively elastic demand \\
$E_d = \infty $&
Perfectly elastic demand \\ \hline
\end{tabular} 
\end{centering}
\end{frame}


\begin{frame}
\[ \epsilon = \frac{\operatorname d Q/Q}{\operatorname d P/P} \]



 Revenue is simply the product of unit price times quantity:
  \[ \mbox{Revenue} = PQ_d\]
\end{frame}
\end{document}

Maths Resource

http://www.csusm.edu/mathlab/documents/M132BusCalcFormulas%20r1-12e.pdf
http://www.math.ubc.ca/~chau/elasticity.pdf
http://www.textbooksonline.tn.nic.in/books/12/std12-bm-em-1.pdf



%========================================================================== %
\begin{frame}[fragile]
\frametitle{Chi-Square Goodness of Fit}
\begin{itemize}
\item Twenty-five machine operators produce items that are sometimes
defective. 
\item Random samples of the production of each operator give the number of
defective items each produces out of 10 items sampled. 
\end{itemize}
\end{frame}
%========================================================================== %
\begin{frame}[fragile]
\frametitle{Chi-Square Goodness of Fit}
These numbers are as below:
\begin{framed}
\begin{verbatim}
2 0 0 1 0 0 1 0 0 1
1 1 0 0 3 0 1 2 0 0
1 3 1 2 0
\end{verbatim}
\end{framed}


Do these results give enough evidence to show that the operators do not all have a 10\%
rate of producing defective items?

\end{frame}
%========================================================================== %
\begin{frame}[fragile]
	\frametitle{Chi-Square Goodness of Fit}
	
We suppose that each operator is independent of the others, and that items are
produced independently of each other, and that the rate of production of defective
items is constant for each operator. The Null Hypothesis is that the rate of production
of defective items is 10\% for each operator. Then each of the numbers in the table


\end{frame}
%========================================================================== %
\begin{frame}[fragile]
	\frametitle{Chi-Square Goodness of Fit}
We suppose that each operator is independent of the others, and that items are
produced independently of each other, and that the rate of production of defective
items is constant for each operator. The Null Hypothesis is that the rate of production
of defective items is 10\% for each operator. Then each of the numbers in the table





%----------------------------------------------------------%
\begin{frame}
\frametitle{Chi-Square : Expected Values}

The expected value for cell (i,j) is computed as
\[  \frac{\mbox{Row Total}\times \mbox{Column Total}}{\mbox{Overall Total}} \]


% 15 &   35  &  50 


\end{frame}
%----------------------------------------------------------%
\begin{frame}
\frametitle{Chi Square Test}
Determine the contribution of each cell to the overall test statistics


\[  \frac{\mbox{Row Total}\times \mbox{Column Total}}{\mbox{Overall Total}} \]

\end{frame}
%----------------------------------------------------------%
\begin{frame}
\frametitle{Chi Square Test = Computing the critical value}
The degrees of freedom ( df or $\eta$ ) is computed as

\[ \eta = (r-1)\times(c-1) \]

\begin{itemize}
\item $r$ number of rows
\item $c$ number of columns
\end{itemize}
\end{frame}

%----------------------------------------------------------%
\begin{frame}

END SLIDE


\end{frame}
%----------------------------------------------------------%
\end{document}

Tags
Chi Square, Chi-Square, Chi-Squared , Chi-Square test, Chi-squared test, Chi-square test for relationship
testing association, Chi testing, Chi square hypothesis test, X squared, hypothesis testing,
Chi Square hypothesis test,

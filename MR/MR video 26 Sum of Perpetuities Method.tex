\documentclass{beamer}

\usepackage{amsmath}
\usepackage{amssymb}

\begin{document}


\begin{frame}
\Large
\[
\mbox{Continuous Compounding}
\]
\end{frame}

%------------------------------------------------%
%----------------------------------------------------
\begin{frame}
\frametitle{Sum of Perpetuities Method}

The PEG ratio is a special case in the Sum of Perpetuities Method (SPM) equation. 
A generalized version of the Walter model (1956), 
SPM considers the effects of dividends, earnings growth, as well as 
the risk profile of a firm on a stock's value. 
Derived from the compound interest formula using the present value of a 
perpetuity equation, SPM is an alternative to the Gordon Growth Model. 

\end{frame}
%----------------------------------------------------
\begin{frame}
\frametitle{Sum of Perpetuities Method}
The variables are:

\begin{itemize}
\item P is the value of the stock or business
\item E is a company's earnings
\item G is the company's constant growth rate
\item K is the company's risk adjusted discount rate
\item D is the company's dividend payment
\end{itemize}

\[ P = (\frac{E\times G}{K^2}) + (\frac{D}{K})\]

In a special case where K is equal to 10\%, and the company does not pay dividends, SPM reduces to the PEG ratio.

\end{frame}



\end{document}

Maths Resource

http://www.csusm.edu/mathlab/documents/M132BusCalcFormulas%20r1-12e.pdf
http://www.math.ubc.ca/~chau/elasticity.pdf
http://www.textbooksonline.tn.nic.in/books/12/std12-bm-em-1.pdf

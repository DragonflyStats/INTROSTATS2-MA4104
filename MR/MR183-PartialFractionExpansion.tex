%---------------------------------------------------------%
\begin{frame}
\[(3x + 5)/(1 - 2x)^2\] can be decomposed in the form
\[ \frac{3x + 5}{(1-2x)^2} = \frac{A}{(1-2x)^2} + \frac{B}{(1-2x)}. \]
Clearing denominators shows that 3x + 5 = A + B(1 - 2x). 

\end{frame}
%---------------------------------------------------------%
\begin{frame}
Expanding and equating the coefficients of powers of x gives

5 = A + B and 3x = -2Bx

Solving for A and B yields A = 13/2 and B = -3/2. Hence,


\[  \frac{3x + 5}{(1-2x)^2} = \frac{13/2}{(1-2x)^2} + \frac{-3/2}{(1-2x)}. \]
\end{frame}
%---------------------------------------------------------%






%-----------------------------------------------------------------
\begin{frame}
\frametitle{Partial Fractions}

Example 1[edit]
\[ f(x)=\frac{1}{x^2+2x-3} \]
Here, the denominator splits into two distinct linear factors:
q(x)=x^2+2x-3=(x+3)(x-1)
so we have the partial fraction decomposition
f(x)=\frac{1}{x^2+2x-3} =\frac{A}{x+3}+\frac{B}{x-1}
\end{frame}
%-----------------------------------------------------------------
\begin{frame}
\frametitle{Partial Fractions}
Multiplying through by x2 + 2x - 3, we have the polynomial identity
\[ 1=A(x-1)+B(x+3) \]
Substituting x = -3 into this equation gives A = -1/4, and substituting x = 1 gives B = 1/4, so that
\[f(x) =\frac{1}{x^2+2x-3} =\frac{1}{4}\left(\frac{-1}{x+3}+\frac{1}{x-1}\right) \]
\end{frame}
%-----------------------------------------------------------------
\begin{frame}
\frametitle{Partial Fractions}

Example 2
\[f(x)=\frac{x^3+16}{x^3-4x^2+8x}\]
After long-division, we have
\[ f(x)=1+\frac{4x^2-8x+16}{x^3-4x^2+8x}=1+\frac{4x^2-8x+16}{x(x^2-4x+8)} \]
Since (-4)2 - 4(8) = -16 < 0, x2 - 4x + 8 is irreducible, and so
\[ \frac{4x^2-8x+16}{x(x^2-4x+8)}=\frac{A}{x}+\frac{Bx+C}{x^2-4x+8} \]
\end{frame}
%-----------------------------------------------------------------
\begin{frame}
\frametitle{Partial Fractions}
Multiplying through by x3 - 4x2 + 8x, we have the polynomial identity
\[ 4x^2-8x+16 = A(x^2-4x+8)+(Bx+C)x \]
Taking x = 0, we see that 16 = 8A, so A = 2. 

Comparing the x2 coefficients, we see that 4 = A + B = 2 + B, so B = 2. 
\end{frame}
%-----------------------------------------------------------------
\begin{frame}
\frametitle{Partial Fractions}
Comparing linear coefficients, we see that -8 = -4A + C = -8 + C, so C = 0. Altogether,
\[ f(x)=1+2\left(\frac{1}{x}+\frac{x}{x^2-4x+8}\right)  \]
The following example illustrates almost all the "tricks" one would need to use short of consulting a computer algebra system.
\end{frame}
%-----------------------------------------------------------------%
\begin{frame}
Example 3
f(x)=\frac{x^9-2x^6+2x^5-7x^4+13x^3-11x^2+12x-4}{x^7-3x^6+5x^5-7x^4+7x^3-5x^2+3x-1}
After long-division and factoring the denominator, we have
\[ f(x)=x^2+3x+4+\frac{2x^6-4x^5+5x^4-3x^3+x^2+3x}{(x-1)^3(x^2+1)^2} \]

\end{frame}
%-----------------------------------------------------------------
\begin{frame}
\frametitle{Partial Fractions}

The partial fraction decomposition takes the form
\[ \frac{2x^6-4x^5+5x^4-3x^3+x^2+3x}{(x-1)^3(x^2+1)^2}=\frac{A}{x-1}+\frac{B}{(x-1)^2}+\frac{C}{(x-1)^3}+\frac{Dx+E}{x^2+1}+\frac{Fx+G}{(x^2+1)^2} \]
Multiplying through by (x - 1)3(x2 + 1)2 we have the polynomial identity
\end{frame}
%-----------------------------------------------------------------
\begin{frame}
\frametitle{Partial Fractions}

\begin{align}
& {} \quad 2x^6-4x^5+5x^4-3x^3+x^2+3x \\
& =A(x-1)^2(x^2+1)^2+B(x-1)(x^2+1)^2+C(x^2+1)^2+(Dx+E)(x-1)^3(x^2+1)+(Fx+G)(x-1)^3
\end{align}
Taking x = 1 gives 4 = 4C, so C = 1. Similarly, taking x = i gives 2 + 2i = (Fi + G)(2 + 2i), so Fi + G = 1, so F = 0 and G = 1 by equating real and imaginary parts. With C = G = 1 and F = 0, taking x = 0 we get A - B + 1 - E - 1 = 0, thus E = A - B.
We now have the identity
\end{frame}
%-----------------------------------------------------------------
\begin{frame}
\frametitle{Partial Fractions}

\begin{align}
 & {} 2x^6-4x^5+5x^4-3x^3+x^2+3x \\
 & = A(x-1)^2(x^2+1)^2+B(x-1)(x^2+1)^2+(x^2+1)^2+(Dx+(A-B))(x-1)^3(x^2+1)+(x-1)^3 \\
 & = A((x-1)^2(x^2+1)^2 + (x-1)^3(x^2+1)) + B((x-1)(x^2+1) - (x-1)^3(x^2+1)) + (x^2+1)^2 + Dx(x-1)^3(x^2+1)+(x-1)^3
\end{align}
Expanding and sorting by exponents of x we get
\end{frame}
%-----------------------------------------------------------------
\begin{frame}
\frametitle{Partial Fractions}

\begin{align}
 & {} 2 x^6 -4 x^5 +5 x^4 -3 x^3 + x^2 +3 x \\
 & = (A + D) x^6 + (-A - 3D) x^5 + (2B + 4D + 1) x^4 + (-2B - 4D + 1) x^3 + (-A + 2B + 3D - 1) x^2 + (A - 2B - D + 3) x 

\end{align}
We can now compare the coefficients and see that
\end{frame}
%----------------------------------------------------------------- 
\begin{frame}
\frametitle{Partial Fractions}

\begin{align}
 A + D &=& 2  \\
 -A - 3D &=&  -4 \\
2B + 4D + 1 &=& 5 \\
-2B - 4D + 1 &=& -3 \\
-A + 2B + 3D - 1 &=& 1 \\
A - 2B - D + 3 &=& 3 ,
\end{align}
\end{frame}
%-----------------------------------------------------------------
\begin{frame}
\frametitle{Partial Fractions}
with A = 2 - D and -A -3 D =-4 we get A = D = 1 and so B = 0, furthermore is C = 1, E = A - B = 1, F = 0 and G = 1.
The partial fraction decomposition of ƒ(x) is thus
f(x)=x^2+3x+4+\frac{1}{(x-1)} + \frac{1}{(x - 1)^3} + \frac{x + 1}{x^2+1}+\frac{1}{(x^2+1)^2}.
Alternatively, instead of expanding, one can obtain other linear dependences on the coefficients computing 
some derivatives at x=1 and at x=i in the above polynomial identity. 
\end{frame}
%-----------------------------------------------------------------
\begin{frame}
\frametitle{Partial Fractions}
(To this end, recall that the derivative at x=a of (x-a)mp(x) vanishes if m > 1 and it is just p(a) if m=1.) Thus, for instance the first derivative at x=1 gives
 2\cdot6-4\cdot5+5\cdot4-3\cdot3+2+3   = A\cdot(0+0) + B\cdot( 2+ 0) + 8 + D\cdot0 
that is 8 = 2B + 8 so B=0.
Example 4 (residue method)[edit]
 f(z)=\frac{z^{2}-5}{(z^2-1)(z^2+1)}=\frac{z^{2}-5}{(z+1)(z-1)(z+i)(z-i)}
\end{frame}
%-----------------------------------------------------------------
\begin{frame}
\frametitle{Partial Fractions}
Thus, f(z) can be decomposed into rational functions whose denominators are z+1, z-1, z+i, z-i. Since each term is of power one, -1, 1, -i and i are simple poles.
Hence, the residues associated with each pole, given by
\[
\frac{P(z_i)}{Q'(z_i)} = \frac{z_i^2 - 5}{4z_i^3}\]
are
\[ 1, -1, \tfrac{3i}{2}, -\tfrac{3i}{2}\]
respectively, and
\[ f(z)=\frac{1}{z+1}-\frac{1}{z-1}+\frac{3i}{2}\frac{1}{z+i}-\frac{3i}{2}\frac{1}{z-i}\]

\end{frame}

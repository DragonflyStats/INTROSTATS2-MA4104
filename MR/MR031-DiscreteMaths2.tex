%-------------------------------------------------%
\section{Logic}

\begin{frame}
\frametitle{Logic}
\begin{itemize}
\item Important Logical Operators
\item Conditional Connectives
\item Logic Tables
\item Using Logic Tables for Proofs
\item Logical Gate Networks
\end{itemize}
\end{frame}

%==================================================================================================%
\begin{frame}
\frametitle{Important Logical Operators}

\begin{description}
\item[$p \vee q$] $p$ and $q$
\item[$p \wedge q$] $p$ or $q$
\item[$\neg p$] Not P
\end{description}
\end{frame}
%=================================================================================================%
\subsection*{Logic Gates}

\begin{description}
\item AND gates
\item OR gates
\item XOR gates
\item NOT gates
\end{description}
\newpage
%-------------------------------------------------%
\section{Numbers and Number Systems}
\begin{itemize}
\item Binary Numbers
\item Hexadecimal Numbers
\item Octal Numbers and Base 5 Numbers
\end{itemize}

\subsection*{Hexadecimal Numbers}
Hexadecimal numbers are commonly used to represent memory addresses in computer systems.
 
 
Hexadecimal Numbers uses sixteen distinct symbols, most often the symbols 0–9 to represent values zero to nine, and A, B, C, D, E, F (or alternatively a–f) to represent values ten to fifteen. For example, the hexadecimal number 2AF3 is equal, in decimal, to $(2 \times 16^3)$ + $(10 \times 16^2)$ + $(15 \times 16^1$) + ($3 \times 16^0$), or 10995.

%------------------------------------------------
\section{Graph Theory}
\begin{itemize}
\item Introduction to Graph Theory
\item KEy Terms and Definitions in Graph Theury
\item Isomorphism
\item Digraphs
\end{itemize}


\section{Proof by Induction}


\section{Relations and Functions}
\begin{itemize}
\item Ordered Pairs
\item Cartesian Products
\end{itemize}

\subsection*{Partial Ordering Relations}
A relation R on a set S is called a partial ordering or a partial order of S if R is reflexive, antisymmetric, and
transitive. A set S together with a partial ordering R is called a partially ordered set or poset.

\end{document}

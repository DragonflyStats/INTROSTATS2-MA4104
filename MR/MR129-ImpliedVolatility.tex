\documentclass{beamer}

\usepackage{amsmath}
\usepackage{amssymb}

\begin{document}


\begin{frame}
\Large
\[
\mbox{Continuous Compounding}
\]
\end{frame}

%------------------------------------------------%
\begin{frame}


\frametitle{Implied Volatility}
A European call option, C_{XYZ} \,, on 100 shares of non-dividend-paying XYZ Corp. The option is struck at $50 and expires in 32 days. The risk-free interest rate is 5\%. XYZ stock is currently trading at $51.25 and the current market price of C_{XYZ} \, is $2.00. 
\end{frame}

%------------------------------------------------%
\begin{frame}


\frametitle{Implied Volatility}

Using a standard Black–Scholes pricing model, the volatility implied by the market price C_{XYZ} \, is 18.7\%, or:
\[ \sigma_\bar{C} = g(\bar{C}, \cdot) = 18.7\% \,  \]
To verify, we apply the implied volatility back into the pricing model, f and we generate a theoretical value of $2.0004:
\[ C_{theo} = f(\sigma_\bar{C}, \cdot) = $2.0004 \,  \]
which confirms our computation of the market implied volatility.


\end{frame}

%------------------------------------------------%
\begin{frame}

\frametitle{Continuous Compounding}

\end{frame}



\end{document}
%----------------------------------------------%
%----------------------------------------------%
Maths Resource

http://www.csusm.edu/mathlab/documents/M132BusCalcFormulas%20r1-12e.pdf
http://www.math.ubc.ca/~chau/elasticity.pdf
http://www.textbooksonline.tn.nic.in/books/12/std12-bm-em-1.pdf

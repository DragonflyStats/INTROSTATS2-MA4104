\documentclass{beamer}
\usepackage{amsmath}
\usepackage{amssymb}
\begin{document}
%WACC
%CISI IAN 


\begin{frame}
\frametitle{Binary or digital options}
may be the simplest form of exotic contracts: they only differ from the vanilla options by the terminal payoff L(S)   that can be any positive function of the asset price S. Some binaries can be obtained from the superposition of vanilla options: straddles, bullish / bearish vertical spreads and butterfly spreads are the subject of exercises 2.05-2.07. Others have payoffs that remind well known functions, such as the cash-or-nothing call reproducing the Heavyside H(x)
 $\displaystyle \Lambda_\mathrm{cash-or-nothing} =b \mathcal{H}(S-K)$	 


\end{frame}
%-----------------------------------------------
\begin{frame}
and the supershare reminding the Dirac delta function
 $\displaystyle \Lambda_\mathrm{supershare} =\frac{1}{d}[\mathcal{H}(S-K) -\mathcal{H}(S-K-d)]$	 (2.1.4#eq.2)


with terminal payoff diagrams illustrated in (2.1.4#fig.1).
Figure 2.1.4#fig.1: Example of binary / digital options with a general terminal payoff  $ \Lambda (S)$ : a cash-or-nothing call (left) and a super-share (right).


\end{frame}
%-----------------------------------------------
\begin{frame}
This generalization does not present any formal difficulty, but the discontinuities in terminal payoff L(S)   do however seriously stretch the non-arbitrage arguments that will be used to derive the Black-Scholes equation in chapter 3. Indeed, a large amount of cash would be required to hedge small changes in the price of the underlying as the option jumps from zero to a finite value, only to fall back to zero shortly afterwards.

\end{frame}
%-----------------------------------------------
\begin{frame}
\frametitle{Barrier options}

are characterized by a a condition set on the existence of the option. When triggered, the right to exercise the option either appears (in) or disappears (out) if the asset price is above (up) or below (down) a prescribed barrier B:
up-and-in options come into existence if S > B before expiry,  
up-and-out options cease to exist if S > B before expiry,  
down-and-in options come into existence if S < B before expiry,  
down-and-out options cease to exist if S < B before expiry.  
\end{frame}
%-----------------------------------------------
\begin{frame}
\frametitle{Barrier options}
Barrier options can be further complicated by making the knockout boundary a function of time B(t) or by having a rebate if the barrier is activated. In the latter case, the holder of the option receives a specified amount if the barrier is reached.
\end{frame}

%--------------------------------------------------------------------------%
\end{document}


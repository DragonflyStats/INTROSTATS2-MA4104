\documentclass[12pt]{article}

% http://www.cims.nyu.edu/~regev/teaching/discrete_math_fall_2005/dmbook.pdf

% http://math.about.com/gi/o.htm?zi=1/XJ&zTi=1&sdn=math&cdn=education&tm=17&f=10&tt=14&bt=8&bts=8&zu=http%3A//www.ping.be/~ping1339/tel.htm


%opening
\title{Discrete Mathematics}
\author{Kevin O'Brien}

\begin{document}

\maketitle

\begin{abstract}
Discrete Mathematics Udemy Course
\end{abstract}


%-------------------------------------------------%
\section{Set Theory}

\begin{itemize}
\item Set Operations
\item Power Sets
\item Subsets
\item Venn Diagrams
\end{itemize}

\subsection*{Introduction to Sets}
%http://people.umass.edu/partee/NZ_2006/Set%20Theory%20Basics.pdf
A set is a collection of objects which are called the members or elements of 
that set. If we have a set we say that some objects belong (or do not belong) to this set, are
(or are not) in the set. We say also that sets consist of their elements. 


\subsection*{Specifying Sets}

There are three main ways to specify a set: 

\begin{itemize}
\item[(1)] by listing all its members (list notation); 
\item[(2)] by stating a property of its elements (predicate notation); 
\item[(3)] by defining a set of rules which generates (defines) its members (recursive rules).
\end{itemize}

\begin{itemize}
\item $\emptyset$ The empty set
\item $\mathcal{p}(S)$ The power set
\end{itemize}


%-------------------------------------------------%
\section{Counting}
\begin{itemize}
\item Combinations
\item Permutations
\item The choose operator
\end{itemize}

\subsection*{Factorial Function}
The product of the positive integers from 1 to n inclusive is denoted by $n!$, read “n factorial.” Namely:
\[n! = 1 \times 2 \times 3 \times \ldots \times (n−2) \times(n−1)\times n\] 

Accordingly, 1! = 1 and $n! = n(n − l)!$. 

It is also convenient to define 0! = 1.


\[ {n \choose k} = \frac{n!}{k! \times (n-k)!} \]


\subsection*{Permutation}
Any arrangement of a set of n objects in a given order is called a permutation of the object (taken all at a time).

\newpage

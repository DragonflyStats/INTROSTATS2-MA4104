\documentclass[a4]{beamer}
\usepackage{amssymb}
\usepackage{graphicx}
\usepackage{subfigure}
\usepackage{newlfont}
\usepackage{amsmath,amsthm,amsfonts}
%\usepackage{beamerthemesplit}
\usepackage{pgf,pgfarrows,pgfnodes,pgfautomata,pgfheaps,pgfshade}
\usepackage{mathptmx}  % Font Family
\usepackage{helvet}   % Font Family
\usepackage{color}

\mode<presentation> {
 \usetheme{Default} % was Frankfurt
 \useinnertheme{rounded}
 \useoutertheme{infolines}
 \usefonttheme{serif}
 %\usecolortheme{wolverine}
% \usecolortheme{rose}
\usefonttheme{structurebold}
}

\setbeamercovered{dynamic}

\title[MA4413]{Statistics for Computing \\ {\normalsize MA4413 Lecture 2B}}
\author[Kevin O'Brien]{Kevin O'Brien \\ {\scriptsize Kevin.obrien@ul.ie}}
\date{Autumn Semester 2012}
\institute[Maths \& Stats]{Dept. of Mathematics \& Statistics, \\ University \textit{of} Limerick}

\renewcommand{\arraystretch}{1.5}

\begin{document}

\begin{frame}
\titlepage
\end{frame}


%--------------------------------------------------%
\frame{
\begin{itemize}
\item Computing Sample Variance and Standard Deviation
\item Quartiles
\item The Interquartile Range
\item Outliers
\item Symmetry and Skew
\item Expected Values
\item Variance of Random Variables
\end{itemize}
}


%--------------------------------------------------%
\frame{
\frametitle{Computing Sample Variance}
Recall three data sets from yesterday's class:
\begin{itemize}
\item $X= \{900,925,950,975,1025,1050,1075,1100 \}$
\item $Y= \{900,905,910,920,1080,1090,1095,1100\}$
\item $Z= \{900,985,990,995,1005,1010,1015,1100\}$
\end{itemize}\bigskip
Recall that in each case the mean is $1000$.
\bigskip
\begin{itemize}
\item $Z$ will have comparatively low sample variance because most values are close to the mean.
\item $Y$ will have comparatively high sample variance because most values are quite far from the mean.
\item $X$ will have a sample variance between that of $Y$ and $Z$ because the values are evenly distributed.
\end{itemize}
}
%--------------------------------------------------------%

\frame{
\frametitle{Computing Sample Variance}


\begin{center}
\includegraphics[scale=0.4]{2AVariance}
\end{center}

}

\frame{
\frametitle{Computing Sample Variance}
We shall use the following formulae to compute the sample variance of each data set (i.e. $s^2_x$, $s^2_y$ and $s^2_z$) respectively.

\[ s^2_x = { \sum (x-\bar{x})^2  \over n-1}\]
\[ s^2_y = { \sum (y-\bar{y})^2  \over n-1}\]
\[ s^2_z = { \sum (z-\bar{z})^2  \over n-1}\]
\begin{itemize}
\item Mean for each is 1000: $\bar{x} = \bar{y} = \bar{z}  = 1000$
\item Sample size of each data set is 8 : $ n=8 $
\item Therefore $ n-1 = 7$
\end{itemize}
}
%--------------------------------------------------%
\frame{
\frametitle{Computing Sample Variance}
\small
\[ s^2_x = {(900-1000)^2 +(925-1000)^2+ \ldots \ldots +(1075-1000)^2+(1100-1000)^2   \over 7}\]

\[ s^2_x = {(-100)^2 +(-75)^2 +(-50)^2+(-25)^2 + (25)^2 +(50)^2 +(75)^2+(100)^2   \over 7}\]

\[ s^2_x = {37500 \over 7}  = 5357.143\]

\normalsize
\bigskip The sample variance of $X$ is $5357.14$ square units. Recall that the sample standard deviation ($s$) is the square root of the variance, so for $X$  the sample standard deviation is $s_x = 73.19$ units.

}

%--------------------------------------------------%
\frame{
\frametitle{Computing Sample Variance}
Similarly \small
\[ s^2_y = {67050 \over 7}  = 9578.571 \mbox{ square units} \]

\[ s^2_z = {20700  \over 7} = 2957.143 \mbox{ square units} \]


\normalsize
\bigskip The sample standard deviations for $Y$ and $Z$ are $s_y = 97.87$ units, and $s_z =  54.38$ units respectively.
}



%----------------------------------------------------------%#

\frame{

\frametitle{Quartiles}

\begin{itemize}
\item Quartiles are values that divide a sample of data into four groups containing (as far as possible) equal numbers of observations.
\item
A data set has three quartiles. References to quartiles often relate to just the outer two, the upper and the lower quartiles; the second quartile being equal to the median. \item The lower quartile $Q_1$  is the data value a quarter way up through the ordered data set \item The upper quartile $Q_3$ is the data value a quarter way down from the highest value in an ordered data set.
\end{itemize}
}

%----------------------------------------------------------%#

\frame{

\frametitle{Computing Quartiles}
\begin{itemize}
\item Yesterday we discussed how to compute the median for odd-sized and even-sized data sets respectively (i.e. middle value or average of middle pair of values)
\item To compute $Q1$ and $Q3$, it is best to consider them as the median of the lower half of values and higher half of values respectively. \bigskip
\item Consider the following data set (ordered with 10 items):  $\{ 6, 7 ,15, 36, 39, 41, 41 ,43, 43, 47 \}$
\item The lower half to the data set is : $\{ 6, 7 ,15, 36, 39 \}$
\item The upper half to the data set is : $\{ 41, 41 ,43, 43, 47 \}$
\end{itemize}
}
%----------------------------------------------------------%#

\frame{

\frametitle{Computing Quartiles}
\begin{itemize}
\item For both the lower and upper halves, there is an odd number of items contained.
\item Recall that the median of the odd-sized data set is the middle value of the ordered set.
\item The medians are 15 and 43 respectively.
\item So the first and third quartiles are $Q_1 = 15$ and $Q_3 = 43$ respectively.
\end{itemize}
}

%----------------------------------------------------------%#

\frame{
\frametitle{Computing Quartiles}
In the last example, the full data set comprised an even number of items, and it was easy to split it into an lower half and an upper half. Consider the following data set.
\begin{itemize}
\item Data:  $\{6, 47, 49, 15, 43, 41, 7, 39, 43, 41, 36\}$
\item Ordered Data: $\{ 6, 7 ,15, 36, 39, 41, 41 ,43, 43, 47, 49 \}$
\item Sample size: 11
\item Median:  41 (6th item)
\end{itemize}
How do we split up the ordered data into two halves here?
}
%----------------------------------------------------------%#

\frame{
\frametitle{Computing Quartiles}
\begin{itemize}
\item The lower half of values  \begin{itemize} \item[A] - could contain either the lowest 5 values (i.e. excluding the median)

    \[\{ 6, 7 ,15, 36, 39\}\]
    \item[B] - or the lowest six values (i.e. including the median).  \[\{ 6, 7 ,15, 36, 39,41\}\] \end{itemize}
\item Consequently the upper half of values \begin{itemize} \item[A] -  will contain either the highest 5 values (i.e. excluding the median) \[\{ 41 ,43, 43, 47, 49 \}\] \item[B] - or the highest 6 values (i.e. including the median).
    \[\{ 41, 41 ,43, 43, 47, 49 \}\] \end{itemize}

\end{itemize}

}
%----------------------------------------------------------%#

\frame{
\frametitle{Computing Quartiles}
\begin{itemize}
\item The value of $Q_1$ and $Q_3$ depend on which approach is used to calculate them.
\begin{itemize}
\item For option $A$ : $Q_1 = 15$ and $Q_3 = 43$.
\item For option $B$ : $Q_1 = 25.5$ and $Q_3 = 43$.
\end{itemize}
\item Unfortunately there is no consensus on which approach to take; different textbooks use different approaches.
\item In some computing environments, different commands yields different results for quartiles.
\item For larger data sets, the difference in outcomes is often negligible.
\item In this module, we will use option $A$ (i.e. excluding the median) only.
\end{itemize}

}
%----------------------------------------------------------%#

\frame{
\frametitle{Interquartile Range}
\begin{itemize}
\item The interquartile range (IQR) is measure of dispersion, that can be used as an alternative to variance/standard deviation.
\item It is computed as follows:  \[ \mbox{ IQR }  = Q_3 - Q_1 \]
\item For the data set used previously:
 \[ \mbox{ IQR }  = 43 - 15  = 28 \]
\end{itemize}

}



%--------------------------------------------------------%
\frame{
\frametitle{Quantiles}
Quartiles are just one type of statistic known as ``Quantiles".
Quantiles are values which divide the distribution such that there is a given proportion of observations below the quantile.\\ \bigskip
\textbf{Other Examples of Quantiles}\\
\begin{description}
\item[Deciles]  any of the nine values that divide the sorted data into ten equal parts,
\item[Quintiles] any of the four values that divide the sorted data into five equal parts,
\item[Percentiles]any value below which a certain percentage of observations fall,
\end{description}

Quantiles can be expressed in terms of other quantiles. For example, the first decile is equivalent to the $10\%$ percentile, $Q_1$, the median and $Q_3$ are equivalent to the $25\%$, $50\%$, $75\%$ percentile.

}
%--------------------------------------------------------%
\frame{
\frametitle{Tukey five-number summary}
The Tukey five-number summary is a statistical summary that provides information about a dataset.
The summary consists of the five most commonly used sample quantiles:

\begin{itemize}
\item the lowest value in the dataset
\item the first quartile ($Q_1$)
\item the median ($Q_2$)
\item the third quartile ($Q_3$)
\item the highest value in the dataset
\end{itemize}

}

%-------------------------------------------------------------------------%

\section{Skewness and Outliers}

\frame{
\frametitle{Outliers}

\begin{itemize}
\item
An outlier is an observation in a data set which is far removed in value from the others in the data set. It is an unusually large or an unusually small value compared to the others.
\item
An outlier might be the result of an error in measurement, in which case it will distort the interpretation of the data, having undue influence on many summary statistics, for example, the mean and variance.
\item Outliers are said to \textbf{\emph{skew}} the distribution of values.
\item
If an outlier is a genuine result, it is important because it might indicate an extreme of behaviour of the process under study. \item For this reason, all outliers must be examined carefully before embarking on any formal analysis. Outliers should not routinely be removed without further justification.
\end{itemize}
}

%-------------------------------------------------------------------------%
\frame{
\frametitle{Outliers}
Compute the sample mean and median of the following data set
\[X = \{ 5, 6, 7, 8 ,9,11, 15, 16, 94\}\]

\begin{itemize}
\item The sample mean $\bar{x} = 19$
\item The median of the sample is 9.
\item What causes the discrepancy between mean and median?

\item Which measure of centrality do you feel is more representative of the data?
\begin{itemize}
\item The median  - most values are between 5 and 16.
\end{itemize}
\end{itemize}
}

%-------------------------------------------------------------------------%
\frame{
\frametitle{Symmetric and Skewed Distributions }
\begin{itemize}
\item  A data set is said to be \textbf{\emph{symmetric}} when data values are distributed in the same way above and below the middle of the sample.
\item Typically , but not necessarily, distributions are considered to symmetric because the data set does not contain any outliers. (Potentially a symmetric data set may contain outliers on either side of the mean. )
\item When data sets are symmetrically distributed, the sample mean and median have values close to each other.
\item Distributions are considered to \textbf{\emph{skewed}} when the data set contains an outlier, or cluster of outliers, distributed away from the main cluster of items, such as in our example.
\item When data sets have skewed distribution, the sample mean and median have values different to each other.
\end{itemize}
}
%-------------------------------------------------------------------------%
\frame{
\frametitle{Symmetric and Skewed Distributions }
\begin{itemize}
\item When a data set is \textbf{\emph{symmetric}}, the best measure of centrality is the sample mean.
\item As the variance is computed using the sample mean, it is considered the best measure of dispersion for symmetrically distributed data. \bigskip
\item When a data set is \textbf{\emph{skewed}}, the best measure of centrality is the median.
\item The best measure of dispersion for data with skewed distribution is the IQR.
\end{itemize}
}

\section{Random Variables: Expected Value and Variance}

%--------------------------------------------------------%
\frame{

\frametitle{Expected Value for Discrete Random Variables}

\begin{itemize}

\item Consider the experiment of throwing a die repeatedly, where a record is kept of the mean score of all dice throws.
\item As the number of throws of the die, this average value will converge towards a specific value ( which is 3.5).
\item This value is the \textbf{\emph{expected value}} of all throws of a die.
\end{itemize}
}
%----------------------------------------------------------------%
\frame{
\frametitle{Expected Value for Discrete Random Variables}

\textbf{Recall from yesterday}

Suppose we roll a die 8 times and get the following scores: $x = \{ 5, 2, 1, 6, 3, 5, 3, 1\}$ \\ \bigskip

What is the sample mean of the scores $\bar{x}$?
\[ \bar{x}  = {5 + 2 +  1 +  6 +  3 +  5 +  3 +  1 \over 8 } = {26 \over 8} =  3.25 \]

Suppose we roll the dice a further 4 times, yielding $\{ 5, 6, 1, 4 \}$, what is the sample mean then?

\[ \bar{x}  = {5 + 2 +  1 +  6 +  3 +  5 +  3 +  1  + 5 + 6 + 1 + 3 \over 12 } = {41 \over 12} =  3.416 \]
}
%--------------------------------------------------------%

\frame{
\frametitle{Expected Value for Discrete Random Variables}
\begin{itemize}
\item As the number of throws increases, the average value will start to settle around a particular value.
\item On the plot on the following slide we can see this convergence.
\item The red line indicates the values of sample mean of throws as the number of throws increases.
\item The green line indicates the level to which the mean value will converge to.
\end{itemize}

}
%--------------------------------------------------------%


\begin{frame}[fragile]
\frametitle{Dice Simulation in \texttt{R}}
\begin{verbatim}
> #simulate 10 rolls of a die
> n = 10
> x = sample(1:6,n,replace=TRUE)
> x
 [1] 5 1 4 2 6 6 2 4 2 5
>
> #Compute the cumulative average
> cumsum(x)/1:n
[1] 5.00 3.00 3.33 3.00 3.60 4.00 3.71 3.75 3.56 3.70
> 
\end{verbatim}
We will return to sampling at a later date. Meanwhile, you can try out this code for larger values of $n$.
\end{frame}

%--------------------------------------------------------%
\frame{
\frametitle{Expected Value for Discrete Random Variables}

\begin{center}
\includegraphics[scale=0.4]{2BDieMean}
\end{center}

}
%--------------------------------------------------------%
\frame{
\frametitle{Thought Experiment}

\begin{itemize}
\item Suppose you roll a die 100 times. You record the outcome for each roll, and compute the sum of the 100 rolls at the end.
\item What sum do you expect to end up with?
\item The lowest sum that is mathematically possible is 100. The highest is 600. To get either of these, you would have to roll the same number (either 1 or 6) 100 times in a row. Very unlikely ( but not impossible).
\item Instead you would expect to end up half way between 100 and 600, i.e. 350.
\item Every roll of a die should be worth 3.5 on average to your overall score.
\end{itemize}
}
%--------------------------------------------------------%

\frame{

\frametitle{Expected Value for Discrete Random Variables}

\begin{itemize}
\item The expected value of a discrete random variable $X$ is symbolized by E(X).

\item If X is a discrete random variable with possible values $\{ x_1, x_2, x_3,\ldots , x_n\}$, and $p(x_i)$ denotes $P(X = x_i)$, then the expected value of $X$ is defined by:

\[
E(X) = \sum x_i \times p(x_i)
\]

where the elements are summed over all values of the random variable $X$.

\end{itemize}

\textbf{Remark: } Expected values for continuous random variables are defined, but are not part of this module.
}
%--------------------------------------------------------%
\frame{
\frametitle{Expected Value for Discrete Random Variables}
When a die is thrown, each of the possible sides 1, 2, 3, 4, 5, 6 (i.e. the $x_i$ 's) has a
probability of 1/6 (the $p(x_i)$ s) of showing.
\\ \bigskip
The expected value of the face showing is therefore:

\[E(X) = (1 \times 1/6) + (2 \times 1/6) + (3 \times 1/6) + (4 \times 1/6) + (5 \times 1/6) + (6 \times 1/6)\]

\[E(X) = 21/6 = 3.5 \]
\bigskip
Notice that, in this case, $E(X)$ is $3.5$, which is not a possible value of X.


}
%--------------------------------------------------------%
\frame{
\frametitle{Expected Value for Discrete Random Variables}

Suppose we are playing a game where the points scored in each round are the square of the value shown by the die?
What is expected value of the score for each round.
\\ \bigskip
As we have defined a random variable $X$ to represent the number shown by each roll, we will define another $X^2$ to represent the points accrued by each roll.
\\ \bigskip
The expected value can be computed as follows:
\[
E(X^2) = \sum (x_i^2) \times p(x_i)
\]
}
%--------------------------------------------------------%
\frame{
\frametitle{Expected Value for Discrete Random Variables}
The expected value of the points is therefore:

\[E(X^2) = (1^2 \times 1/6) + (2^2 \times 1/6) + (3^2 \times 1/6) + (4^2 \times 1/6) + (5^2 \times 1/6) + (6^2 \times 1/6)\]

\[E(X^2) = 91/6 = 15.16 \]


}






%--------------------------------------------------------%
\frame{
\frametitle{Variance of a Discrete Random Variable}
We may require to know the variance of the random variable.

The variance of the random variable X, denoted $V(X)$, is defined to be:
\[ V(X) = E(X^2) - E(X)^2 \]
where $E(X)$ is the expected value of the random variable X.
}


%--------------------------------------------------------%
\frame{
\frametitle{Variance of a Discrete Random Variable}
Compute the variance for outcomes of the throws of a die, using
\[ V(X) = E(X^2) - E(X)^2 \].

We know that \begin{itemize} \item $E(X) = 21/6$ , so  $E(X)^2 = 441/36$. \\

\item $E(X^2)  = 91/6  = 546/36$

\item Therefore $V(X) = 546/36 - 441/36 = 105/36 = 2.91$
\end{itemize}

}
\end{document}



Notes

\begin{itemize}
\item the larger the variance, the further that individual values of the random variable (observations) tend to be from the mean, on average;
\item the smaller the variance, the closer that individual values of the random variable (observations) tend to be to the mean, on average;
\item the variance and standard deviation of a random variable are always non-negative (i.e. almost always positive, but theoretically you can get a result of zero).
\end{itemize}


%--------------------------------------------------%
\frame{
\frametitle{Frame Title}
\Large
}


\frame{
\frametitle{Frame Title}
\begin{description}[Second Item]
\item[First Item] Description of first item
\item[Second Item] Description of second item
\item[Third Item] Description of third item
\item[Forth Item] Description of forth item
\end{description}
}
%--------------------------------------------------%
\frame{
\frametitle{Frame Title}


}
%--------------------------------------------------%












%--------------------------------------------------------%
\end{document}
%--------------------------------------------------%
\frame{
\frametitle{Frame Title}



}

}
%--------------------------------------------------%
\frame{
\frametitle{Frame Title}


}
%--------------------------------------------------------------------------------------------------%
X=c(900,925,950,975,1025,1050,1075,1100)
Y=c(900,905,910,920,1080,1090,1095,1100)
Z=c(900,985,990,995,1005,1010,1015,1100)


Z.y = rep(3,8)
Y.y = rep(4,8)
X.y = rep(5,8)

plot(Z,Z.y,pch=16,col="red",ylim=c(2.5,5.5),main=c("Variance"),font.lab=2,ylab=" ", xlab="X: Green  Y: Blue  Z: Red" )

points(Y,Y.y,pch=16,col="blue" )
points(X,X.y,pch=16,col="green" )
points(c(1000,1000,1000),c(3,4,5),pch=18,cex=1.2)
lines(c(1000,1000),c(2.75,5.25),lty=3)

%--------------------------------------------------------------------------------------------------%
%--------------------------------------------------------%

\frame{
\frametitle{Expected Values}
A box contains two gold balls and three silver balls. You are allowed to choose
successively balls from the box at random. You win 1 dollar each time you
draw a gold ball and lose 1 dollar each time you draw a silver ball. After a
draw, the ball is not replaced. Show that, if you draw until you are ahead by
1 dollar or until there are no more gold balls, this is a favorable game.

}

%--------------------------------------------------------%

\frame{

\frametitle{ Frequency Table }
\begin{itemize}
\item A frequency table is a way of summarising a set of data. It is a record of how often each value (or set of values) of the variable in question occurs. \item  It may be enhanced by the addition of percentages that fall into each category.

\item A frequency table is used to summarise categorical, nominal, and ordinal data.

\item
It may also be used to summarise continuous data once the data set has been divided up into sensible groups.
\item
When we have more than one categorical variable in our data set, a frequency table is sometimes called a contingency table because the figures found in the rows are contingent upon (dependent upon) those found in the columns.
\end{itemize}
}

\frame{
Example
Suppose that in thirty shots at a target, a marksman makes the following scores:
5 2 2 3 4 4 3 2 0 3 0 3 2 1 5
1 3 1 5 5 2 4 0 0 4 5 4 4 5 5

The frequencies of the different scores can be summarised as:
Score  Frequency Relativa Frequency (%)
0 4 13%
1 3 10%
2 5 17%
3 5 17%
4 6 20%
5 7 23%



}
\end{document}

%--------------------------------------------------%
\frame{
\frametitle{Frame Title}
\Large

}

%--------------------------------------------------------%
\frame{
\frametitle{Measures of Dispersion}

\textbf{Range}\\
\begin{itemize}
\item The range is a very simple measure of dispersion.
\item It is simply the difference between the maximum and minimum values.
\end{itemize}

Consider the following data set
\[ X= \{3,5,6,7,8,9\}\]

Range =  Max - Min \\ i.e. 9-3 = 6

} 
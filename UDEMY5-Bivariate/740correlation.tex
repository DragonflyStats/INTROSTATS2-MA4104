\documentclass[a4]{beamer}
\usepackage{amssymb}
\usepackage{graphicx}
\usepackage{subfigure}
\usepackage{newlfont}
\usepackage{amsmath,amsthm,amsfonts}
\usepackage{beamerthemesplit}
\usepackage{pgf,pgfarrows,pgfnodes,pgfautomata,pgfheaps,pgfshade}
\usepackage{mathptmx}  % Font Family
\usepackage{helvet}   % Font Family
\usepackage{color}

\mode<presentation> {
 \usetheme{Default} % was
 \useinnertheme{rounded}
 \useoutertheme{infolines}
 \usefonttheme{serif}
 %\usecolortheme{wolverine}
% \usecolortheme{rose}
\usefonttheme{structurebold}
}

\setbeamercovered{dynamic}

\title[Stats-Lab.com]{\LARGE Introduction to Statistics and Probability \\ {\Large Probability : Contingency Tables}}
\author[Kevin O'Brien]{Kevin O'Brien}
\date{Spring 2014}


\renewcommand{\arraystretch}{1.5}

\begin{document}


\begin{frame}
\titlepage
\end{frame}
%--------------------------------------------- %
% What is Correlation 
% Example of Positive / Neg
% Pearson Correlation
% Non-parametric procedures
%
%--------------------------------------------- %
\begin{frame}
\frametitle{Correlation}
\Large
\begin{itemize}
\item Correlation is a measure of the relation between two or more variables. 
%\item The measurement scales used should be at least interval scales, but other correlation coefficients are available to handle other types of data.
\item Correlation coefficients can range from -1.00 to +1.00. The value of -1.00 represents a perfect negative correlation while a value of +1.00 represents a perfect positive correlation. A value of 0.00 represents a lack of correlation.
\end{itemize}
 


\end{frame}
%--------------------------------------------- %
\begin{frame}
\frametitle{Correlation}
\Large
\begin{itemize}

\item 
The most widely-used type of correlation coefficient is Pearson r, also called linear or product- moment correlation.
The pearson correlation coefficient is a metric.
\end{itemize}
 


\end{frame}

\begin{frame}
\frametitle{Correlation}
\Large
\begin{itemize}
\item  Two variables that have no linear relationship have a correlation close to zero. 

\item Scatter plots are a useful way of determing the likely relationship between two variables. 

\item The Pearson correlation coefficient is most commonly used estimate for correlation. 

\item Other types of correlation are tbe \textit{\textbf{Spearman Rho}} and the \textit{\textbf{Kendal Tau}} correlation coefficients. 

%\item  These are not not part of this course,but it is important to know that they exist. 
\end{itemize}
\end{frame}

%-----------------------------------------------------%
%--------------------------------------------- %
\begin{frame}
\frametitle{Correlation}
\Large
\begin{itemize}
\item Correlation is a measure of strength of \textbf{Linear Relationship} between two variables.
\item The Pearson correlation coefficient (denoted $r$) is the most comonly used statistical estimate for correlation. 
\item Correlation estimates are defined to be between -1 and 1. It is not possible to have a correlation value outside this range of values
\[ -1 \leq r \leq 1\]
\item 
Additionally correlation estimates are not denominated in any units. (Contrast this to standard deviation, which is denominated in the same units as the mean).
\end{itemize}
\end{frame}

%--------------------------------------------- %
\begin{frame}
\frametitle{Correlation}
\Large
\begin{itemize}
\item A strong positive linear relationship describes a relationship between two variables whereby an increase in one variable will closely coincide with an increase in the other variable. 
\item Conversely a strong negative linear relationship describes a relationship whereby an increase in one variable closely coincides with a decrease in the other. 
\end{itemize}
\end{frame}
%-----------------------------------------------------%



\begin{frame}
\frametitle{Correlation}
\Large
\begin{itemize}
\item The Pearson correlation estimate, which us based on sample data, is denoted r (although related metrics use capital R).
\item This measure is used as an estimate for the Population correlation, denoted by the greek letter $\rho$ ( pronounced ``Rho"). 
The estimate is computed using summation identities.
% (See the formulae ).
 
%\item 
%Equivalently it can be computed using the  Sums of Squares Identities that are used to compute covariance and standard deviation <INSERT FORMULA> .
%Example
%determine the correlation estimate for the Spend V Impressions data.  
\end{itemize}
\end{frame}
%-------------------------------------------------------- %

\begin{frame}
\frametitle{Outliers}
Outliers can greatly influence the computed value of an estimate.
Correlation is closely related to Simple linear regression models, in that both are concerned with the linear relationship between variables. However Linear Regression has a different emphasis.
Simple Linear Regression describes one independent variable (IV) and the response of the dependent variable (DV). 
\end{frame}


\begin{frame}
\frametitle{Correlation and Causality }
Implicit is simple linear regression is the notion of causality. The dependent variable changes as the independent variable changes. The converse is not true.
 <some examples : hot temperature / ice cream example> .
Correlation is not concerned with causality at all, hence the often used expression "causation does not imply causality ".

\end{frame}
\end{document}


\begin{frame}
\frametitle{Regression analysis: Example}
In a study of a wholesaler’s distribution costs, undertaken with a view to controlling cost, the volume of goods handled and the overall costs were recorded for one month in each of ten depots in a distribution network. The results are presented in the following table. Perform a regression analysis of the cost (Y ) on the volume (X).

\end{frame}
%-------------------------------------------------------%
\begin{frame}
\frametitle{Regression analysis: Example}
\begin{center}

\begin{tabular}{|c|c|c|}\hline
      &  Volume (X)   &  Costs (Y) \\ \hline
1     &     48     &   20 \\
2     &    57      &   22 \\
3     &    49      &   19 \\
4     &    45      &   18 \\
5     &    50      &   20 \\
6     &    62      &   24 \\
7     &    58      &   21 \\
8     &    55      &   21 \\
9     &    38      &   15 \\
10    &    51      &  20 \\ \hline
\end{tabular}
\end{center}
%X=c(48,57,49,45,50,62,58,55,38,51)
%Y=c(20,22,19,18,20,24,21,21,15,20)

\end{frame} 




 





%-----------------------------------------------------------------------%

% # cor(X,Y) = 0.92
% X = c(19.5, 23.84, 15.34, 23.37, 13.82, 16.48, 16.15, 16.76, 17.49, 17.4, 25.42, 29.29)
% Y = c(24.41, 26.91, 24.33, 25.5, 22.84, 24.35, 23.59, 23.98, 24.65, 22.56, 26.78, 28.68)
%
% X=c(20.88, 11.72, 21.39, 15.97, 19.58, 17.2, 16.47, 20.04, 16.7, 22.23, 24.87, 23.94)
% Y=c(24.18, 27.28, 23.79, 24.84, 24.36, 24.75, 25.93, 24.76, 25.26, 22.97, 23.71, 22.75)

%-----------------------------------------------------------------------%
\end{document}

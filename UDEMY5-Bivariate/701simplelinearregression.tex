
\documentclass{beamer}

\usepackage{amsmath}
\usepackage{amssymb}

\begin{document}

\section{Simple Linear Regression}

\begin{frame}
\frametitle{Scatter plots}

The first part of the question will require the drawing of a scatter plot. 
When doing do, remember to label the axes, and to put in the relevant units. (I.e. Metres, Degrees, Hours etc)

The Explanatory variable is on the X-axis and the Response variable is on the Y Axis.

A Trend line will be useful in demonstrating what type of relationship exists between the response variable and the explanatory variable.
\end{frame}
%-------------------------------------------------------------------------------------------%
\begin{frame}
\begin{enumerate}
\item There are five possible plot types
\item Strong positive linear relationship 
\item Weak positive linear relationship
\item Strong negative linear relationship
\item Weak negative linear relationship
\item No Relationship
\end{enumerate}
In the strong case – the points of the graph correspond to the trend line quite closely, whereas in the weak case they don’t.
In the positive case the response values Y increase as the explanatory values X increases. In the negative case the response values Y decrease as the explanatory values X increases. 

\end{frame}
%-------------------------------------------------------------------------------------------%

\section{Correlation}
\begin{frame}
This requires a simple calculation based in values given and the relevant formula.

The formula for the Correlation estimate is as follows.
\end{frame}
%--------------------------------------------------------------------------------------------%
\begin{frame}
The calculated value should be between -1 and 1.

The following conclusions are drawn , depending on the Correlation estimate value:
\begin{itemize}
\item Greater than 0.9 		Very strong positive linear relationship 
\item Between 0.7 and 0.9		Strong positive linear relationship 
\item Between 0.2 and 0.7	 	Weak positive linear relationship
\item Between -0.2 and 0.2		No relationship
\item Between -0.7 and -0.2		Weak negative linear relationship
\item Between -0.9 and -0.7		Strong negative linear relationship
\item Less than -0.9			Very strong negative linear relationship
\end{itemize}
Your answer should concur with your interpretation of the scatterplot.
\end{frame}
%--------------------------------------------------------------------------------------------%
\section{Intercept Slope calculations}
\begin{frame}
We are asked to calculate the following
\begin{itemize} 
\item an estimate for the intercept value
\item an estimate for the slope value
\end{itemize}
(The chevron sign denotes that the value in question is an estimate.)
\end{frame}
%--------------------------------------------------------------------------------------------%
\section{Intercept Slope calculations}
\begin{frame}
We calculate the estimate for slope first.

To calculate the estimate for the intercept, we first must determine the values for the means of X and Y (i.e  and ). We are given the values of the summations of X and Y (i.e. and ), which we divide by the number of XY pairs (‘n’). 
\end{frame}
%--------------------------------------------------------------------------------------------%
\begin{frame}
We then construct the regression model equation , which estimate a value for Y for a  given X value. It takes the form:	

The second part of the question will give us a particular X value and ask us to calculate a corresponding estimate for Y.

\end{frame}
\end{document}
Part I Interpreting scatterplot


This scatterplot suggests a weak positive linear relationship between the daily high temperatures and the year.

Σ xi = 465;         Σ yi = 387.23
SX,Y = 60.065;       SX,X = 2247.5;           SY,Y = 8.0033.

Part II 	Correlation


This Correlation value indicates weak positive linear relationship between temperatures and year.
%-------------------------------------------------------------------------------------------%
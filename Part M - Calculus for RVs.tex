\documentclass[a4]{beamer}
\usepackage{amssymb}
\usepackage{graphicx}
\usepackage{subfigure}
\usepackage{newlfont}
\usepackage{amsmath,amsthm,amsfonts}
\usepackage{beamerthemesplit}
\usepackage{pgf,pgfarrows,pgfnodes,pgfautomata,pgfheaps,pgfshade}
\usepackage{mathptmx}  % Font Family
\usepackage{helvet}   % Font Family
\usepackage{color}

\mode<presentation> {
 \usetheme{Default} % was
 \useinnertheme{rounded}
 \useoutertheme{infolines}
 \usefonttheme{serif}
 %\usecolortheme{wolverine}
% \usecolortheme{rose}
\usefonttheme{structurebold}
}

\setbeamercovered{dynamic}

\title[Stats-Lab.com]{\LARGE Introduction to Statistics and Probability \\ {\Large Calculus For Random Variables}}
\author[Kevin O'Brien]{Kevin O'Brien}
\date{Spring 2014}


\renewcommand{\arraystretch}{1.5}

\begin{document}


\begin{frame}
\titlepage
\end{frame}
%------------------------------------------- %
\begin{frame}
\frametitle{Calculus For Random Variables}
\Large

The random variables $X$ has the probability density function $f(X)$ given by:
\[ f(x) = kx^2(1-x), \phantom{space} 0 \leq x \leq 1 \]

\begin{itemize}
\item[1.] Compute the value for $k$,
\item[2.] Compute the mean and variance for $X$,
\item[3.] Determine the cumulative distribution function $F(x)$,
\item[4.] Compute the probability that X lies within one standard deviation of its mean.
\end{itemize}
\end{frame}
%-------------------------------------- %
\begin{frame}
\frametitle{Calculus For Random Variables}
\Large
\vspace{-2.8cm}
\textbf{Part 1}\\
The definite integral of $f(x)$ between 0 and 1 must equal 1.

\[ \int^1_0 f(x)\;dx = \int^1_0 kx^2(1-x)\;dx = \int^1_0 kx^2-kx^3\;dx  \]
\end{frame}

%-------------------------------------- %
\begin{frame}
\frametitle{Calculus For Random Variables}
\Large
\vspace{-0.5cm}
\textbf{Part 1}\\
The definite integral of $f(x)$ between 0 and 1 must equal 1.

\[ \int^1_0 f(x)\;dx = \int^1_0 kx^2(1-x)\;dx = \int^1_0 kx^2-kx^3\;dx  \]

\[ = \left[\frac{kx^3}{3} \right]^1_0  - \left[\frac{kx^4}{4} \right]^1_0
\phantom{sce} = \frac{k}{3} - \frac{k}{4} \phantom{sce} = \frac{k}{12} \phantom{sce}(= 1)
\] 

\[ \boldsymbol{k = 12}\]
\end{frame}

%-------------------------------------- %
\begin{frame}
\frametitle{Calculus For Random Variables}
\Large
\vspace{-0.5cm}
\textbf{Part 2 :  Compute the Mean and Variance}\\
\[ \mathrm{E}(x) = \int^1_0 x\; f(x)\;dx  \]

\[ \mathrm{Var}(x) = \mathrm{E}(x^2)  - [\mathrm{E}(x)]^2   \]

\[ \mathrm{E}(x^2) = \int^1_0 x^2\; f(x)\;dx \]
\end{frame}
%-------------------------------------- %
\begin{frame}
\frametitle{Calculus For Random Variables}
\Large
\vspace{-2.6cm}
\textbf{Part 2 :  Compute the Mean and Variance}\\
\[ \mathrm{E}(x) \; = \; \int^1_0 x\; f(x)\;dx \;   \]
\[ \mathrm{E}(x) \; = \int^1_0 x(12x^2-12x^3)\;dx  \; = \int^1_0 12x^3-12x^4\;dx  \]

\end{frame}
%-------------------------------------- %
\begin{frame}
\frametitle{Calculus For Random Variables}
\Large
\vspace{-1.5cm}
\textbf{Part 2 :  Compute the Mean and Variance}\\
\[ \mathrm{E}(x) \; = \; \int^1_0 x\; f(x)\;dx \;   \]
\[ \mathrm{E}(x) \; = \int^1_0 x(12x^2-12x^3)\;dx  \; = \int^1_0 12x^3-12x^4\;dx  \]

\[ \mathrm{E}(x)  = \left[\frac{12x^4}{4}-\frac{12x^5}{5} \right]^1_0 = \frac{6}{5} \]
\end{frame}
%-------------------------------------- %
\begin{frame}
\frametitle{Calculus For Random Variables}
\Large
\vspace{-2.6cm}
\textbf{Part 2 :  Compute the Mean and Variance}\\
\[ \mathrm{E}(x^2) \; = \; \int^1_0 x^2\; f(x)\;dx \;   \]
\[ \mathrm{E}(x^2) \; = \int^1_0 x^2(12x^2-12x^3)\;dx  \; = \int^1_0 12x^4-12x^5\;dx  \]

\end{frame}
%-------------------------------------- %
\begin{frame}
\frametitle{Calculus For Random Variables}
\Large
\vspace{-1.5cm}
\textbf{Part 2 :  Compute the Mean and Variance}\\
\[ \mathrm{E}(x^2) \; = \; \int^1_0 x^2\; f(x)\;dx \;   \]
\[ \mathrm{E}(x^2) \; = \int^1_0 x^2(12x^2-12x^3)\;dx  \; = \int^1_0 12x^4-12x^5\;dx  \]

\[ \mathrm{E}(x)  = \left[\frac{12x^5}{5}-\frac{12x^6}{6} \right]^1_0 = \frac{2}{5} \]
\end{frame}

\begin{frame}
\Large
\vspace{-1.5cm}
\textbf{Part 2 :  Compute the Mean and Variance}\\

\[ \mathrm{Var}(x) = \mathrm{E}(x^2)  - [\mathrm{E}(x)]^2   \]

\[ \mathrm{Var}(x) = \frac{2}{5}  - \left(\frac{3}{5}\right)^2  \]
\end{frame}
%--------------------------------- %
\begin{frame}
\frametitle{Calculus For Random Variables}
\Large
\vspace{-2.90cm}
\textbf{Part 3 : Determine the cumulative distribution function $F(x)$.}

\[F(x) = \int_o^x f(u) \;du  =  \int_o^x 12u^2-12u^3 \;du\]

\end{frame}
%--------------------------------- %
\begin{frame}
\frametitle{Calculus For Random Variables}
\Large
\vspace{-0.8cm}
\textbf{Part 3 : Determine the cumulative distribution function $F(x)$.}

\[F(x) = \int_o^x f(u) \;du  =  \int_o^x 12u^2-12u^3 \;du\]

\[ F(x) = \left[\frac{12u^3}{3} - \frac{12u^4}{4} \right]^x_0 \]

\[ F(x) = \left[\frac{12x^3}{3} - \frac{12x^4}{4} \right] = \boldsymbol{ 4x^3 - 3x^4 } \]

\end{frame}
%--------------------------------- %
\begin{frame}
\frametitle{Calculus For Random Variables}
\Large
\vspace{-0.8cm}
\textbf{Part 4 :}
\[P(1 \leq X \leq 2)   \int_1^2 12x^2-12x^3 \;dx\]

\end{frame}
%-------------------------------- %
\end{document}
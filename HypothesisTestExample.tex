MA4413 Statistics for Computing
 
Lecture Notes 11 : Hypothesis Testing

MA4302 Revision Class 2
Thursday 13:00 - 14:00 Lecture -   FB028
 
Hypothesis Testing 
Important Considerations
 
1) Sample size n
 
    Is it large or small?        "small" less than or equal to thirty.
 
2) Significance level 
 
    95% confidence means = 0.05
    99% confidence means = 0.01
 
3) Number of tails in procedure
    
    Procedures are either one tailed or two tailed.  k=1 or 2
    Confidennce intervals are always two tailed.
 
4) Standard Error Formula
 
    Back of exam paper
 
5) Tables to use
    
    Table 3 - Normal "Z" Distribution
    Table 7 - student's "t" Distribution
    Table 8 - Chi Square Distribution
 
6) Degrees of freedom
    
    notation is sometimes 
    
    Large samples    df = 
    Small samples    df = n-1
    
    Chi-Square        df = (r-1)x(c-1)
                 r  = number of rows
                 c =  number of columns
              
7) Hypothesis tests usually have the following format.
 
Step 1 : Formally state the null and alternative hypotheses
Step 2 : Determine the test statistic
Step 3 : Determine the critical value
Step 4 : Decision Rule
 
8) Hypothesis testing using p-values
 
If asked to use p-value, we have a slightly diffferent approach.
The first two steps are the same as in note 7.
  

Step A : Formally state the null and alternative hypotheses

Step B : Determine the test statistic
Step C : Determine the p-value
Step D : Decision Rule for p-values.
 
9)  What is a p-value?
 
P-Value = P(Z|TS|)   TS: Test Statistic
 
P-value  is found from Murdoch Barnes Tables 3
 
For example, if the test statistic is 1.96, then the p-value is 0.025
 
10) How to interpret the p-value
 
    (see previous notes)
 
 is the significance value
 
k is the number of tails

If the p-value is less than k, we reject the null hypothesis.
If the p-value is greater than k, we reject the null hypothesis.
 
11) The general structure of a test statistic
 
TS =Observed Value-Null ValueStd. Error 
 
12) General Structure of a Confidence Interval

 

 
Quantiles are compute the same way as critical values.
 
 
Question 2 part b
 
A study of 1000 randomly chosen adults indicated that 450 had been to the cinema at least once in the previous year.
 
A cinema wants to test the hypothesis that 50% of all Irish adults have been to the cinema in the last year.
 
Calculate the p-value for such a test and draw the appropriate conclusion.
 
Discussion: Based on this sample, we estimate the proportion to be 0.45  (i.e. 45%)
 
p= 0.45
 
Step A : Formally state the null and alternative hypotheses
 
p : true proportion of Irish adults who have been to the cinema in the last year.
 
Null Hypothesis               Ho:p = 0.50        True proportion is 50%
 
Alternative Hypothesis      Ha:p 0.50        True proportion is not 50%.
 
 
N.B. This is a two-tailed procedure.
 
 
Step B : Compute the test statistic.
 
Remember the general structure of a test statistic
 
TS =Observed Value-Null ValueStd. Error 
 
 
 
From the formulae
 
We have to compute the standard error for a proportion. 
 
( From formulae at back of exam paper)
 
 S.E.(p) =p(1-p)n=0.450.551000= 0.0157
 
 
 
 
 
 
Step 3: Calculate p-value
 
P-value is found from Murdoch Barnes Tables 3 ( Normal distribution)
 
Absolute value  |-3.18| =3.18
 
P-Value = 
 
 
P(Z3.18) = 0.00074
 
 
Step 4: Interpret the p-value to make a decision.
 
The significance level is 5%.  The procedure is a two tailed test.
 
 
[ Black Board ]
 
 
Question 4 part a - Paired T test
The weight of 6 individuals (in kgs) was observed before and after a diet regime (diet given below)
Compute the mean difference and standard deviation of the differences.
 
Remark : Sample size n= 6
 
Person
A
B
C
D
E
F
Weight Before
89
79
106
92
88
98
Weight After
87
76
105
92
84
96
Difference
2
3
1
0
4
2
 
 
Before we start, we need to compute the average difference and the standard deviations of the differences. 
 
 

Sample Mean:           x=xin      
 
 
What is the mean difference d
 
  
d=din=2+3+1+0+4+26= 2
  
 
 
Standard Deviation:           Sx=(xi-x)2n-1      
 

Person
A
B
C
D
E
F
di
2
3
1
0
4
2
di-d
0
1
-1
-2
2
0
(di-d)2
0
1
1
4
4
0
 
 Sx=(di-d)26-1=( 0 + 1 +1 +4 +4 +0)6-1
 
 Sd=105=2 
 
Now we are ready to perform our hypothesis test.
 

Step1 : Formally state the null and alternative hypotheses
 
d : true difference in weight before and after the diet regime 
 
Null Hypothesis               Ho:d = 0        True difference is zero
 
Alternative Hypothesis      Ha:d 0       True difference is not zero 
 
Remark: This is a two tailed test
 
 
Step 2 : Compute the test statistic.
 

Remember the general structure of a test statistic
 
TS =Observed Value-Null ValueStd. Error 
 
 
 
From the formulae
 
We have to compute the standard error for a sample mean. 
 
( From formulae at back of exam paper)
 
 S.E.(d) =Sdn=26= 0.5773 
 
 
 
 
 
Step 3 :  Determine the Critical Value
 
Small sample (group is less than 30).
(Population variance is unknown.)
 
Use t distribution with n-1 degrees of Freedom.
 
We use Murdoch Barnes Table 7 (Student T distribution)
 
Significance levels is 1%.  This is a two tailed test.
 
[Blackboard]
 
 
Critical Value is 2.571
 
 
Step 4 : Decision
 

[Blackboard]
 
Is the test statistic in the acceptance region or the rejection region?
 
It is in the rejection region. We reject the null hypothesis. The diet does work.
 
  
 
Question 5 Part b : Confidence interval for the difference in means of two samples.
 
b) The heights of 100 Americans and 50 Spaniards were observed.
 The mean and standard deviation of the heights of a sample according to nationality are given below
 
 
Number
Mean
Std. Dev.
American
100
172
13
Spanish
50
167
12
 
i)        Calculate a 99% confidence interval for the difference between the mean height of all americans
and the mean height of all spaniards.(7 males)
 
General Structure of a Confidence Interval
 


 
            
Observed difference
 
let X denote the heights of americans    X= 172
let Y denote the heights of spaniards     Y= 167
 
The difference in the mean of weights X-Y= 5
 
Quantile
 
Large sample (both groups are greater than 30).
 
Population variance is unknown.
 
Use t distribution with  degrees of Freedom.
 
 
Confidence level is 99%. Therefore significance levels is 1%.
Remark : Confidence intervals are always two tailed procedures.
 
 
Column = k =0.012= 0.005

 
Murdoch Barnes table 7

 
	Row: df =  
	Column = 0.005
 
    Quantile =  2.576 
 
 
Stardard Error
  
 

 
 

  
Confidence Interval is therefore
 
99% CI = 5(2.5762.137) 
 
 
Part (ii)
Based on this confidence interval, test the hypothesis that on average male students are 6kgs heavier than female students.
State your hypotheses clearly. What is the significance level of this test?   (3 marks)
 
 
Null :                 True difference between means is zero
 
Alternative:         True difference between means is not zero
 
Or alternatively
 
Null :              H0:X-Y= 0   True difference between means is zero
 
Alternative:         True difference between means is not zero
 
 
 
Since the null value, 0 is inside the confidence interval  .
 
This means that the true difference of means could be zero.
 
We do not reject the null hypothesis at a significance level of 1%.
 
 
Question 4b  
Chi-square test for independence and Probability [10 Marks]

This question consists of two parts
    
    Basic Probability
    Chi Square Hypothesis test
 
Lets look at the table first
 
Important points from the table
 
1) There are 120 students , 60 male and 60 female
 
2) There are three programmes
 
	1) Maths - with room for 30 students
	2) Equine studies - with room for 30 students
	3) Chemistry - with room for 60 students.

Let us assume that there a male and female student are equally likely to enter each program. (This is the null hypothesis).

We would expect the each program to have the following compositions.

	1) Maths - would have 15 male and 15 female students (each year on average)
	2) Equine studies - would have 15 male and 15 female students (each year on average)
	3) Chemistry -would have 30 male and 30 female students (each year on average)
 

Probability
Observed Values


Maths
Eq. Studies
Chemistry
Sum
Male
20
10
30
60
Female
10
20
30
60
Sum
30
30
60
120

What is the probability that a randomly chosen person from the sample is an equine student?
 
There are 120 students altogether, 30 of those are equine science students.

P(Eq) =30120= 0.25

Given that a student is female, what is the probability that that she is an equine science student.

P(Eq |F) =P(EqandF)P(F)=20/12060/120= 0.33
 
 
Chi Square Test 
Replace the question marks with the expected number of observations
in a cell under the nulll hypothesis that the choice of course doesnt depend on gender.

Expected Values ( under null hypothesis)


Maths
Eq. Studies
Chemistry
Sum
Male
15
15
30
60
Female
15
15
30
60
Sum
30
30
60
120


 
Hypothesis Test
 
Step 1:  Formally write out null and alternative hypothesis
 
 
    Gender and Choice of College coure are independent of each other.
 
    Gender and Choice of College coure are not independent of each other.
 
 
Step 2: Test Statistic
 
We use a special Test statistic for this test.
 
For each of the six subgroups, perform the following calculation.
 
        (nij-eij)2eij
 
    nij : observed number for subgroup
    eij : expected number for subgroup
 
Add up all these terms.
 
 T=(20-15)215+(10-15)215+(10-15)215+(20-15)215+(30-30)230+(30-30)230
 
 
 T= 1.667 + 1.667 +1.667 +1.667 +0 +0 
 
 
 T = 6.6667 
 
 
Step 3: Test Statistic
 
Murdoch Barnes Table 8
 
Significance level is 5%
Number of tails is 2
degrees of freedom = (2-1)(3-1) = 1 x 2 = 2
 
 





0.025


1






2


7.378


3














 
 
Critical value is 7.378
 
Step 4 Decision Rule
 
Is the Test statistic greater than the Critical value
 
is 6.6667 > 7.378
 
No! We fail to reject the null hypothesis.
 
We do not have enough evidence to say that there is a relationship between gender and college courses.
 
 
 
 
 
 
Question 5b : Simple Linear Regression

 
part i
Use the above table, give the equation of the regression line expressing pressure as a function of volume
 
Intercept estimate :	 70.048
Slope estimate :         -7.402
 
Regression line  : Pres.= 70.048 - 7.402Vol. 

Pres.  is the pressure estimated from the model for a certain volume. 
 
 

 
part ii
 
According to the model, what is the average effect on pressure resulting from an increase in volume of one cubic metre.
 
 
For every increase of one cubic metre in volume, the pressure is estimate to decrease by -7.402  bars.

 
Part iii
 
Using this model, estimate the air pressure when the slope is 5 metres cubed.
 
		33.038 bars.
 
Part iv
 
The observation if the air pressure at a volume of 5 cubic metres was 19.87 bars.
 
Calculate the residual from the regression model corresponding to this observation
 
Residual = Estimate - Observered value = 30.038 - 19.87
 
Residual = 13.168 
 
Part v: Hypothesis test on the slope.
 
 
If the true value of the slope is zero, then there is no relationship between volume and pressure.
 
(i.e. Null hypothesis means that Pressure doesnt depend on volume).
 
 
The null and alternative hypotheses are as follows
 
Ho:1= 0  	 The true value of the slope is zero
 
 
		The true value of the slope is not zero
 
 
Let us assume a significance value of 5% (0.05). Also, this hypothesis is a two tailed test. 
 
If the p-value (referred to by SPSS as the "sig.") is greater than 0.025, we fail to reject the null hypothesis. We can not rightly contradict the statement that the true slope is zero.
 
If the p-value is less than 0.025, we reject the null hypothesis.
 
We reject the null hypothesis, accepting the alternative hypothesis that the true slope is not zero.
 
(If the true slope is not zero, there two variable are not independent of each other).
 
Here the p-value is very very low ("sig"  is 0.000). It is less than our threshold of 0.025. Therefore we reject the null hypothesis.
 
Part vi
 
 
Briefly explain why the  use of linear regression to describe pressure as a funciton of voumen is inappropriate.
 
Looking at the scatterplot, we can clearly see that the relationship between pressure and volume is curved (i.e. not linear). 
 
Therefore linear regression is not an appropriate analysis for this data.
 
  
 

%========================================================================% 
Sample mean
Sample proportion
Two Sample test
Two Sample test of proportions. 

%========================================================================%
D Hypothesis Testing - Example
The catering manager in a hotel suspects that the weight of loaves of bread delivered
daily by a bakery is consistently below the nominal weight of 800g. To test this,
10 loaves chosen at random from a day’s deliveries are weighed. The mean and
standard deviation of the ten weights are 792g and 25g, respectively.
 
1) Carry out a formal significance test.
2) List the steps involved in this test 
3) Calculate a 95% confidence interval for the average weight of loaves produced
4) Comment on the correspondence between the interval, as calculated, and the
    result of the test.
%========================================================================%

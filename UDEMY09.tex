
Chapter 9
\chapter{CURVE FITTING, REGRESSION AND CORRELATION}
IN THIS CHAPTER:
l/ Cun/e Fitting
l/ Regression
0/ The Method of Least Squares
0/ The Least-Squares Line
l/ The Least-Squares Regression Line
in Terms of Sample Variances
and Covariance
0/ Standard Error of Estimate
l/ The Linear Correlation Coefficient
l/ Generalized Correlation Coefficient
0/ Correlation and Dependence

%%9 9
%%Copyright 2001 by the McGraW-Hill Companies, Inc. Click Hem for Terms of Use


%%-100 PROBABILITY AND STATISTICS
\section{Curve Fitting}
Very often in practice a relationship is found
to exist between two (or more) variables, and one wishes to express this relationship
in mathematical form by determining an equation connecting the variables.

A first step is the collection of data showing corresponding values of the variables. For example, suppose x and y denote, respectively, the height and weight of an adult male. Then a sample of n individuals
would reveal the heights x , x ,  x and the corresponding weights y
l Z 11 1‘
y2,..., yn.
A next step is to plot the points (xpyl), (x2,yZ),..., (xwyn) on a rec-
tangular coordinate system. The resulting set of points is sometimes
called a scatter diagram.
%================================================================================%
From the scatter diagram it is often possible to visualize a smooth
curve approximating the data. Such a curve is called an approximating
curve. In Figure 9-1, for example, the data appear to be approximated
well by :1 straight line, and We say that a linear relationship exists
between the variables. In Figure 9-2, however, although a relationship
exists between the variables, it is not a linear relationship and so we call
it a nonlinear relationship. In Figure 9-3 there appears to be no rela-
tionship between the variables.
.V
X
Figure 9-1



CHAPTER 9: Cun/e Fitting, Regression, and Correlation 101
,.
.." '°
I
Figure 9-2
Y
- -',
~ _ ,*-
: - ,
X
Figure 9-3
The general problem of finding equations approximating curves
that fit given sets of data is called carve fitting, In practice the type of
equation is often suggested from the scatter diagram. For Figure 9-1 we
could use a straight line:
\[y = a + bx\]
while for Figure 9-2 We could try a parabola or quadratic curve:
\[y = a + bx + cx^2\]
For the purposes of this book, we will only concem ourselves with
the data sets exhibiting a linear relationship.

%==========================================================================
%%- 102 PROBABILITY AND STATISTICS
Sometimes it helps to plot scatter diagrams in terms of transformed
variables. For example, if log y V84 log x leads to a straight line, we
would try log y = a + bx as an equation for the approximating curve.
\section{Regression}
One of the main purposes of curve fitting is to esti~ K
mate one of the variables (the dependent variable) from the other (the independent variable). The ‘
. . . 6 . Q1
process of estimation is often referred to as a 5'“
regression. If y is to be estimated from x by means
of some equation, we call the equation a regression
equation 0f y on x and the corresponding curve a
regression curve of y on x. Since we are only con-
sidering the linear case, we can call this the regres-
sion line ofy on x.

\section{The Method of Least Squares}
Generally, more than one curve of a given type will appear to fit a set of
data. To avoid individual judgment in constructing lines, parzibolas, or
other approximating curves, it is necessary to agree on a definition of a
“best-fitting line," “best~fitting parabola," etc.

To motivate a possible definition, consider Figure 9-4 in which the
data points are (xl,y1),...,(x”,y"). For a given value ofx, say xi, there will
be a difference between the value yl and the corresponding value as
determined by the curve C. We denote the difference by dl, which is
sometimes referred to as a deviation error, or residual and may be pos-
itive, negative, or zero, Similarly, corresponding values x2,  m, We
obtain the deviations dz ,..,, dn.



CHAPTER 9: Cun/e Fitting, Regression, and Correlation 103
Y
(X... y.)
. d. C
(Xi - Yi) .
dl J1 . 0
(~‘:- 1'1)
x
Figure 9-4
A measure of the fit of the curve C to the set of data is provided by
the quantity dlz +11% +~-11,? . If this is small, the fit is good; if it is large,
the fit is bad. We therefore make the following definition.
%================================================================================%
Definition Of all curves in a given family of curves approximating a set of n data points, a curve having the property
that
dig +d22 +~~d3= a minimum

is called a best-fitting curve in the family.
A curve having this property is said to fit the data in the least-
squares sense and is called a least-squares regression curve, or simply
a least-squares curve. A line having this property is called a least-
squares line; a parabola that has this property is called a least-squares
parabola; etc.
It is customary to employ the new definition when x is the inde-
pendent variable and y is the dependent variable. If x is the dependent
variable, the definition is modified by considering horizontal deviations
instead of vertical deviations, which amounts to interchanging the x and 
y axes. 

%===============================================================================%
%%-104 PROBABILITY AND STATISTICSThese two definitions lead in general to two different least-
squares curves. Unless otherwise specified we shall consider y the
dependent and x the independent variable

You Need to Know I
It is possible to define another least-squares curve by
considering perpendicular distances from the data points
to the curve instead of either vertical or horizontal dis-
tances. However, this is not used very often.
\subsection{The Least-Squares Line}
By using the above definition, we can show that the least~squares line
approximating the set of points (xl,yl),...,(x",_y") has the equation
y=a+bx (1)
where the constants a and b are determined by solving simultaneously
the equations
2y=an+b2x
2xy=a2x+bZx2 (2)
Which are called the rmrmal equations for the least~squares line. Note
rt n
that We have for brevity used Z)‘ ,  instead of Zyj , Zhjyjo
/=1 ,'=1
%=============================================================================%
%%-CHAPTER 9: Cun/e Fitting, Regression, and Correlation 105

The normal equation (2) is easily remembered by observing that the
first equation can be obtained formally by summing on both sides of (I),
While the second equation is obtained formally by first multiplying both
sides of (1) by x and then summing. Of course, this is not a derivation
of the normal equations but only a means for remembering them.
The values of a and b obtained from (Z) are given by
g=(Zy)(2x*)—( mi ,,;2X»<2X>i2y>@
»2x1~< t ~2»*—<2X>2
The result for I: can also be written as
MM
><>-<
b=a2(""7)(”_y) <4)
Ea -if
Here, as usual, a bar indicates mean, e.g, J? = (Ex)/n . Division
of both sides of the first normal equation in (2) by n yields
y=a+1»? (5)
If desired, we can first find b from (3) or (4) and then use (5) to find
a = Y — bi. This is equivalent to writing the least-squares line as
.v—>*=h<x-r> or y_y=g2g;x*jg)§y)<X_z) <6)
The result (6) shows that the constant h, which is the slope of the
line (I), is the fundamental constant in determining the line. From (6) it
is also seen that the least~squares line passes through the point (Ii),
which is called the centroid or center of gravity of the data.

%=========================================================================%
%%-106 PROBABILITY AND STATISTICS
The slope b of the regression line is independent of the origin of the
coordinates. This means that if we make the transformation (often
called a translation of axes) given by
x=x'+h y=y'+k (7)
where h and k are any constants, then b is also given by
b I "E/Y’-(Zr’)(§;r’) I Zo — my -2?) (8)
112')/2 —(2x’) 2(x'—X')
Where x, y have simply been replaced by x’, y’ [for this reason we say
that b is invariant under the transformation (7)]. It should be noted,
however, that a, which determines the intercept on the x axis, does
depend on the origin (and so is not invariant).
In the particular case where It = Z, k = y , (8) simplifies to
x'y’
1; = Ln (9)
Ex
The results (8) and (9) are often useful in simplifying the labor
involved in obtaining the leasbsquares line,
The above remarks also hold for the regression line of x on y. The
results are formally obtained by simply interchanging X and y. For
example, the least-squares regression line ofx on y is
x_i=Z(X*?)(y—i) J
e2(y_y)2 (y ,) <10)
It should be noted that in general (10) is not the same as (6).


%%-CHAPTER 9: Cun/e Fitting, Regression, and Correlation 107
Remember
You should try to find the equation for the regression line if and only if your data set has a linear relationship.

Example 9.1. Table 9-1 shows the respective heights x and y of a sample of 12 fathers and their oldest sons. Find the least-squares regression line of y on x.

Table 9-l
Hcighhofhlherlinches) 65 63 67 64 68 62
HcightyofSon(inches) 68 66 68 65 69 66
Heightx0fFather(inches) 70 66 68 67 69 71
Hci5hlyofSon(inchcs) 68 65 71 67 68 70

The regression line ofy on x is given by y = ax + b are obtained by solving the normal equations
Z)i=an+b2x and 2/cy=a2‘x+b2;x2

The sums are computed as follows:



108 PROBABILITY AND STATISTICS
Tabl
‘<
1
x
‘Y
.v’
Q 0~<!\o\0\a~a~0\o\
4225
3969
4489
4096
4624
3844
4900
4356
4624
4489
4761
5041
4420
4158
4556
4l60
4692
4092
4760
4290
4828
4489
4692
4970
4624
4356
4624
4225
4761
4356
4624
4225
5041
4489
4624
4900
Z;-soo {Fan §1'=s3.41s Zxy=54,l07 Z,/==s4s49
l2a + 8006
00476, s
800a + 53,4181:
For Wh1Ch We find a = 35 82 and b =
1s the equatlon for the regression line,
v-
h
Usmg these sums, the normal equations become
The sample vanances and covariance of x and y are given by
107
0 that y = 35.82 + 0.476):
The Least-Squares Regression Line in Terms
of Sample Var|ances and Covariance
X-x -*2 -- --
S I s :(y y)‘ LIEU X)(y y) (U)

%%--CHAPTER 9: Cun/e Fitting, Regression, and Correlation 109

In terms of these, the least-squares regression lines of y on x and 1
on _v can be written, respectively, as
9, H
».~:_
» %
<1»;
y—§= (x—JY) and x—)?: (y—_y_') (12)
if we formally define the sample correlation cueficient by
sh.
r = —' (13)
sxs‘,
then (12) can be written
Ly: (X_i) and §i=,@ (14)
sy sx sx s‘,
In view of the fact that (x — Y)/si and (y — )7")/sy are standardized
sample values or standard scores, the results in (14) provide a simple
Way of remembering the regression lines. It is clear that the two lines in
(I4) are different unless r = il, in which case all sample points lie in a
line and there is perfect linear correlation and regression.
It is also of interest to note that if the two regression lines (14) are
written as y = ax + b, 1 = c + dy, respectively, then
lid = r2 (I5)
Up to now we have not considered the precise significance of the
correlation coefficient but have only defined it formally in terms of the
variances a.nd covariance

%=====================================================================%

%%- 110 PROBABILITY AND STATISTICS
\section{Standard Error of Estimate}
If We let ym denote the estimated value of y for a given value of x, as
obtained from the regression curve of y on x, then a measure of the scat-
ter about the regression curve is supplied by the quantity
Sm :  ([6)
n
which is called the standard error of estimate y on x. Since
2(y—ym)2 = zdz , as used in the definition we saw earlier, we see
that out of all possible regression curves the least-squares curve has the
smallest standard error of estimate.
In the case of a regression line ym = a + bx, with a and b given by
(2), we have
x3Y=2y2_a€y—b2xy (17)
Of
Z<y—,»*>* —bZ<x—2><y—r>
S§—,\' = g <18)
We can also express six for the least~squares regression line in
terms of variance and correlation coefficient as
Sf" = .g§(1- r2) (19)
from which it incidentally follows as a corollary that r2 5 l, i,e., —l 5 r
£1.

%%-CHAPTER 9: Cun/e Fitting, Regression, and Correlation 1 1 1

The standard error of estimate has properties analogous to those of
standard deviation. For example, if we construct pairs of lines parallel
to the regression line of y on x at respective vertical distances sm, and
Zsw, and 3s“ from it, We should find if n is large enough that there
would be included between these pairs of lines about 68\%, 95\%, and
99.7\% of the sample points, respectively.

Just as there is an unbiased estimate of population variance given
by f2 = nsz/(n — l), so there is an unbiased estimate of the square of the
standard error of estimate. This is given by fix = nffu /(n — 2). For this
reason some statisticians prefer to give (16) with n — 2 instead of n in
the denominator.
The above remarks are easily modified for the regression line ofx
on y (in which case the standard error of estimate is denoted by sm) or
for nonlinear or multiple regression.
The Linear Correlation Coefficient
Up to now we have defined the correlation coefficient formally by (13)
but have not examined its significance. In attempting to do this, let us
note that from (19) and the definitions of sm and sv, we have
i V 2
r2 :1_ Zo >15/2 (20)
2 (y - .v)
Now we can show that
Z<y—v>2 = 2(,v—>1»,,)2 + Em, —i>2 <21)


%==================================================================%
%%- 112 PROBABILITY AND STATISTICS
The quantity on the left of (21) is called the total variation. The first
sum on the right of (21) is then called the unexplained variation, while
the second sum is called the explained variation. This terminology aris-
es because the deviations y — ym behave in a random or unpredictable
manner while the deviations ym — Y are explained by the least-squares
regression line and so tend to follow a definite pattern. It follows from
(20) and (21) that
Eon, — W
r2 = +2 = explained variation
2 (y _ y ) total variation (22)
Therefore, r2 can be interpreted as the fraction
of the total variation that is explained by the least- _ 
squares regression line. In other words, r measures  ’;
how well the least-squares regression line fits the ‘$3, Q
sample data. If the total variation is all explained
by the regression line, i.e,, r2 = l or r = il , we say
that there is a perfect linear correlation (and in
such case also perfect linear regression). On the
other hand, if the total variation is all unexplained, then the explained
variation is zero and so r : 0. In practice the quantity rl , sometimes call
the coeflicient ofdetermination, lies between O and l.
The correlation coefficient can be computed from either of the
results
mi: Zo-i>o—,v> (23)
W» \/2(x—>?)2\/2(y—§)2
01’
Zo W
'2 I  = explained variation
Z (y _ y ) total variation (24)


%==============================================================================%
%%- CHAPTER 9: Curve Fitting, Regression, and Correlation 1 13
which for linear regression are equivalent. The formula (23) is often
referred to as the product-momenrfurmula for linear regression.
Formulas equivalent to those above, which are often used in prac-
tice, are
, = "ZW"(Z*)(Z>‘) (25)
t/w2#—(2»)’1i»2fi—(2i)21
and
r = a 5‘ ii <26)
vo’ — i’><>»2 —§2>
If We use the transformation on (7), We find
 aw-<2x'><2y'> (2,,
\/MEX" —(2»<')21i"2y"—(Z>")21
which shows that r is invariant under a translation of axes. In particular,
ifh :;?, k : Y, (27) becomes
r =  (28)
i/(2X”)(2>"’)
which is often useful in computation.
The linear correlation coefficient may be positive or negative. If r
is positive, y tends to increase with x (the slope of the least-squares
regression line is positive) while if r is negative, y tends to decrease
with x (the slope is negative). The sign is automatically taken into
account if we use the result (23), (Z5), (26), (27), or (28). However, if
we use (24) to obtain r, we must apply the proper sign.

%=============================================================================================%
%%-114 PROBABILITY AND STATISTICS
\section{Generalized Correlation Coefficient}
The definition (23) [or any equivalent forms (25) through (28)] for the
correlation coefficient involves only sample values x, y. Consequently,
it yields the same number for all forms of regression curves and is use-
less as a measure of fit, except in the case of linear regression, where it
happens to coincide with (24). However, the latter definition, i.e.,
2 2(yen _ y)2
r = +2 explained variation
2 (Y 7 y ) total variation (29)
does reflect the form of the regression curve (via the ym) and so is suit-
able as the definition of a generalized correlation coefficient r. We use
(29) to obtain nonlinear correlation coefficients (Which measure how
Well a nonlinear regression curve fits the data) or, by appropriate gen-
eralization, multiple correlation coeflicients. The connection (I9)
between the correlation coefficient and the standard error of estimate
holds as well for nonlinear correlation.
Example 9.2. Find the coefficient of determination and the coeffi-
cient of correlation from Example 8.2.
Recall that the correlation of determination is r2 :
r2 = explained variation: 1922 = 0.4938
total variation 3892
The coefficient of correlation is simply r.
r2 = ix/04938 = 10,7027

%%--CHAPTER 9: Cun/e Fitting, Regression, and Correlation 115
Since the variable ym increases as x increases (i.e., the slope of the
regression line is positive), the correlation is positive, and we therefore
write r = 0.7027, or r = 0.70 to two significant figures.

Since a correlation coefficient merely measures how well a given
regression curve (or surface) fits sample data, it is clearly senseless to
use a linear correlation coefficient where the data is nonlinear. Suppose,
however, that one does apply (23) to nonlinear data and obtains a value
that is numerically considerably less than 1. Then the conclusion to be
drawn is not that there is little correlation (a conclusion sometimes
reached by those unfamiliar with the fundamentals of correlation theo-
ry) but that there is little linear correlationl There may be in fact a large
nonlinear correlation.
%-----------------------------------------------------------------------------%
\subsection{Correlation and Dependence}
Whenever two random variables X and Y have a nonzero correlation
coefficient, r, we know that they are dependent in the probability sense.
Furthermore, we can use an equation of the form (6) to predict the value
of Y from the value of X4
You Need to Know /
It is important to realize that “correlation” and “depen-
dence” in the above sense do not necessarily imply a
direct causation interdependence of X and Y


%%- 1 16 PROBABILITY AND STATISTICS
\subsection{Example 9.3.} 
If X represents teachers’ salaries over the years while Y represents the amount of crime, the correlation coefficient may be dif-
ferent from zero and We may be able to find a regression line predicting
one variable from the other. But we would hardly be Willing to say that
there is a direct interdependence between X and Y

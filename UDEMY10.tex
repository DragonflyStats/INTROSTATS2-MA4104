



Chapter 10
\chapter{OTHER PROBABILITY DISTRIBUTIONS}

0/ The Multinomial Distribution
8/ The Hypergeometric Distribution
V The uhitehh Distribution
8/ The Cauchy Distribution
8/ The Gamma Distribution
8/ The Beta Distribution
V The Chi-Square Distribution
8/ Student’s t Distribution
0/ The F Distribution
8/ Relationships Among Chi-Square, t,
and F Distributions
1 1 7

%%- Copyright 2001 by the McGraW-Hill Companies, Inc. Click Hem for Terms of Use
%%- 118 PROBABILITY AND STATISTICS
%===========================================================================================%
\section{The Multinomial Distribution}
Suppose that events Al, A2,..., Ak are mutually exclusive, and can occur
with respective probabilities pl, p2,  pk wherepl +p2+ ~-»+pk+ l. If
X1, X2,  Xk are the random variables, respectively, giving the number
of times that A1, A2,..., Ak occur in a total ofn trials, so that XI + X2 +
 X, = n, then
P(X1: "i»X2 :"2»-~-’X1< : "U : %P1n'P;k "'1'/ilk (1)
n1.n2.-»-n,,.
Where nl + n2 + --- nk= n, is thejoint probability function for the random
variables X1, X2,  Xk.
This distribution, which is a generalization of the binomial distribution, is called the multinomial distribution since the equation above is the general term in the multinomial expansion of (pl +p2+ < - pk)”.
%===========================================================================================%
\section{The Hypergeometric Distribution}
Suppose that a box contains b blue marbles and r red marbles. Let us
perform rt trials of an experiment in which a marble is chosen at ran-
dom, its color observed, and then the marble is put back in the box. This
type of experiment is often referred to as sampling with replacement, In
such a case, if X is the random variable denoting the number of blue
marbles chosen (successes) in n trials, then using the binomial distribu-
tion we see that the probability of exactly x successes is
n barn-.t
P(X=x)= ,1 , x=O,l,...,n (2)
x (b+r)
sincep=b/(b+r),q=l—p=r/(b+r).

\section{CHAPTER 10; Other Probability Distributions 119}
If We modify the above so that sampling is without replacement,
i.e., the marbles are not replaced after being chosen, then
x n — x
("I ' J
P(X = x) = 7 , x = max(O, n — r),m, mir1(n,b) (3)
b + r
n
This is the hypergeumetric distribution. 
The mean and variance for this distribution are
_ nb 2_ nbr(b+r—n)
”'b+r’ 6 _(b+r)2(b+r—l) (4)


If we let the total number of blue and red marbles be N, While the
proportions of blue and red marbles are p and q = 1 — p, respectively,
then
_ b _£ _ ' _L
P b+r N’ q b+r N or 1,_1\/P, r=1\/q
This leads us to the following
x —x
(””l l
P(X = x) = L (5)
ii]
N_
/1=np, a2=@1\‘]Z_l") <6)
2=‘
E


%% 120 PROBABILITY AND STATISTICS
Note that as N a <><> (or N is large when compared with rt), these
two formulas reduce to the following
P<X = x> =  <7)
H = rm. 6* =np<1 (8)

Notice that this is the same as the mean and variance for the bino-
mial distribution. The results are just what We would expect, since for
large N, sampling without replacement is practically identical to sam-
pling with replacement.
\subsection{Example 10.1} 
A box contains 6 blue marbles and 4 red marbles. An experiment is performed in which a marble is chosen at random and its
color is observed, but the marble is not replaced. Find the probability that after 5 trials of the experiment, 3 blue marbles will have been chosen.
The number of different Ways of selecting 3 blue marbles out of 6
6
marbles is (33 . The number of different ways of selecting the remaining
4 .
2 marbles out of the 4 red marbles is 2 . Therefore, the number of dif-
6 4
ferent samples containing 3 blue marbles and 2 red marbles is {JKZJ .
Now the total number of different Ways of selecting 5 marbles out
l0
of the 10 marbles (6 + 4) in the box is £5  Therefore, the required
probability is given by


%%- CHAPTER 10: Other Probability Distributions 121
['3] 2‘
%===========================================================================%
\section{The Uniform Distribution}
A random variable X is said to be uniformly distributed in a S x S b if
its density function is
I/(12-0) aSxSb
f(X)={ 0 (9)
otherwise
and the distribution is called a uniform distribution.

The distribution function is given by
O x<a
F(x)=P(X£)c)= (x—a)/(b—a) a§x<b (IO)
1 x2b
The mean and variance arc, respectively
,u=i(a+b), <r2=i(b—a)2 (11)
2 12
%========================================================================%
\section{The Cauchy Distribution}
A random variable X is said to be Cauchy distributed, or to have the
Cauchy distribution, if the density function of X is



122 PROBABILITY AND STATISTICS
'( FL (),_s° oo
f x 7[(X2+a2) a> <1t< (12)
The density function is symmetrical about x = O so that its median
is zero. However, the mean and variance do not exist.
%--------------------------------------------------------------%
\subsection{The Gamma Distribution}
A random variable X is said to have the gamma distribution, or to be
gamma distributed, if the density function is
a—I —x/B
X €
x > 0
f(,() = Bal-(Or) (ot,[3 > O) (I3)
0 x S O
Where F(0t) is the gamma function (see Appendix A). The mean and
variance are given by
,u = 0c,B 0'2 = oz/72 (I4)
%--------------------------------------------------------------%
\subsection{The Beta Distribution}
A random variable is said t0 have the beta distribution, or to be beta distributed, if the density function is
exu—1(l—x)fi—l 0 < x <1
f(X) : B(0t,l3)
O otherwise (01, B > O) (15)



%% CHAPTER 10; Other Probability Distributions 123
where B(0r, fii) is the beta function (see Appendix A), In view of the rela-
tion between the beta and gamma functions, the beta distribution can
also be defined by the density function
Lmx“'l(1—x)p" ()<x<l
fix) I F(0<)F(l3) (16)
O otherwise
where 0t, B are positive. The mean and variance are
=i gm (17)
H a+fi’ G (06+fi)2(06+B+l)
For zx > l, B > 1 there is a unique mode at the value
05-1
= 7 18
xlT\OKl€ a+B_2 ( )
%--------------------------------------------------------------%
\subsection{The Chi-Square Distribution}
Let X1, X2, ...,XV be v independent normally distributed random vari-
ables with mean zero and variance one. Consider the random variable
X2:X§+X§+--»+X§ (19)
Where X2 is called chi square. Then we can show that for all x 2 O,



124 PROBABILITY AND STATISTICS
P(Xz§x): {u</2>le /» du (20)
and P(X2 £ x) = O for x > 0.
The distribution above is called the chi-square distribution, and v is
called the number 0f degrees 0f freedom. The distribution defined above
has corresponding density function given by
l tv/2)—1 1-/2
V/2 X e x > O
fix): 2 l“(v/2) (21)
O XSO
It is seen that the chi-square distribution is a special case of the
gamma distribution with 0: = v/ 2 and fi = 2. Therefore,
/1 = v, 0'2 = 2v (22)
For large v (v 2 30), we can show that \/2X2 7 1/2v 71 is very near-
ly normally distributed with mean 0 and variance one.
Three theorems that Will be useful in later work are as follows:
\subsection*{Theorem 10-1:} Let X1, X2,  XV be independent normally random
variables With mean 0 and variance 1. Then X2 = X? +
X§ +  + Xf is chi square distributed with v degrees of
freedom.
\subsection*{Theorem 10-2:} Let U], U2, ..., Uk be independent random variables
that are chi square distributed with vl, v2,  vk
degrees of freedom, respectively. Then their sum W =
U‘ + U2 +»--U,‘ is chi square distributed with v| + v2 +
~~-vk degrees of freedom.



CHAPTER 10; Other Probability Distributions 125
\subsection*{Theorem 10-3:} Let V1 and V2 be independent random variables.
Suppose that V1 is chi square distributed with vl
degrees of freedom while V = V1 = V2 is chi square
distributed with v degrees of freedom, where v > v|,
Then V, is chi square distributed with v — vl degrees
of freedom.

In connection with the chi-square distribution, the I distribution, the F distribution, and others, it is common in statistical work to use the same symbol for both the random variable and a value of the random variable. Therefore, percentile values of the chi-square distribution for v degrees of freedom are denoted by Xi“, , or briefly Xi, if v is under-
stood, and not by XI“, or xp. (See Appendix D.) This is an ambiguous
notation, and the reader should use care with it, especially when chang-
ing variables in density functions.

\subsubsection{Example 10.2.} The graph of the chi-square distribution wizth E
degrees of freedom is shown in Figure 10-1. Find the values for Xi » X2
for which the shaded area on the right = 0.05 and the total shaded area
= 0,05.
zf xi z’
Figure 10-1



126 PROBABILITY AND STATISTICS
If the shaded area on the right is 0.05, then the area to the left of X; is
(1 — 0.05) = 0.95, and Xg represents the 95“ percentile, X395.
Referring to the table in Appendix D, proceed downward under
the column headed v until entry 5 is reached. Then proceed right to the
column headed X395. The result, 11.1, is the required value of 12.
Secondly, since the distribution is not symmetric, there are many
values for which the total shaded area : 0.05. For example, the right-
handed shaded area could be 0.04 while the left-handed area is 0.01. It
is customary, however, unless otherwise specified, to choose the two
areas equal. In this case, then, each area : 0.025.
1f the shaded area on the right is 0.025, the area to the left of Xi is
1 — 0.025 = 0.975 and Xg represents the 97.5"‘ percentile X5975 , which
from Appendix D is 12.8.
Similarly, if the shaded area on the left is 0.025, the area to the left
of Xf is 0.025 and X? represents the 2.5"‘ percentile, X3 025 , which
equals 0.831.
Therefore, the values are 0.831 and 12.8.
Student’s t Distribution
If a random variable has the density function
 2 —(v+|)/2
f(;):%(1+Lj —><><r<<» (23)
M rial V
it is said to have the Student's r distribution, briefly the I distribution,
with v degrees of freedom. If v is large (v 2 30), the graph off(z) close-
ly approximates the normal curve, as indicated in Figure 10-2.



CHAPTER 10: Other Probability Distributions 127
I0)
M Normal
\
v-4
v-l
02 X
0|
-r s-a .; . ". ;“.,
Figure 10-2
Percentile values of the t distribution for v degrees of freedom
are denoted by rm or briefly tp if v is understood. For a table giving
such values, see Appendix C, Since the I distribution is symmetrical,
tl_p = —tp; for example, £05 = -1095.
For the t distribution we have
ii=0 and a2=i (1/>2) <24)
v — 2
The following theorem is important in later Work.
Theorem 10-4: Let Yand Z be independent random variables, where
Y is normally distributed with mean O and variance l
While Z is chi square distributed With v degrees of
freedom. Then the random variable
TIL
\Z/v (25)
has the t distribution with v degrees of freedom.



128 PROBABILITY AND STATISTICS
Example 10.3. The graph of Student’s t distribution with 9 degrees
of freedom is shown in Figure l0-3. Find the value of tl for which the
shaded area on the right = 0405 and the total unshaded area = O99.
-1, I,
Figure 10-3
If the shaded area on the right is O05, then the area to the left of I1 is (1
— O05) = 0095, and 1| represents the 95th percentile, 10.95. Referring to
the table in Appendix C, proceed downward under the column headed v
until entry 9 is reached. Then proceed right to the column headed I035.
The result 1.83 is the required value of t4
Next, if the total unshaded area is 0.99, then the total shaded area
is (l — O99) = 0.01, and the shaded area to the right is 0.01 / 2 = 0.005.
From the table we find tn 995 = 3.25.
%--------------------------------------------------------------%
\subsection{The F Distribution}
A random variable is said to have the F distribution (named after R. A‘
Fisher) with vl and v2 degrees nffreedom if its density function is given
by
F vl +v2
MVP/iv;/2u<»',/21-1(v2 + VIu)—<»\,+\@>/z u > 0
f(u) =  (26)
O u S 0



Percentile
are denoted by
CHAPTER 10; Other Probability Distributions 129
values of the F distribution for v], v2 degrees of freedom
FWN2, or briefly Fp if vl and v2 are understood.
For a table giving such values in the case where p = 0.95 and p =
0.99, see Appendix E.
The mean
and variance are given, respectively, by
2v§(v + V2 +2)
=£ 2 a 2:‘? 27
ll V2_2 (v2>) an G ‘/|(Vz_4)(V2_2)2 ( )
The distribution has a unique mode at the value
v —2 v
  ‘V->2) <28>
The following theorems are important in later Work.
Theorem 11-5
Theorem 10-6:
: Let VI and V2 be independent random variables that
are chi square distributed with vl and v2 degrees of
freedom, respectively. Then the random variable
V  (29)
V2/V2
has the F distribution with vl and v2 degrees of free-
dom.
l
F-  = — 00>
I P ‘ 



130 PROBABILITY AND STATISTICS
bution are all valid for large sample
Remember
While specially used with small sam-
ples, Student‘s tdistribution, the chi-
square distribution, and the F distri-
sizes as well.
Relationships Among Chi-Square,
t, and F Distributions
Theorem 10-7: n_,,v,_\, : l§_(,,,m (31)
2
Theorem 10-s= 1r,,v,,,,<, =  <32)
V
Example 10.4. Verify Theorem I0-7 by showing that I395 = t§_975 .
Compare the entries in the first column of the FM5 table in Appendix E
with those in the t distribution under toms. We see that
161 = (12471)2’ 18,5 = (4t3O)Z, 10t1 =(3.18)2, 7.71=(2t78)Z,etct,
which provides the required verification.
Example 10.5. Verify Theorem I0-8 forp = 0.99.
Compare the entries in the last row of the F 0.99 table in Appendix E (cor-
responding to vz = w) with the entries under X3 99 in Appendix D. Then
we see that



CHAPTER 10: Other Probability Distributions 131
6.63: 6'63, 4.61: 9'21, 3.78: U3, 3.32:13'3, etc.,
1 2 3 4
which provides the required verification.


\documentclass{beamer}

\usepackage{amsmath}
\usepackage{amssymb}

\begin{document}


\begin{frame}


\begin{frame}
\frametitle{Quantiles}
\Large
\vspace{-1cm}
\begin{itemize}
\item Quantiles are statistics that describe various subdivisions of a frequency distribution into equal proportions. 
\item The simplest division that can be envisioned is into two equal halves and the quantile that does this, the 
median value of the variate, is used also as a measure of central tendency for the distribution.
\end{itemize}
\end{frame}
%----------------------------------------------------------------%
\begin{frame}
\frametitle{Quantiles}
\Large
\vspace{-1cm}
\begin{itemize}
\item When division is into four parts the values of the variate corresponding to 25\%, 50\% and 75\% of the total 
distribution are called quartiles.\item The difference between the 1st and 3rd quartiles is called the inter-quartile range. 
\item It embraces the central 50\% of the distribution and gives a measure of the dispersion of the distribution. 
\item The 2nd quartile is just the median under another name.
\end{itemize}
\end{frame}
%----------------------------------------------------------------%
\begin{frame}
\frametitle{Quantiles}
\Large
\vspace{-1cm}
\begin{itemize}
\item Other quantiles that may be encountered include quintiles (distribution divided at 20\%, 40\%, 60\% and 80\%), 
deciles (inter-decile range from 1st decile to 9th decile holds 80% of the distribution) and percentiles.
\end{itemize}
\end{frame}
%----------------------------------------------------------------%
\end{document}
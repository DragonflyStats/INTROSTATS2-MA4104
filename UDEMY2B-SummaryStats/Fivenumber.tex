\documentclass{beamer}

\usepackage{amsmath}
\usepackage{amssymb}


\begin{document}

\begin{frame}
\Huge
\[\mbox{Introduction to Statistics}\]
\huge
\[\mbox{The Five-number Summary, the IQR,  }\]
\[\mbox{the Midhinge and the Trimean}\]
\Large
\[\mbox{www.Stats-Lab.com}\]
\[\mbox{Twitter: @StatsLabDublin}\]
\end{frame}
%------------------------------------------------------------------------------- %
\begin{frame}
\frametitle{The Five-number Summary}
\Large
The five-number summary is a descriptive statistic that provides information about a set of observations. 

It consists of the five most important sample percentiles:
\begin{itemize}
\item the sample minimum (smallest observation)
\item the lower quartile or first quartile ($Q_1$)
\item the median (middle value)
\item the upper quartile or third quartile ($Q_3$)
\item the sample maximum (largest observation)
\end{itemize}
\end{frame}

\begin{frame}
\frametitle{The Five-number Summary : Sample Data}
\Large
Suppose a sample has the following five-number summary: 
\begin{itemize}
\item the sample minimum = 11
\item the lower quartile $Q_1$ = 25
\item the median = 27
\item the upper quartile $Q_3$ = 31
\item the sample maximum $ = 38$
\end{itemize}
We will use these values for later examples.
\end{frame}
%------------------------------------- %
\begin{frame}
\frametitle{Interquartile Range}
\vspace{-0.5cm}
\Large
The interquartile range (IQR) is a measure of statistical dispersion, being equal to the difference between the first and third quartiles,
\[IQR = Q_3 -  Q_1. \]

For our sample data, the interquartile range is 

\[IQR = 31 -  25 = 6 \]

(The median is the corresponding measure of location or central tendency.)
\end{frame}
\begin{frame}
\frametitle{Interquartile Range and Outliers}
\Large
The interquartile range is often used to find outliers in data.

 Using this approach, outliers are observations that fall below the \textbf{lower fence} \[\mbox{Lower fence} = Q1 - (1.5\times IQR)\] or above the \textbf{upper fence} \[\mbox{Upper fence} = Q3 + (1.5 \times IQR)\].
\end{frame}
%-------------------------------------------------- %
\begin{frame}
\frametitle{Interquartile Range and Outliers}
\Large
\textbf{Lower Fence}
\begin{itemize}
\item Lower fence $= Q1 - (1.5\times IQR)$
\item Lower fence $= 25 - (1.5\times 6) = 25 - 9 = 16$
\end{itemize}
Any value less than 16 (for example, the minimum value 11) is considered an outlier.\\
\vspace{0.5cm}
\end{frame}
%-------------------------------------------------- %
\begin{frame}
\frametitle{Interquartile Range and Outliers}
\Large
\textbf{Upper Fence}
\begin{itemize}
\item Upper fence$ = Q3 + (1.5 \times IQR)$
\item Upper fence$= 31 + (1.5\times 6) = 31 + 9 = 40$
\end{itemize}

Any value greater than 40 would considered an outlier. However, as the maximum value is 38, there is no high-value outliers

\end{frame}
%-------------------------------------------------- %
\begin{frame}
\frametitle{Midhinge}
\Large
The midhinge is a measure of central location, determined as the average of the first and third quartiles.

\[  \mbox{midhinge} = \frac{Q_1 + Q_3}{2}\]

For the sample data, the midhinge is computed as

\[  \mbox{midhinge} = \frac{25 + 31}{2} = \frac{56}{2} = 28 \]


\end{frame}
%-------------------------------------------------- %
\begin{frame}
\frametitle{Trimean}
\Large
The \textbf{trimean} (TM) is a measure of central location defined as a weighted average of the distribution's median and its two quartiles:
\[TM= \frac{Q_1 + 2Q_2 + Q_3}{4}\]
This is equivalent to the average of the median and the midhinge:
\[TM= \frac{1}{2}\left(Q_2 + \frac{Q_1 + Q_3}{2}\right)\]

\end{frame}
%-------------------------------------------------- %
\begin{frame}
\frametitle{Trimean}
\Large
\[TM= \frac{Q_1 + 2Q_2 + Q_3}{4}\]

For the sample data, the trimean is computed as
\[TM= \frac{25 + (2\times 27) + 31}{4} = \frac{110}{4} = 27.5\]

\end{frame}
\begin{frame}

\end{frame}
\end{document}
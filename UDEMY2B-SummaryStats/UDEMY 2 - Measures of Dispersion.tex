\documentclass{beamer}

\usepackage{amsmath}
\usepackage{amssymb}

\begin{document}


\section{Measures of Dispersion}
\begin{frame}
It is unlikely we will only be interested in the average value of our data, we will want
to know how large the spread or dispersion of values is about it. 

\noindent The simplest measure of dispersion is the \textbf{range}.
The range is simply the difference of the lowest and highest values. The range is
another easy-to-understand measure, but it will clearly be very affected by a few
extreme values.
\end{frame}
%------------------------------------------------------------------ %
\begin{frame}
\frametitle{Measures of Dispersion}
Aside from the range, the most common measures of dispersion are:
\begin{itemize}
\item Variance
\item Standard deviation
\item Mean Absolute Deviation (MAD)
\item Inter-quartile range.
\end{itemize}
\end{frame}
%------------------------------------------------------------------ %
\begin{frame}
\frametitle{Measures of Dispersion}
\begin{itemize}
\item The first three are related to the use of the arithmetic mean, and are computed using deviations of each observation from the mean.
\item  The mean absolute deviation (MAD) uses the absolute values of the deviations from
the mean and perhaps gives us a more intuitively understandable measure of deviation than
variance and standard deviation.
\end{itemize} 
\end{frame}
\section{Coefficient of variation}
\begin{frame}
\frametitle{Coefficient of Variation}
The coefficient of variation, CV, indicates the relative magnitude of the standard deviation as compared
with the mean of the distribution of measurements, as a percentage. Thus, the formulas are
\begin{eqnarray*}
\mbox{ Population : } CV = \frac{\sigma}{\mu } \times 100 \\
\mbox{ Sample : } CV = \frac{s}{\bar{x}} \times 100
\end{eqnarray*}

The coefficient of variation is useful when we wish to compare the variability of two data sets relative to the
general level of values (and thus relative to the mean) in each set.
\end{frame}
%------------------------------------------------------------------ %
\end{document}
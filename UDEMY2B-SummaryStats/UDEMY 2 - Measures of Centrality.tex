\documentclass{beamer}

\usepackage{amsmath}
\usepackage{amssymb}
\usepackage{framed}
\begin{document}
% Summations
% Factorial
% Absolute Value


%---------------------------------------------------------------%
\begin{frame}
\frametitle{Sampling Error}
\Large
\begin{itemize}
\item A \textbf{statistical error} to which an analyst exposes a model simply because he or she is working with sample data rather than population or census data.
\item Using sample data presents the risk that results found in an analysis do not represent the results that would be 
obtained from using data involving the entire population from which the sample was derived.
\end{itemize}
\end{frame}
%---------------------------------------------------------------%
\begin{frame}
\frametitle{Sampling Error}
\Large
\begin{itemize}
\item The use of a sample relative to an entire population is often necessary for practical or budgetary reasons. 
\item Although there are likely to be some differences between sample analysis results and population analysis results, the degree to which these can 
differ is not expected to be substantial. 
\end{itemize}
\end{frame}
%---------------------------------------------------------------%
\begin{frame}
\frametitle{Sampling Error}
\Large
\begin{itemize}
\item Methods of reducing sampling error include increasing the sample size and ensuring that the sample adequately represents the entire population.
\end{itemize}
\end{frame}
%---------------------------------------------------------------%
\begin{frame}
\frametitle{Non-Sampling Error}
\Large
\begin{itemize}

\item A statistical error caused by human error to which a specific statistical analysis is exposed. 
\item These errors can include, but are not limited to, data entry errors, biased questions in a
 questionnaire, biased processing/decision making, inappropriate analysis conclusions and false 
information provided by respondents.
\end{itemize}
\end{frame}
%---------------------------------------------------------------%
\begin{frame}
\frametitle{Non-Sampling Error}
\Large
\begin{itemize}
\item 
Non-sampling errors are part of the total error that can arise from doing a statistical analysis. 
\item The remainder of the total error arises from sampling error. 
\item Unlike sampling error, increasing the sample size will not have any effect on reducing non-sanpling error.
\item Unfortunately, it is virtually impossible to eliminate non-sampling errors entirely. 
\end{itemize}
\end{frame}
%---------------------------------------------------------------%
%
%---------------------------------------------------------------%
\begin{frame}
\frametitle{Probability Rules}
\Large
\vspace{-1cm}
There are two rules which are very important.
\begin{itemize}
\item All probabilities are between 0 and 1 inclusive
   \[0 \leq P(E) \leq 1\]
\item The sum of all the probabilities in the sample space is 1
\end{itemize}

\end{frame}
%---------------------------------------------------------------%

%---------------------------------------------------------------%
\begin{frame}
\frametitle{Probability Rules}
\begin{itemize}

\item The probability of an event which cannot occur is 0.

\item The probability of any event which is not in the sample space is zero.

\item The probability of an event which must occur is 1.

\item The probability of the sample space is 1.
\end{itemize}
\end{frame}
%---------------------------------------------------------------%
\begin{frame}
\frametitle{Probability Rules}
\textbf{The Complement Rule}
\begin{itemize}
\item 
The probability of an event not occurring is one minus the probability of it occurring.

   \[P(E^{C}) = 1 - P(E)\]
\end{itemize}
\end{frame}
%---------------------------------------------------------------%
%---------------------------------------------------------------%
\begin{frame}
\frametitle{Probability: Addition Rule for Any Two Events}
\Large
\vspace{-0.3cm}
\begin{itemize}
\item For any two events A and B, the probability of A or B is the sum of the probability of A and the probability of B minus the probability of both A and B:
\[P(A \cup B) = P(A) + P(B) - P(A \cap B)\]
\vspace{-0.2cm}
\item We subtract the probability of $A \cap B$ to prevent it getting counted twice.
\large
\item \textit{($A \cup B$ and $A \cap B$ denotes  ``A or B" and ``A and B" respectively) }

\end{itemize}
\end{frame}
%---------------------------------------------------------------%
%---------------------------------------------------------------%
\begin{frame}
\frametitle{Probability: Addition Rule for Any Two Events}
\Large
\begin{itemize}
\item If events A and B are \textbf{mutually exclusive}, then the probability of A or B is the sum of the probability of A and the probability of B:

\[P(A \cup B) = P(A) + P(B)\]

\item If A and B are mutually exclusive, then the probability of both A and B is zero.
\end{itemize}
\end{frame}
%---------------------------------------------------------------%
%---------------------------------------------------------------%
\begin{frame}
\frametitle{Coefficient Of Variation (CV)}
\Large
\begin{itemize}
\item This is a statistical measure of the dispersion of data points in a data series around the mean. 
\item It is calculated as follows:

\[  \mbox{ CV}  = \frac{\sigma}{\mu}  \]
\[  \mbox{ CV}  = \frac{s}{\bar{x}}  \]

\item The coefficient of variation represents the ratio of the standard deviation to the mean

\item The coefficient of variation is a useful statistic for comparing the degree of dispersion from one data set
 to another, even if the means are very different from each other.
\end{itemize}
\end{frame}
%---------------------------------------------------------------%

%------------------------------------------------------------------------------%
\begin{frame}
% http://www.wyzant.com/resources/lessons/math/statistics_and_probability/introduction/data
\frametitle{Data}
\Large
\begin{itemize}
\item Data can be defined as groups of information that represent the qualitative or quantitative attributes of a variable or set of variables, which is the same as saying that data can be any set of information that describes a given entity. Data in statistics can be classified into grouped data and ungrouped data.

\item Any data that you first gather is ungrouped data. 
\item Ungrouped data is data in the raw. An example of ungrouped data is a any list of numbers that you can think of.
\end{itemize}
\end{frame}
%------------------------------------------------------------------------------%
\begin{frame}
\frametitle{Grouped Data}
\Large
\begin{itemize}
\item Grouped data is data that has been organized into groups known as classes. 
\item Grouped data has been 'classified' and thus some level of 
data analysis has taken place, which means that the data is no longer raw.
\item A data class is group of data which is related by some user defined property. 
\item 
For example, if you were collecting the ages of the people you met as you walked down the street, you could group them into classes as those in 
their teens, twenties, thirties, forties and so on. Each of those groups is called a class.
\end{itemize}
\end{frame}
%------------------------------------------------------------------------------%
\begin{frame}
% http://www.wyzant.com/resources/lessons/math/statistics_and_probability/introduction/data
\frametitle{Class Intervals}
\Large
\begin{itemize}
\item Each of those classes is of a certain width and this is referred to as the \textbf{Class Interval}. 
\item This class interval is very important when it comes to drawing Histograms and Frequency diagrams. 
\item All the classes may have the same class size or they may have different classes sizes depending on how you group your data. 
\item The class interval is always a whole number.
\end{itemize}

\end{frame}
%------------------------------------------------------------------------------%
\begin{frame}
% http://www.wyzant.com/resources/lessons/math/statistics_and_probability/introduction/data
Below is an example of grouped data where the classes have the same class interval.
\begin{center}
\begin{tabular}{|c|c|}
\hline Age (years)	&	Frequency	\\ \hline
0 - 9	&	12	\\ \hline
10 - 19	&	30	\\ \hline
20 - 29	&	18	\\ \hline
30 - 39	&	12	\\ \hline
40 - 49	&	9	\\ \hline
50 - 59	&	6	\\ \hline
60 - 69	&	0	\\ \hline
\end{tabular} 
\end{center}

\end{frame}
%------------------------------------------------------------------------------%
\begin{frame}
Solution:

Below is an example of grouped data where the classes have different class interval.

Age (years)	 Frequency		 Class Interval
0 - 9	 	15	 	10
10 - 19	 	18	 	10
20 - 29	 	17	 	10
30 - 49	 	35	 	20
50 - 79	 	20	 	30

\end{frame}
%------------------------------------------------------------------------------%
\begin{frame}
Calculating Class Interval
Given a set of raw or ungrouped data, how would you group that data into suitable classes that are easy to work with and at the same time meaningful?

The first step is to determine how many classes you want to have. 
Next, you subtract the lowest value in the data set from the highest value in the data set and then you divide by the number of classes that you want to have:

\end{frame}
%------------------------------------------------------------------------------%
\begin{frame}

Example 1:

Group the following raw data into ten classes.



Solution:

The first step is to identify the highest and lowest number





\end{frame}
%------------------------------------------------------------------------------%
\begin{frame}


Class interval should always be a whole number and yet in this case we have a decimal number. The solution to this problem is to round off to the nearest whole number.

In this example, 2.8 gets rounded up to 3. So now our class width will be 3; meaning that we group the above data into groups of 3 as in the table below.


\end{frame}
%------------------------------------------------------------------------------%
\begin{frame}

Number	 Frequency
1 - 3	 7
4 - 6	 6
7 - 9	 4
10 - 12	 2
13 - 15	 2
16 - 18	 8
19 - 21	 1
22 - 24	 2
25 - 27	 3
28 - 30	 2


\end{frame}
%------------------------------------------------------------------------------%
\begin{frame}
\frametitle{Class Limits and Class Boundaries}
\Large
\vspace{-1cm}
\begin{itemize}
\item 
Class limits refer to the actual values that you see in the table. 
\item Taking an example of the table above, 1 and 3 would be the class limits of the first class. 
\item 
Class limits are divided into two categories: lower class limit and upper class limit. \item In the table above, for the first class, 1 is the lower class limit while 3 is the upper class limit.
\end{itemize}
\end{frame}
%------------------------------------------------------------------------------%
\begin{frame}
\frametitle{Class Limits and Class Boundaries}
\Large
\vspace{-1cm}
\begin{itemize}
\item On the other hand, class boundaries are not always observed in the frequency table.
\item  Class boundaries give the true class interval, and similar to class limits, are also divided into lower and upper class boundaries.

\item The relationship between the class boundaries and the class interval is given as follows:
\end{itemize}

\end{frame}
%------------------------------------------------------------------------------%
\begin{frame}

Class boundaries are related to class limits by the given relationships:





As a result of the above, the lower class boundary of one class is equal to the upper class boundary of the previous class.

Class limits and class boundaries play separate roles when it comes to representing statistical data diagrammatically as we shall see in a moment.
\end{frame}
%---------------------------------------------------------------%
\begin{frame}
\frametitle{Sampling Distribution}
\begin{itemize}

\item A probability distribution of a statistic obtained through a large number of samples drawn from a specific population. 
\item The sampling distribution of a given population is the distribution of frequencies of a range of different outcomes 
that could possibly occur for a statistic of a population. 
\end{itemize}
\end{frame}
%---------------------------------------------------------------%

\begin{frame}[fragile]
\frametitle{Measures of Centrality}
\Large
\begin{center}
\begin{framed}

\begin{verbatim}
       4.10, 4.10, 4.25, 4.25, 4.25,  
       4.35, 4.40, 4.53, 4.90, 5.20, 
       5.26, 5.35, 5.45, 5.71, 6.09, 
       6.10, 6.30, 6.50, 6.80, 7.11.
\end{verbatim}

\end{framed}
\end{center}
\begin{itemize}
\item There are 20 values in this data set.
\item The sum of the values is 105.
\end{itemize}
\end{frame}

\begin{frame}
\frametitle{The Mean Absolute Deviation} 
\begin{itemize}
\item The mean absolute deviation, or MAD, is based on the absolute value of the difference between each value
in the data set and the mean of the group. 
\item It is sometimes called the “average deviation.” 
\item The mean average of
these absolute values is then determined. 
\item The absolute values of the differences are used because the sum of all
of the plus and minus differences (rather than the absolute differences) is always equal to zero. 
\item Thus the
respective formulas for the population and sample MAD are
\[ \mbox{Population MAD }= \frac{\sum |x_i - \mu|}{N} \]
\[ \mbox{Sample MAD }= \frac{\sum |x_i - \bar{x}|}{n} \]
\end{itemize}
\end{frame}
\end{document}

\begin{frame}
\frametitle{Arithmetic mean} 
\begin{itemize}
\item One of the basic quantities is the \textbf{arithmetic mean} (it is sometimes
called the `average’ but there are in fact other measures of average apart from the
mean ). 
\item The arithmetic mean is calculated by
adding the measures of the number of observations in which you are interested and
dividing by the number of observations.

\[ \bar{x} =  \frac{\sum x}{n}  \]

\item For our data set $\bar{x} = \frac{22}{5}  = \textbf{4.4}$.
\end{itemize}
\end{frame}
%-------------------------------------------------------------- %
\section{Medians and modes}
\begin{frame}

\begin{itemize}
\item The median ($\tilde{x}$) is the value that separates a sample into two groups; 50\% of observations are greater
than the median and 50\% are less than it.
\item The set of n numbers is arranged in ascending order, say as $x_(1), x_(2), x_(3), \dots x_(n)$, where $x_(1)$ is the smallest of the observations and $x_(n)$ is the largest.
\item Computation of the median differs for samples that have an odd number size, and samples with an even number size. If sample size $n$ is odd
\[ \tilde{x} =  x_{(\frac{n+1}{2})} ,\]
or if $n$ is even 
\[ \tilde{x} =  \frac{  x_{(\frac{n}{2})}  + x_{(\frac{n+1}{2})} }{2}. \]
\end{itemize}
\end{frame}
%-------------------------------------------------------------- %
\begin{frame}
\begin{itemize}
\item A visual inspection of the ordered data set will be useful for a quick determination of the median.
\item For example, the median of the numbers 1, 3, 4, 5 and 7 is 4 (check this by
rearranging the numbers in order!) and for 1, 3, 4 and 5, the median is (3+4)/2, that is
3.5.
\item The \textbf{mode} is the most frequently occurring value. There is not necessarily only one
such value. 
\item For example, the figures 1, 2, 2, 3, 5, 9, 9, 11 have two modes: the
numbers 2 and 9.
\end{itemize}
\end{frame}
\end{document}

\documentclass{beamer}

\usepackage{amsmath}
\usepackage{amssymb}

\begin{document}

%-----------------------------------------------------------------------------------------------------------%
\section*{Exploratory Data Analysis}

\begin{frame}

\frametitle{Analyzing the Data}
\Large
\vspace{-1cm}
\begin{itemize}
\item Statistical data analysis divides the methods for analyzing data into two categories: exploratory methods and confirmatory methods. 

\item Exploratory methods are used to discover what the data seems to be saying by using simple arithmetic and easy-to-draw pictures to summarize data. 
\end{itemize}
\end{frame}
%-----------------------------------------------------------------------------------------------------------%

\begin{frame}
\frametitle{Exploratory Data Analysis}
\Large
\vspace{-1cm}
The objectives of Exploratory Data Analysis are to:
\begin{itemize}
\item Suggest hypotheses about the causes of observed phenomena,
\item Assess assumptions on which statistical inference will be based,
\item Support the selection of appropriate statistical tools and techniques,
\item Provide a basis for further data collection through surveys or experiments.
\end{itemize}
%\bigskip
%Confirmatory methods use ideas from probability theory in the attempt to answer specific questions. 
\end{frame}
%-----------------------------------------------------------------------------------------------------------%
\begin{frame}
\frametitle{The Weighted Mean}
\Large
\vspace{-0.7cm}
\begin{itemize}
\item  The weighted mean (or weighted average) is an arithmetic mean in which each value is weighted according
to its importance in the overall group. 
\vspace{0.4cm}
\item The formulas for the population, and sample weighted means are
identical:
\[ \mu_w = \bar{x}_w =\frac{\sum w_i x_i}{\sum w_i} \]
%Operationally, each value in the group ($x_i$) is multiplied by the appropriate weight factor ($w_i$), and the products are then summed and divided by the sum of the weights.
\end{itemize}
 
\end{frame}
\end{document}
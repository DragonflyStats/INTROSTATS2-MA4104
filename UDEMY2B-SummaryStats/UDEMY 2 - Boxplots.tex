
\documentclass{beamer}

\usepackage{amsmath}
\usepackage{amssymb}

\begin{document}

%------------------------------------------------------------------------------------------%
\begin{frame}
\frametitle{Five number summary and the boxplot}
The five number summary consists of the smallest value, the lower quartile, the median, the upper quartile and the largest value in ascending order.

A quarter of the measurements in the data set lie between each of the four pairs of values. 

The five-number summary can be used to create a simple graph called a boxplot to visually describe the distribution.
\end{frame}

%------------------------------------------------------------------------------------------%
\begin{frame}
\frametitle{Box-plots}

From the boxplot, you can quickly detect any skewing in the shape of the distribution and see whether there are any outliers present in the data set.

An outlier can be caused by human error when entering data or by malfunctioning equipment. 

However, outliers can also be valid measurements, and for this reason, it is necessary to isolate them as soon as possible in the analysis. 
The Box-plot was designed for this purpose.
\end{frame}

%------------------------------------------------------------------------------------------%
\begin{frame}
\frametitle{Constructing Box-plots}

1) Calculate Q1, the median, Q3 and the IQR.

2) Draw a horizontal line to represent the scale of measurement.


3) Draw a box just above  the line with the right and left ends at Q1 and Q3.

4) Draw a line through the box at the location of the median.

\end{frame}

%------------------------------------------------------------------------------------------%
\begin{frame}
\frametitle{Constructing Box-plots}
5) To detect outliers you need to determine a lower fence and an upper fence.	
a. Lower fence is 
b. Upper fence is 

6) Any values below the lower fence or above the upper fence are classes as outliers.

7) To finish the boxplot
a. Mark any outliers with an asterisk (*) on the graph.
b. Extend horizontal lines , called whiskers, from the ends of the box to the smallest and largest values that are not outliers.
c. (Remark – a variation is to extend to the lower and upper fences)

	

\end{frame}

%------------------------------------------------------------------------------------------%
\begin{frame}
\frametitle{What to Look for}

1) Outliers
The first feature that you look for when analysing a boxplot is the presence of outliers.
 
Outliers are extreme values and can greatly influence your analysis. For that reason, you should check your data and make sure you have entered it correctly.
 
You also have the option of removing outliers, making a note that you have removed them, and presenting your analysis without them.
 
2) Skewed Data
The second feature is the degree of skewness. As you learned earlier, the quartiles divide the data into four sections, each containg 25% of the measurements. 

You are interested in how spread out or tightly packed the data are. The length of the whiskers and the position of the median in the box tell you this. Notice that 25% of the values in the boxplot are less than Q1 and this includes the outliers.


\end{frame}

%------------------------------------------------------------------------------------------%
\begin{frame}


%[Page 30]

\frametitle{IQR}
The third feature is the variation/dispersion around the median. The IQR is the middle 50% of the data. When you are dealing with skewed data, the IQR is the most reliable measure of variation. Outliers affect the mean, making it an unrealistic measure of centrality (for symmetric data).

The most common use of box plots is for comparing two data sets on the same scale. 

For now, it is important that you are clear what a box-plot tells you about a distribution of data and what measure of centrality and variability are most appropriate based on the distribution.
\end{frame}
%------------------------------------------------------------------------------------------%
\end{document}
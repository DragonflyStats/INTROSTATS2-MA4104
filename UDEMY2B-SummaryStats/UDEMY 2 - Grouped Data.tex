\documentclass{beamer}

\usepackage{amsmath}
\usepackage{amssymb}

\begin{document}

%-----------------------------------------------------------------------------------------------------------%
\section*{Statistics for grouped data}

\begin{frame}
\frametitle{Statistics for grouped data}
grouped data refers to the arrangement of raw data with a wide range of values into groups. This process makes the data more manageable. Graphs and frequency diagrams can then be drawn showing the class intervals chosen instead of individual values.


\noindent An estimate, $\bar{x}$, of the mean of the population from which the data are drawn can be calculated from the grouped data as:
\[ \bar{x} = \frac{\sum f x }{\sum f}\]
In this formula, $x$ refers to the mid-point of the class intervals, and $f$ is the class frequency. Note that the result of this will be different from the sample mean of the ungrouped data.

\end{frame}
\begin{frame}
\begin{tabular}{|c|c|c|}
\hline
Class limits& Class midpoint & frequency \\
  \hline
\$240 - 259.99 & \$250 &7\\
\$260 - 279.99 & \$270 &20\\
\$280 - 299.99 & \$290 &33\\
\$300 - 319.99 & \$310 &25\\
\$320 - 339.99 & \$330 &11\\
\$340 - 359.99 & \$350 &4\\
\hline
& & Total = 100\\
  \hline
\end{tabular}
\end{frame}

\begin{frame}
\frametitle{Cumulative frequency}
The graph of a cumulative frequency distribution is called an ogive (pronounced ``o-jive"). For the less-than
type of cumulative distribution, this graph indicates the cumulative frequency below each exact class limit of
the frequency distribution. When such a line graph is smoothed, it is called an ogive curve.
\end{frame}

\begin{frame}
\frametitle{Relative frequency}
A relative frequency distribution is one in which the number of observations associated with each class has
been converted into a relative frequency by dividing by the total number of observations in the entire
distribution. Each relative frequency is thus a proportion, and can be converted into a percentage by multiplying
by 100.\\

\noindent One of the advantages associated with preparing a relative frequency distribution is that the cumulative
distribution and the ogive for such a distribution indicate the cumulative proportion (or percentage) of
observations up to the various possible values of the variable. A percentile value is the cumulative percentage of
observations up to a designated value of a variable.
\end{frame}

\end{document}
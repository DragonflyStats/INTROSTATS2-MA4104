

%-----------------------------------------------------------------------------------------%

\section{Chi Square test for goodness of fit}

The chi-squared test applied to contingency tables.

The Chi-squared test is the most commonly used test for frequency data and goodness-of-fit. In theory, it is nonparametric but because it has no parametric equivalent, it is not classified as such. It is not an exact test and with the current level of computing facilities, there is not much excuse not to use Fisher’s exact test for 2x2 contingency table analysis instead of Chi-squared test. Also for larger contingency tables, the G-test (log-likelihood ratio test) may be a better choice. The Chi-square value is obtained by summing up the values (residual2/fit) for each cell in a contingency. In this formula, residual is the difference between the observed value and its expected counterpart and fit is the expected value.


%-----------------------------------------------------------------------------------------%

\subsection{Yates's correction}

The approximation of the Chi-square statistic in small $2\times2$ tables can be improved by reducing the absolute value of differences between expected and observed frequencies by 0.5 before squaring. This correction, which makes the estimation more conservative, is usually applied when the table contains only small observed frequencies (<20).

The effect of this correction is to bring the distribution based on discontinuous frequencies nearer to the continuous Chi-squared distribution. This correction is best suited to the contingency tables with fixed marginal totals. Its use in other types of contingency tables (for independence and homogeneity) results in very conservative significance probabilities. This correction is no longer needed since exact tests are available.


%-----------------------------------------------------------------------------------------%

\end{document}

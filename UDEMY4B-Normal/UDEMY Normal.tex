\documentclass[a4paper,12pt]{report}
%%%%%%%%%%%%%%%%%%%%%%%%%%%%%%%%%%%%%%%%%%%%%%%%%%%%%%%%%%%%%%%%%%%%%%%%%%%%%%%%%%%%%%%%%%%%%%%%%%%%%%%%%%%%%%%%%%%%%%%%%%%%%%%%%%%%%%%%%%%%%%%%%%%%%%%%%%%%%%%%%%%%%%%%%%%%%%%%%%%%%%%%%%%%%%%%%%%%%%%%%%%%%%%%%%%%%%%%%%%%%%%%%%%%%%%%%%%%%%%%%%%%%%%%%%%%
\usepackage{eurosym}
\usepackage{vmargin}
\usepackage{amsmath}
\usepackage{graphics}
\usepackage{epsfig}
\usepackage{subfigure}
\usepackage{fancyhdr}
\usepackage{listings}
\usepackage{framed}
\usepackage{graphicx}
\usepackage{amsmath}
\usepackage{chngpage}
%\usepackage{bigints}


\setcounter{MaxMatrixCols}{10}
%TCIDATA{OutputFilter=LATEX.DLL}
%TCIDATA{Version=5.00.0.2570}
%TCIDATA{<META NAME="SaveForMode" CONTENT="1">}
%TCIDATA{LastRevised=Wednesday, February 23, 2011 13:24:34}
%TCIDATA{<META NAME="GraphicsSave" CONTENT="32">}
%TCIDATA{Language=American English}

%\pagestyle{fancy}
%\setmarginsrb{20mm}{0mm}{20mm}{25mm}{12mm}{11mm}{0mm}{11mm}
%\lhead{MA4413} \rhead{Mr. Kevin O'Brien}
%\chead{Statistics For Computing}
%\input{tcilatex}




\begin{document}

\title{Statistics Lecture Notes}
\author{Kevin O'Brien}

\tableofcontents \setcounter{tocdepth}{2}


%------------------------------------------------------------------------------------------------%
%------------------------------------------------------------------------------------------------%
\chapter{Introductory Statistics}
\section{Syllabus}


The concept of a random sample, the sampling distribution of the sample mean with applications to confidence intervals, hypothesis testing, and sample size determination, the sampling distribution of the sample proportion with applications to confidence intervals, hypothesis testing, and sample size determination, comparing two means, comparing two proportions, the chi-squared test of independence, Simpson's Paradox, simple linear regression, correlation, residuals.

On successful completion of this module, students should be able to:

\begin{itemize}
\item[1.] Calculate probabilities based on the normal distribution.

\item[2.] Construct and use control charts based on individual measurements, subgroup means and subgroup ranges.

\item[3.] Interpret computer output from common statistical software packages for basic statistical inference procedures such as hypothesis testing and confidence intervals for: a mean, a proportion, difference between independent means, and differences between independent proportions.

\item[4.] Calculate the required sample size for tests of hypothesis and confidence intervals based on a single parameter.

\item[5.] Interpret computer output and diagnostic plots from common statistical software packages for simple linear regression and multiple linear regression.

\item[6.] Test the statistical significance of the difference between several conditional frequency distributions and outline the chi-squared formula used for the test.

\end{itemize}

%------------------------------------------------------------------------------------------------%
%------------------------------------------------------------------------------------------------%


\chapter{Probability Distributions : The Normal Distribution}





\section{Normal Distribution} The normal probability distribution is important in statistical inference for three distinct reasons:

\begin{itemize}
\item[(1)] The measurements obtained in many random processes are known to follow this distribution.
\item[(2)] Normal probabilities can often be used to approximate other probability distributions, such as the
binomial and Poisson distributions.
\item[(3)] Distributions of such statistics as the sample mean and sample proportion are normally distributed
when the sample size is large, regardless of the distribution of the parent population.
\end{itemize}
As is true for any continuous probability distribution, a probability value for a continuous random variable can be determined only for an interval of values. The height of the density function, or probability curve, for a normally distributed variable is given by an integral formula.

%----------------------------------------------------%

\section{The Standard Normal Distribution}

Since every different combination of m and s would generate a different normal
probability distribution, tables of normal probabilities are based on one
particular distribution: the standard normal distribution. This is the normal probability distribution with $\mu=0$
and $\sigma=1$. Any value X from a normally distributed population can be converted into the equivalent standard
normal value Z (i.e. a `Z value') by the formula
\[ Z = \frac{X - \mu}{\sigma}\]

%-------------------------------------------------------------------------%
\section{Calculations}

\begin{itemize}
\item The Complement Rule
\begin{equation}
P(Z \leq A) = 1 - P(Z \geq A)
\end{equation}
\item The Symmetry Rule
\begin{equation}
P(Z \leq -A) = P(Z \geq A)
\end{equation}
\item The Interval Rule.
Where $L$ and $U$ are the lower and upper bounds of an interval.
\begin{equation}
P(L \leq Z \leq U) = P(Z \geq L) -  P(Z \geq U)
\end{equation}

\end{itemize}

\subsection*{Symmetric intervals}
\begin{equation}
P(-A \leq Z \leq A) = 1 - 2\times P(Z \geq A)
\end{equation}


$P(-1.96 \leq Z \leq 1.96) = 1 - 2\times P(Z \geq 1.96) = 1 - (2 \times 0.025) = 0.95$

%--------------------------------------------------------------%

\section{Rules of Thumb}
Additionally, every normal curve (regardless of its mean or standard deviation) conforms to the following "rule".

\begin{itemize}
\item About $68\%$ of the area under the curve falls within 1 standard deviation of the mean.
\item About $95\%$ of the area under the curve falls within 2 standard deviations of the mean.
\item About $99.7\%$ of the area under the curve falls within 3 standard deviations of the mean.
\end{itemize}

Collectively, these points are known as the empirical rule or the 68-95-99.7 rule. Clearly, given a normal distribution, most outcomes will be within 3 standard deviations of the mean.


\section{Worked Example 1}

In an examination the scores of students who attend schools of type A are
normally distributed about a mean of 55 with a standard deviation of 6. The
scores of students who attend type B schools are normally distributed about a
mean of 60 with a standard deviation of 5.

Which type of school would have a higher proportion of students with marks above 70?

\begin{itemize}
\item $\mu_A$ = 55
\item $\sigma_A$ = 8
\item $\mu_B$ = 60
\item $\sigma_B$ = 5
\end{itemize}

We have to fins $P(X_A \geq 70)$
and $P(X_B \geq 70)$.


using the standardisation formula
$Z_A = \frac{70 - 55}{6} = \frac{15}{6} = 2.5 $

$Z_B = \frac{70 - 60}{5} = \frac{10}{5} = 2 $

%----------------------------------------------------------------%
\section{Worked Example 2}
IQ scores are assumed to have a normal distribution with mean 100 and standard deviation 15.

\begin{itemize}
\item What IQ would you have if you were in the 80th percentile?
\item Estimate the threshold for the top 10 percent?
\item What is the probability of having an IQ above 142?
\item What is the probability of having an IQ below 97?
\end{itemize}


%----------------------------------------------------%

\section{Worked Example 3}
Assume that the number of weekly study hours for students at a certain university
is approximately normally distributed with a mean of 22 and a standard deviation
of 6.
\begin{enumerate}
\item Find the probability that a randomly chosen student studies less than 12
hours.
\item Estimate the percentage of students that study more than 37 hours.
\end{enumerate}

\textbf{solution}
$X \sim \mathcal(22,6^2)$  ( in form $X \sim \mathcal(\mu,\sigma^2)$\\

Part 1: $P(X \leq 12)$\\


$Z_1 = \frac{12 - 22}{6} = \frac{-10}{6} = -1.66 $\\

Part 2: $P(X \geq 37)$\\

$Z_2 = \frac{37 - 22}{6} = \frac{15}{6} = 2.5 $\\

%----------------------------------------------------%

\section{Worked Example 4}
The mean is 550kg, with standard deviation 150kg, and we are interested in the area that is greater than 600kg.

\begin{equation}
Z = \frac{ X - \mu }{ \sigma }
\end{equation}

Here X = 600kg,
$\mu$ , the mean = 550kg
$\sigma$, the standard deviation = 150kg
\begin{itemize}
\item $z = ( 600 - 550 ) / 150$
\item $z = 50 / 150$
\item $z = 0.33$
\end{itemize}

Look in the table down the left hand column for z = 0.3, and across under 0.03.
The number in the table is the tail area for z=0.33 which is 0.3707.
This is the probability that the weight will exceed 600kg.

%----------------------------------------------------%

\section{Exercises}

\subsection{Question 1}
Assume that the number of weekly study hours for students at a certain university is approximately normally distributed with a mean of 22 and a standard deviation of 6.

\begin{itemize}
\item[i.] Find the probability that a randomly chosen student studies less than 12 hours.
\item[ii.] Estimate the percentage of students that study more than 37 hours.
\end{itemize}

\subsection{Question 2}
\emph{Taken from MA4104 Business Statistics Examination paper, Spring 2008}\\

Q1. (a) A tyre manufacturer claims that under normal driving conditions, the tread life of a certain tyre follows a normal distribution with mean 50,000 miles and standard deviation 5000 miles.

(i) If your tyres wear out at 45,000 miles, would you consider this unusual? Support your answer with an appropriate probability calculation using the normal curve. [ 10 marks ]

(ii) If the manufacturer sells 100,000 of these tyres and warrants them to last at least 40,000 miles, about how many tyres will wear out before the warranty expires? [ 10 marks ]

%------------------------------------------------------------------------------------------------%
%------------------------------------------------------------------------------------------------%
\chapter{Inference Procedures for Single Samples }

\section{Confidence Intervals for the Mean}
Suppose that you wish to estimate the mean sales amount per
retail outlet for a particular consumer product during the past
year. The number of retail outlets is large. Determine the
95 percent confidence interval given that the sales amounts are
assumed to be normally distributed, $\bar{X} = $3,425, s = $200$ ,
and $n = 25.$\\ Ans. $3;346:60 to $3;503:40
\\
Referring to previous problem, determine the 95 percent
confidence interval given that the population is assumed to be
normally distributed, $\bar{X} = $3,425, s = $200$ , and $n = 25.$
\\Ans. $3;342:44 to $3;507:56

\section{Confidence Interval of a Mean of the Small Sample}

If the data have a normal probability distribution and the sample standard deviation $s$ is used to estimate the population standard deviation $\sigma$, the interval estimate is given by:
\begin{equation}
\bar{X} \pm t_{1-\alpha/2,n-1}\frac{s}{\sqrt{n}}
\end{equation}
where $t_{1-\alpha/2,n-1}$ is the value providing an area of $\alpha/2$ in the upper tail of a Student�s t distribution with n - 1 degrees of freedom.

%----------------------------------------------------%

\section{The Paired $t-$Test}
The mean and standard deviation of the sample $d$ values (i.e.  the population of case-wise differences) are
obtained by use of the basic formulas as seen previously, except
that $d$, is substituted for X.

The mean difference for a set of differences between paired
observations is $\bar{d} = \frac{\sum d_{i}}{n}$.

The deviations formula and the computational formula for the
standard deviation of the differences between paired observations
are, respectively,

\begin{eqnarray}
S_{d} = \sqrt{\frac{\sum (d_{i}-\bar{d})^2}{n-1}}\\
S_{d} = \sqrt{\frac{ \sum (d^2)- n(\bar{d}^2)}{n-1}}\\
\end{eqnarray}

The standard error of the mean difference between paired
observations is obtained for the standard error of the mean.
\subsubsection{Hypotheses}
\begin{eqnarray*}
H_{0}: \mu_{d} = 0\\
H_{1}: \mu_{d} \neq 0\\
\end{eqnarray*}

%----------------------------------------------------%

\section{Sample size Estimation}
For a certain variable, the standard deviation in a large population is equal to 12.5.
How big a sample is needed to be 95\% sure that the sample mean is within 1.5 units of the population mean?


For a certain variable, the standard deviation in a large population is equal to 8.5.
How big a sample is needed to be 90\% sure that the sample mean is within 1.5
units of the population mean?

%------------------------------------------------------------------------------------------------%
%------------------------------------------------------------------------------------------------%

\chapter{Inference Procedures for Two Sample Procedures}

\section{Difference Between Two Population Proportions}
When we wish to test the hypothesis that the proportions in two populations are not different, the two sample proportions are pooled as a basis for determining the standard error of the
difference between proportions.

Note that this differs from the procedure used previously on statistical estimation, in which
the assumption of no difference was not made.

Further, the present procedure is conceptually similar to that presented in Section 11.1, in which the two sample variances are pooled as the basis for computing the standard error of the difference between means.

\subsection{Pooled Estimate for Population Proportion}
The pooled estimate of the population proportion, based on the proportions obtained in two independent samples.

\section{Worked Example : Difference of Two Proportions}
Two time-sharing systems are compared according to their response
time to an editing command. 
\begin{itemize} 
\item The mean response time of 100 requests
submitted to system 1 was measured to be 600 milliseconds with a
known standard deviation of 20 milliseconds. \item  The mean response time
of 100 requests on system 2 was 592 milliseconds with a known
standard deviation of 23 milliseconds. \end{itemize} Using a significance level of $5\%$,
test the hypothesis that system 2 provides a faster response time than
system 1. 

Clearly state your null and alternative hypotheses and your
conclusion.



%------------------------------------------------------------------------------------------------%
%------------------------------------------------------------------------------------------------%

\chapter{Linear Regression Models}

\section{Introduction}
A regression is a statistical analysis assessing the association between two variables. It is used to find the relationship between two variables.

Linear Regression is a statistical technique that correlates the change in a variable (a series of data that recurs at fixed intervals) to other variable/s. The representation of the relationship is called the linear regression model. It is called linear because the relationship is linearly additive. Below is an example of a linear regression model:

\[Y= a + bx + \epsilon\]

\section{Terminology}



\textbf{Variables} Variables are measurements of occurrences of a recurring event taken at regular intervals or measurements of different instances of similar events that can take on different possible values. E.g. the price of gasoline recorded at monthly intervals, the height of children between the age of 6 and 10.

\textbf{Dependent Variable} A variable whose value depends on the value of other variables in a model. E.g. the price of corn oil, which is directly dependent on the price of corn.

\textbf{Independent Variables} Variables whose value is not dependent on other variables in a model. E.g. in the above example, the price of corn would be one of the independent variables driving the price of corn oil. An independent variable is specific to a model and a variable that is independent in one model can be dependent in another.

\textbf{Model}
A system that represents the relationship between variables, both dependent and independent.

%----------------------------------------%

\section{Pearson's Correlation Coefficient}
\[ r_{XY} = \frac{Sxy}{\sqrt{SxSy}} \]

\subsection*{Correlation Coefficient: Guidelines}
The following conclusions are drawn , depending on the correlation coefficient estimate:

\begin{tabular}{|l|l|}
  \hline
Greater than 0.9 	&	Very strong positive linear relationship\\
Between 0.7 and 0.9	&	Strong positive linear relationship\\
Between 0.2 and 0.7	& 	Weak positive linear relationship\\
Between -0.2 and 0.2 &		No relationship\\
Between -0.7 and -0.2 &		Weak negative linear relationship\\
Between -0.9 and -0.7	&	Strong negative linear relationship\\
Less than -0.9	&		Very strong negative linear relationship\\
  \hline
\end{tabular}


Your answer should concur with your interpretation of the scatter-plot.

\section{SLR Diagnostics}

\begin{itemize}
\item Once the model has been fitted, we must check the residuals.
\item The residuals should be independent and normally distributed with
mean of 0 .
\item Use a histogram of the residuals
\item Can also use a Q-Q plot.
\item The residuals should must have constant variance.
\item Use a plot of fitted values vs residuals.
\end{itemize}

%------------------------------------------------------------------------------------------------%
%------------------------------------------------------------------------------------------------%

\end{document}
\documentclass[12pt]{article}

%opening
\title{Introduction to Statistics and Probability}
\author{Kevin O'Brien}

\begin{document}

\maketitle
\newpage
\tableofcontents

% Section 1 Introduction to Statistics
% Section Descriptive Statistics
% Graphical Techniques
% Counting
% Introduction to Probability
% Discrete RVs
% Binomial Distribution

%--------------------------------------------------------------%
\newpage



\section{Normal Probability Distribution}

\subsection{Bell Curve}
Bell curves show up throughout statistics. Diverse measurements such as diameters of seeds, lengths of fish fins, scores on the SAT and weights of individual sheets of a ream of paper all form bell curves when they are graphed. The general shape of all of these curves is the same. But all of these curves are different, because it is highly unlikely that any of them share the same mean or standard deviation. Bell curves with large standard deviations are wide, and bell curves with small standard deviations are skinny. Bell curves with larger means are shifted more to the right than those with smaller means.

\subsection*{Characteristics of the Normal probability distribution}

\begin{enumerate}
\item The highest point on the normal curve is at the mean, which is also the median and mode of the distribution.

\item The normal probability curve is bell-shaped and symmetric, with the shape of the curve to the left of the mean a mirror image of the shape of the curve to the right of the mean.

\item The standard deviation determines the width of the curve. Larger values of the the standard deviation result in wider flatter curves, showing more dispersion in data.

\item The total area under the curve for the normal probability distribution is 1.
\end{enumerate}

\end{document}

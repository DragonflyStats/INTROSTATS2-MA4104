


%----------------------------------------------------------------------% 
% 2000 - Q6
 
The connectors for mobile phones must have a standard deviation of 2mms or less.  
A major mobile company takes a random sample from one of its suppliers as follows.

\[ 
34.2, 33.7, 31.9, 34.3, 31.6, 32.7, 34.1, 35.2, 31.6, 32.9, 33.0, 32.4.
\] 
Based on the data from the sample you are required to 
 

\begin{itemize} 
\item[(i)]   Construct a 95% confidence interval for the population variance.
\item[(ii)]  At the 5\% level of significance is there evidence to support the supplier’s claim that its products are within specification.
\item[(iii)] Using the sample standard deviation as an estimate of the population standard deviation determine the sample size necessary to estimate the mean length of connectors, assuming that a 99% confidence is required with a margin of error of 0.5 mm
\end{itemize}

%-----------------------------------------------------------------------------% 
% 2001 - Q 6 	

A factory that manufactures tyres claims that the nominal depth of the treads is 2mm and from past experience it is known that the standard deviation of such treads is 0.01mm. Assume that the treads of the tyres are distributed Normally. A random sample of 10 tyres is taken and their tread depths (in mm) are found to be 

\begin{verbatim} 
2.68, 2.13, 2.82, 2.71, 2.36, 2.52, 2.29, 2.77, 2.45, 2.39.
\end{verbatim} 
\begin{itemize} 
\item[(a)] Find a point estimate of the mean depth, $\mu$ of the tyre treads. 	
 
\item[(b)] What is the standard error of the point estimate in (a). 			
 
\item[(c)] Find 95\% confidence limits for the mean depth, µ, of tyre treads. Define a 95\%  confidence interval. 						5
 
\item[(d)] Without formal calculation, discuss whether $\mu$ = 2mm is a plausible  hypothesis,   given the results in (c). 						5
 
\item[(e)] Construct a formal test of the null hypothesis 
H0 : $\mu$ = 2mm, against the alternative  H1 : $\mu$ = 2mm. Interpret the result ($\alpha = 0.05$).		 										
 
 \end{itemize}
 
 
 
 
